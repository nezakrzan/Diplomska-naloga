\documentclass[a4paper,12pt]{article}
\usepackage[slovene]{babel}
\usepackage[utf8]{inputenc}
\usepackage[T1]{fontenc}
\usepackage{lmodern}
\usepackage{amsmath}
\usepackage{amsfonts}
\pagestyle{empty}

\begin{document}
Članek Bayes in the Law podrobno opisuje osnove Bayesa za pravno sklepanje in njegove metode v pravnih postopkih. 
Opisano je tudi zakaj ima majhen vpliv oziroma kako so izkušnje z drugimi statističnimi metodami vplivale na standardno 
nasprotovanje uporabi Bayesove metode v pravu. V članku je omenjeno kar nekaj primerov tožb, kjer je bil vključen verjetnosti račun 
pri zaključku tožbe. Našla sem podrobno opisan primer tožbe medicinske sestre Lucie de B. na Nizozemskem, ki naj bi v dveh bolnišnicah, 
kjer je delala, ubila kar nekaj pacientov. Članek, ki sem ga našla, opisuje vpletenost statistike v pravnem primeru, predvsem pa me je pritegnil
opis Bayesovega pristopa k primeru.
\end{document}