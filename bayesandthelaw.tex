\documentclass[a4paper,12pt]{article}
\usepackage[slovene]{babel}
\usepackage[utf8]{inputenc}
\usepackage[T1]{fontenc}
\usepackage{lmodern}
\usepackage{amsmath}
\usepackage{amsfonts}
\pagestyle{empty}

\begin{document}
Članek Bayes in the Law podrobno opisuje osnove Bayesa za pravno sklepanje in njegove metode v pravnih postopkih. 
Opisano je tudi zakaj ima majhen vpliv oziroma kako so izkušnje z drugimi statističnimi metodami vplivale na standardno 
nasprotovanje uporabi Bayesove metode v pravu. V članku je omenjeno kar nekaj primerov tožb, kjer je bil vključen verjetnosti račun 
pri zaključku tožbe. Našla sem podrobno opisan primer tožbe medicinske sestre Lucie de B. na Nizozemskem, ki naj bi v dveh bolnišnicah, 
kjer je delala, ubila kar nekaj pacientov. Članek, ki sem ga našla, opisuje vpletenost statistike v pravnem primeru, predvsem pa me je pritegnil
opis Bayesovega pristopa k primeru.

Po nadaljnem iskanju virov sem naletela še na dve zmoti v pravu. Prva je bila "The Defense Attorney's Fallacy", druga pa "Jury Observation Fallacy". 

Našla sem članek z naslovom "Interpretation of statistical evidence in criminal trails - The Prosecutor's Fallacy and the Defense Attorney's Fallacy" v 
katerem sta predstavljena dva eksperimenta. V prvem eksperimentu so bili podatki predstavljeni na dva načina in sicer kot pogojne verjetnosti ali kot odstotoki, 
pri čemer je bilo ugotovljeno, da so prvi povzorčali več napak pri sklepanju v korist obtožbe, drugi pa v korist obrambe. V drugem eksperimentu pa so bili udeleženci 
izpostavljeni dvema napačnima argumentoma kako interpretirati statistične dokaze. V obeh eksperimentih se je izkazalo, po primerjavi sodb z Bayesovimi normami, da je 
uporabljeno premalo statističnih dokazov.
Po nadaljnem iskanju literature na to temo, sem naletela na ogromno primerov in študij, vendar nikjer nisem zasledila jasne definicije te zmote, še najbolj razumljivo mi je 
bilo napisano v članku "The effectiveness of Bayesain Jury Instructions in the Defense Attorney's Fallacy" in sicer v prvem razdelku "Introduction". 

Ker me je zanimala še druga zmota, "Jury Observation Fallacy", sem našla članek z naslovom "The “Jury Observation Fallacy” and the use of Bayesian Networks to present 
Probabilistic Legal Arguments", v katerem je opisana ta zmota tudi s pomočjo Bayesovih omrežij, kar sem zasledila tudi že v prvem čanku "Bayes in the Law" 
in se mi je zdelo zanimivo.

Ker so do sedaj vsi članki govorilo o določenih zmotah in jih teoretično opisovali ali na primerih, sem si želela še malo bolj raziskati kakšen je pravzaprav 
postopek pri statističnem sklepanju v kazenskem pravu oziroma pravu. Našla sem članek Statistical Reasoning in the Legal Setting, ki zajame kar veliko 
različnih informacij. Opisuje kako statistični strokovnjaki sodelujejo z odvetniki, da se podatki pravilno prikažejo in možnost tudi napačnega analiziranja podatkov.
Na začetku članka The use of statistic in legal proceedings pa sem našla opisano uporabo statističnih in verjetnostnih postopkov v pravi in primere, 
kako se to običajno preloži sodiščem. Obrazloženo je tudi posebaj kako se postopke uporablja pri podajanju mnenj s strani forenzike v kazenski pravi za 
dokaze npr.v sledovih, odtisih.
\end{document}