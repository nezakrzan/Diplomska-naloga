\documentclass{beamer}
\mode<presentation>

\usepackage[utf8]{inputenc}
\usepackage[T1]{fontenc}
\usepackage[slovene]{babel}
\usepackage{lmodern}
\usepackage{array} 
\usepackage{tikz}

\usetheme{Berlin}
\usecolortheme{default}
\useinnertheme[shadows]{rounded}
\useoutertheme{infolines}
\beamertemplatenavigationsymbolsempty

\usepackage{times}
\newcommand{\ds}{\displaystyle}
\newcommand{\ts}{\textstyle}
\newcommand{\presledek}{\vspace{3mm}}
\newcommand{\ph}{\phantom{1}} 
\newcommand{\tr}{{\rm tr \,}}

\newtheorem{definicija}{Definicija}
\newtheorem{izrek}{Izrek}
\newtheorem{trditev}{Trditev}


\begin{document}

%%%%%%%%%%%%%%%%%%%%%%%%%%%%%%%%%%%%%%%%%%%%%%%%%%%%%%%%%%%%%%%%%%%%%%%%%%%%%%%%%%%%%%%%%%%%%%%%%%%%%%%%%%%%%%%%%%%%%%%%%%%%%%%%%%%%%%%%%%%%
%%%%%%%%%%%%%%%%%%%%%%%%%%%%%%%%%%%%%%%%%%%%%%%%%%%%%%%%%%%%%%%%%%%%%%%%%%%%%%%%%%%%%%%%%%%%%%%%%%%%%%%%%%%%%%%%%%%%%%%%%%%%%%%%%%%%%%%%%%%%
\title{Statistika v kazenskem pravu}
\subtitle{Dolga predstavitev diplomske naloge}
\author[Neža Kržan]{Neža Kržan}
 
\institute[FMF]{Mentor: izr. prof. Jaka Smrekar \\ Fakulteta za matematiko in fiziko \\ \vspace{10mm} Ljubljana, 28.november 2022}
\date[28. november 2022] {}

\subject{Talks}

\begin{frame}
   \titlepage
\end{frame}

%%%%%%%%%%%%%%%%%%%%%%%%%%%%%%%%%%%%%%%%%%%%%%%%%%%%%%%%%%%%%%%%%%%%%%%%%%%%%%%%%%%%%%%%%%%%%%%%%%%%%%%%%%%%%%%%%%%%%%%%%%%%%%%%%%%%%%%%%%%%
%%%%%%%%%%%%%%%%%%%%%%%%%%%%%%%%%%%%%%%%%%%%%%%%%%%%%%%%%%%%%%%%%%%%%%%%%%%%%%%%%%%%%%%%%%%%%%%%%%%%%%%%%%%%%%%%%%%%%%%%%%%%%%%%%%%%%%%%%%%%
\begin{frame}
   \frametitle{Kratek pregled}
   \tableofcontents[pausesections]
\end{frame}

%%%%%%%%%%%%%%%%%%%%%%%%%%%%%%%%%%%%%%%%%%%%%%%%%%%%%%%%%%%%%%%%%%%%%%%%%%%%%%%%%%%%%%%%%%%%%%%%%%%%%%%%%%%%%%%%%%%%%%%%%%%%%%%%%%%%%%%%%%%%
%%%%%%%%%%%%%%%%%%%%%%%%%%%%%%%%%%%%%%%%%%%%%%%%%%%%%%%%%%%%%%%%%%%%%%%%%%%%%%%%%%%%%%%%%%%%%%%%%%%%%%%%%%%%%%%%%%%%%%%%%%%%%%%%%%%%%%%%%%%%
\section{Koncept verjetnosti}

\begin{frame}
    \frametitle{Koncept verjetnosti}
    \begin{definicija}
        Naj bo $H$ nek dogodek in $\bar{H}$ negacija oziroma komplement dogodka $H$. Dogodka $H$ in $\bar{H}$ sta znana kot komplementarna dogodka.
    \end{definicija}
    Opravlja se primerjava verjetnosti dokazov na podlagi dveh konkurenčnih predlogov:\\
    $H_p \dots$ trditev, ki jo predlaga tožilstvo;\\
    $H_d \dots$ trditev, ki jo predlaga obramba.\\ \vspace{5mm}
    Koncept verjetnosti je
    \begin{itemize}
        \item ključen pri ocenjevanju dokazov;
        \item omogoča objektivno oceno vpliva dokaza na verjetnost določene domneve o interesni osebi ali obdolžencu.
    \end{itemize}
\end{frame}

%%%%%%%%%%%%%%%%%%%%%%%%%%%%%%%%%%%%%%%%%%%%%%%%%%%%%%%%%%%%%%%%%%%%%%%%%%%%%%%%%%%%%%%%%%%%%%%%%%%%%%%%%%%%%%%%%%%%%%%%%%%%%%%%%%%%%%%%%%%%
\begin{frame}
    \frametitle{Presoja dokazov}
    \begin{itemize}
        \item Za presojo se uporablja različne metode in tehnike, ki temeljijo na statistični verjetnosti,
        \item metode omogočajo oceno, kako verjetno je, da so dokazi resnični in zanesljivi. 
    \end{itemize}
    \begin{beamerboxesrounded}[]{KONTEKST DOKAZA}
        \begin{itemize}
            \item Upoštevamo druge dokaze in okoliščine primera.
            \item Izvedemo bolj objektivno presojo dokazov.
            \item Vpliv na določeno domnevo oziroma hipotezo o interesni osebi ali obdolžencu.
            \item Upoštevamo verjetnost napake - verjetnost, da so dokazi napačni ali zavajajoči.
        \end{itemize} 
    \end{beamerboxesrounded}
\end{frame}

%%%%%%%%%%%%%%%%%%%%%%%%%%%%%%%%%%%%%%%%%%%%%%%%%%%%%%%%%%%%%%%%%%%%%%%%%%%%%%%%%%%%%%%%%%%%%%%%%%%%%%%%%%%%%%%%%%%%%%%%%%%%%%%%%%%%%%%%%%%%
\begin{frame}
    \frametitle{Koncept verjetnosti}
    \begin{beamerboxesrounded}[]{Verjetnost proti nedolžnosti ali verjetnost za krivdo}
        \[
            \frac{P(H_p)}{P(H_d)}
        \]    
    \end{beamerboxesrounded} \vspace{3mm}
    \begin{beamerboxesrounded}[]{Verjetnost v prid krivdi ob upoštevanju informacij $E$}
        \[
            \frac{P(H_p \lvert E)}{P(H_d \lvert E)} 
        \]    
    \end{beamerboxesrounded} \vspace{5mm}
    Če imamo na voljo dokaz $E$, nas zanima pogojna verjetnost
    \[
        P(kriv \lvert E), \vspace{2mm}
    \]
    pri čemer nam je lahko v pomoč Bayesovo pravilo.
\end{frame}

%%%%%%%%%%%%%%%%%%%%%%%%%%%%%%%%%%%%%%%%%%%%%%%%%%%%%%%%%%%%%%%%%%%%%%%%%%%%%%%%%%%%%%%%%%%%%%%%%%%%%%%%%%%%%%%%%%%%%%%%%%%%%%%%%%%%%%%%%%%%
%%%%%%%%%%%%%%%%%%%%%%%%%%%%%%%%%%%%%%%%%%%%%%%%%%%%%%%%%%%%%%%%%%%%%%%%%%%%%%%%%%%%%%%%%%%%%%%%%%%%%%%%%%%%%%%%%%%%%%%%%%%%%%%%%%%%%%%%%%%%
\section{Bayesova statistika}

\begin{frame}
    \frametitle{Opredelitev}
    Bayesovo sklepanje razlaga verjetnost kot merilo verjetnosti ali zaupanja, ki ga lahko ima posameznik glede nastanka določenega dogodka.
    \begin{itemize}
      \item O dogodku lahko že imamo predhodno prepričanje oziroma apriorno prepričanje.
      \item Ta se lahko spremeni, ko se pojavijo novi dokazi.
    \end{itemize} \vspace{3mm}
    Bayesova statistika nam daje matematične modele za vključevanje naših apriornih prepričanj in dokazov za ustvarjanje novih prepričanj.
\end{frame}

%%%%%%%%%%%%%%%%%%%%%%%%%%%%%%%%%%%%%%%%%%%%%%%%%%%%%%%%%%%%%%%%%%%%%%%%%%%%%%%%%%%%%%%%%%%%%%%%%%%%%%%%%%%%%%%%%%%%%%%%%%%%%%%%%%%%%%%%%%%%
\begin{frame}
    \begin{beamerboxesrounded}[]{Bayesova analiza}
        standardna metoda za posodabljanje verjetnosti po opazovanju več \\ dokazov, zato je zelo primerna za sintezo dokazov.
    \end{beamerboxesrounded}\vspace{3mm}
    \begin{itemize}
        \item Začnemo z nekim predhodnim prepričanjem o hipotezi,
        \item posodabljamo to prepričanje, ko se dokazi ponovno pojavijo.
    \end{itemize}\vspace{3mm}
    Pri uporabi Bayesovega sklepanja moramo utemeljiti predhodne predpostavke, kadar je to mogoče.  \\
    V nasprotnem primeru morajo uporabiti razpon vrednosti predpostavk in analizo občutljivosti, da preverijo zanesljivost rezultata.
glede na te vrednosti.
\end{frame}

%%%%%%%%%%%%%%%%%%%%%%%%%%%%%%%%%%%%%%%%%%%%%%%%%%%%%%%%%%%%%%%%%%%%%%%%%%%%%%%%%%%%%%%%%%%%%%%%%%%%%%%%%%%%%%%%%%%%%%%%%%%%%%%%%%%%%%%%%%%%
\begin{frame}
    \frametitle{Bayesovo pravilo}
    Bayesovo sklepanje temelji na Bayesovem pravilu, ki izraža verjetnost nekega dogodka z verjetnostjo dveh dogodkov in obrnejnje pogojne
    verjetnosti. \vspace{3mm}
    \begin{definicija}
        Pogojna verjetnost dogodka H, glede na dogodek E, je
        \begin{equation}\label{eq:pogojna}
             P(H \lvert E) = \frac{P(H \cap E)}{P(E)}, \vspace{2mm}
        \end{equation}
        ob predpostavki, da je $P(E) > 0$.
    \end{definicija}
  \end{frame}

\begin{frame}
    \frametitle{Bayesovo pravilo}
    Potem pogojno verjetnost uporabimo še v števcu formule \eqref{eq:pogojna} in dobimo Bayesovo pravilo
    \[
        P(H \lvert E) = \frac{P(E \lvert H) \times P(H)}{P(E)}, \vspace{2mm}
    \]  
    verjetnost dogodka E lahko še razpišemo
    \[
        P(H \lvert E) = \frac{P(E \lvert H) \times P(H)}{P(E \lvert H)P(H) + P(E \lvert \neg H)P(\neg H)}. \vspace{2mm}
    \] 
    Obstaja še ena formulacija Bayesovega pravila, ki olajša izračune in je pogosto uporabljena pri Bayesovi analizi DNK dokazov
    \[
        \frac{P(H \lvert E)}{P(\neg H \lvert E)} = \frac{P(E \lvert H)}{P(E \lvert \neg H)} \times \frac{P(H)}{P(\neg H)}. \vspace{2mm}
    \]
\end{frame}

%%%%%%%%%%%%%%%%%%%%%%%%%%%%%%%%%%%%%%%%%%%%%%%%%%%%%%%%%%%%%%%%%%%%%%%%%%%%%%%%%%%%%%%%%%%%%%%%%%%%%%%%%%%%%%%%%%%%%%%%%%%%%%%%%%%%%%%%%%%%
\begin{frame}
    \frametitle{Bayesovo posodabljenje}
    Logična trditev, kako se sčasoma posodabljajo apriorne oziroma predhodne verjetnosti dokazov glede na novo zbrane dokaze oziroma prepričanja. \vspace{3mm}
    \begin{trditev}[Bayesovo posodabljanje]
        Če se dogodek E zgodi ob času $t_1 > t_0$, potem je $P_1(H) = P_0(H \lvert E)$.
    \end{trditev}
    \begin{itemize}
        \item Predhodna oz. apriorna verjetnost = $P_0(H)$ (ob času $t_0$ dogodku H dodelimo verjetnost);
        \item Ko se zgodi dogodek E ob času $t_1$, ki vpliva na naša prepričanja o dogodku H, Bayesovo posodabljanje pravi, da je potrebno apriorno 
        verjetnost dogodka H v času $t_1$ enačiti z pogojno verjetnostjo dogodka H glede na dogodek E v času $t_0$. 
    \end{itemize}
\end{frame}

%%%%%%%%%%%%%%%%%%%%%%%%%%%%%%%%%%%%%%%%%%%%%%%%%%%%%%%%%%%%%%%%%%%%%%%%%%%%%%%%%%%%%%%%%%%%%%%%%%%%%%%%%%%%%%%%%%%%%%%%%%%%%%%%%%%%%%%%%%%%
%%%%%%%%%%%%%%%%%%%%%%%%%%%%%%%%%%%%%%%%%%%%%%%%%%%%%%%%%%%%%%%%%%%%%%%%%%%%%%%%%%%%%%%%%%%%%%%%%%%%%%%%%%%%%%%%%%%%%%%%%%%%%%%%%%%%%%%%%%%%
\section{Bayesova analiza}

\begin{frame}
    \frametitle{Predhodna verjetnost in določitev posteriorne verjetnosti}
    Recimo, da je sodni izvedenec je naprošen, da opravi analizo profila DNK krvi, najdene na kraju kaznivega dejanja, in rezultat primerja s profilom DNK 
    obdolženca.\\ \vspace{3mm}
   Odločitev porotnikov bo delno odvisna od njihove ocene dveh interesnih
    hipotez\\
    $H_1 \dots$ vir krvi je obtoženec,\\
    $H_2 \dots$ vir krvi je druga oseba.\\ \vspace{3mm}
    Porotniki bodo morda želeli, da jim dokončno povemo, katera hipoteza je resnična, ali da jim navedemo verjetnosti vira. Za oceno verjetnosti 
    vira mora forenzični znanstvenik upoštevati tudi druge dokaze v kazenskem primeru.
\end{frame}

%%%%%%%%%%%%%%%%%%%%%%%%%%%%%%%%%%%%%%%%%%%%%%%%%%%%%%%%%%%%%%%%%%%%%%%%%%%%%%%%%%%%%%%%%%%%%%%%%%%%%%%%%%%%%%%%%%%%%%%%%%%%%%%%%%%%%%%%%%%%
\begin{frame}
    \frametitle{Predhodna verjetnost in določitev posteriorne verjetnosti}
    Obtoženec in kri s kraja zločina imata skupen niz t.i. genetskih označevalcev - najdemo pri eni osebi na 1 milijon prebivalcev.\\ \vspace{3mm}
    \centering
    $\rightarrow$ \textbf{pogojna verjetnost ugotovitve rezultatov pri dveh hipotezah o medsebojni povezanosti}
    \begin{itemize}
        \item skupni genetski označevalci skoraj zagotovo najdeni v primeru $H_1$ (vir je bil obtoženec);
        \item 1 možnost na milijon, da bodo najdeni v primeru $H_2$ (vir je bil nekdo drug);
    \end{itemize}
    \centering 
    $\rightarrow$ \textbf{predložimo razmerje verjetnosti}
\end{frame}

\begin{frame}
    \frametitle{Predhodna verjetnost in določitev posteriorne verjetnosti}
    \begin{beamerboxesrounded}{}
        Edini skladen način, kako na podlagi forenzičnih dokazov sklepati o \\ verjetnosti virov, je uporaba Bayesovega pravila, ki zahteva, da \\ začnemo s
        pripisom predhodnih verjetnosti za hipoteze, ki nas zanimajo.
    \end{beamerboxesrounded} \vspace{4mm}
    \begin{beamerboxesrounded}{Bayesov pristop}
        Deluje le, če bomo lahko začeli s predhodno oziroma apriorno \\ verjetnostjo.
    \end{beamerboxesrounded}
\end{frame}

\begin{frame}
    \frametitle{Predhodna verjetnost in določitev posteriorne verjetnosti}
    \begin{beamerboxesrounded}{Ali naj analitiki poskušajo določiti predhodne verjetnosti in če ja, kako naj jih določijo?}
    \end{beamerboxesrounded}\vspace{2mm}
    \begin{itemize}
        \item \textbf{Nevtralno stanje} - analitiki predpostavijo enake predhodne verjetnosti za vse hipoteze v primeru; \\ \vspace{2mm}
        \begin{block}{}
            \centering
            PRAKTIČNA IN OBJEKTIVNA ANALIZA
        \end{block} \vspace{3mm}
        \item uporabimo Bayesovo pravilo za izračun posteriorne verjetnosti; \vspace{2mm}
    \end{itemize}
    Predpostavljanje enake verjetnosti za vse možnosti je problematično, kajti v realnosti se različne hipoteze razlikujejo po svoji verjetnosti.
\end{frame}

%%%%%%%%%%%%%%%%%%%%%%%%%%%%%%%%%%%%%%%%%%%%%%%%%%%%%%%%%%%%%%%%%%%%%%%%%%%%%%%%%%%%%%%%%%%%%%%%%%%%%%%%%%%%%%%%%%%%%%%%%%%%%%%%%%%%%%%%%%%%
\begin{frame}
    \frametitle{Poenostavljena Bayesova analiza}
    Naj bo:\\
    $S$ \dots trditev, da je obtoženec vir sledi DNK s kraja zločina; \\
    $M$ \dots trditev, da se obtoženčev DNK ujema z DNK-jem s kraja zločina; \\
    $f$ \dots funkcija pogostosti ujemanja DNK z DNK-jem s kraja zločina. \\
    Želimo vedeti, kakšna je verjetnost S glede na M, tj. $P(S \lvert M)$. \\ \vspace{3mm}
    Bayesovo pravilo lahko uporabimo na naslednji način:
    \[
        \frac{P(S \lvert M)}{P(\neg S \lvert M)} = \frac{P(M \lvert S)}{P(M \lvert \neg S)} \times \frac{P(S)}{P(\neg S)}.
    \] 
\end{frame} 

\begin{frame}
    \frametitle{Poenostavljena Bayesova analiza}
    \[
        \frac{P(S \lvert M)}{P(\neg S \lvert M)} = \frac{P(M \lvert S)}{P(M \lvert \neg S)} \times \frac{P(S)}{P(\neg S)}. \vspace{3mm}
    \] 
    Verjetnosti $P(S)$ in $P(\neg S)$ je težko oceniti, ker ne vemo kakšna je množica osumljencev;
    \begin{block}{}
        \centering
        $P(M \lvert S) = 1$ - \textbf{lažno negativni rezultat};\\ \vspace{2mm}
        $P(M \lvert \neg S) = f$ - \textbf{lažno pozitivni rezultat}; 
    \end{block}
  Ob upoštevanju lažno negativnega in pozitivnega rezultata sledi:
    \[
        \frac{P(S \lvert M)}{P(\neg S \lvert M)} = \frac{1}{f} \times \frac{P(S)}{P(\neg S)}.
    \]
\end{frame}

%%%%%%%%%%%%%%%%%%%%%%%%%%%%%%%%%%%%%%%%%%%%%%%%%%%%%%%%%%%%%%%%%%%%%%%%%%%%%%%%%%%%%%%%%%%%%%%%%%%%%%%%%%%%%%%%%%%%%%%%%%%%%%%%%%%%%%%%%%%%
\begin{frame}
    \frametitle{Izpopolnjena Bayesova analiza l}
    Za upoštevanje možnosti laboratorijskih napak, bomo namesto $M$ uvedli spremenljivko $M_p$.\\ \vspace{3mm}
    $M_p$ \dots poročano ujemanje laboratorijske analize; \\ 
    $M_t$ \dots trditev, da obstaja dejansko ujemanje v DNK-ju;\\
    $\neg M_t$ \dots trditev, da obstaja neujemanje v DNK-ju.  \\ \vspace{3mm}
    Sledi:
    \[
        P(M_p \lvert \neg S) = P(M_p \lvert M_t)P(M_t \lvert \neg S) + P(M_p \lvert \neg M_t)P(\neg M_t \lvert \neg S). 
    \]
\end{frame}

\begin{frame}
    \frametitle{Izpopolnjena Bayesova analiza l}
    \[
        P(M_p \lvert \neg S) = P(M_p \lvert M_t)P(M_t \lvert \neg S) + P(M_p \lvert \neg M_t)P(\neg M_t \lvert \neg S). \vspace{3mm}
    \]
    Sedaj je:
    \begin{itemize}
        \item $P(M_t \lvert \neg S) = f$ in zato
        \item $P(\neg M_t \lvert \neg S) = 1-f$; \vspace{2mm}
    \end{itemize}  
    \begin{block}{}
        $P(M_p \lvert \neg M_t) = FP$ \dots verjetnost lažno pozitivnih rezultatov laboratorija;\\ \vspace{2mm} 
        $P(M_p \lvert M_t) =  FN$  \dots verjetnost lažno negativnih rezultatov laboratorija;     
    \end{block}
  Sledi: \vspace{2mm}
    \[
        P(M \lvert \neg S) = [(1 - FN) \times f] + [FP \times (1 - f)]. \vspace{3mm}
    \]
\end{frame}
 
\begin{frame}
    \frametitle{Izpopolnjena Bayesova analiza ll}
    Predpostavimo, da je 
    \[
        P(M_p \lvert S) = 1, \vspace{2mm}
    \]
    pri čemer ni upoštevana možnost lažnega negativnega rezultata.  \\ \vspace{2mm} 
    \begin{block}{}
        \centering
        $P(M_p \lvert S) = P(M_p \lvert M_t)P(M_t \lvert S) + P(M_p \lvert \neg M_t)P(\neg M_t \lvert S)$
    \end{block}  \vspace{3mm}
    $P(M_p \lvert S) = 1$ $\rightarrow$ $P(\neg M_p \lvert S) = 0$ in $P(M_p \lvert S) = 1 - FN$
    \begin{block}{}
        \centering
        \[ \frac{P(M_p \lvert S)}{P(M_p \lvert \neg S)} = \frac{1 - FN}{[(1 - FN) \times f] + [FN \times (1 - f)]} \]
    \end{block}
\end{frame}

 
\begin{frame}
    \frametitle{Izpopolnjena Bayesova analiza}
    \begin{itemize}
        \item Relativno majhne stopnje napak lahko bistveno zmanjšajo dokazno vrednost DNK dokazov, saj močno zmanjšajo razmerje verjetnosti.
        \item Vpliv stopnje laboratorijskih napak kaže, da ne glede na to, kako nizka se izkaže pogostost profila, bo ta relativno nepomembna, če pogostosti ne spremlja ocena stopnje laboratorijskih napak.
    \end{itemize} \vspace{3mm}
    Spremenljivost pogostosti profila lahko v Bayesovem okviru upoštevamo na dva načina: 
    \begin{enumerate}
        \item s spremembo predhodne verjetnosti;
        \item s spremembo pogostosti profila.
    \end{enumerate}
\end{frame}

%%%%%%%%%%%%%%%%%%%%%%%%%%%%%%%%%%%%%%%%%%%%%%%%%%%%%%%%%%%%%%%%%%%%%%%%%%%%%%%%%%%%%%%%%%%%%%%%%%%%%%%%%%%%%%%%%%%%%%%%%%%%%%%%%%%%%%%%%%%%
%%%%%%%%%%%%%%%%%%%%%%%%%%%%%%%%%%%%%%%%%%%%%%%%%%%%%%%%%%%%%%%%%%%%%%%%%%%%%%%%%%%%%%%%%%%%%%%%%%%%%%%%%%%%%%%%%%%%%%%%%%%%%%%%%%%%%%%%%%%%
\begin{frame}
    \frametitle{}
\end{frame}

\end{document}