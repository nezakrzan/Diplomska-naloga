\documentclass[12pt,a4paper]{amsart}
\usepackage[slovene]{babel}
%\usepackage[cp1250]{inputenc}
\usepackage[T1]{fontenc}
\usepackage[utf8]{inputenc}
\usepackage{amsmath,amssymb,amsfonts}
\usepackage{url}
\usepackage[normalem]{ulem}
\usepackage[dvipsnames,usenames]{color}
\usepackage{graphicx}

% Oblika strani
\textwidth 15cm
\textheight 24cm
\oddsidemargin.5cm
\evensidemargin.5cm
\topmargin-5mm
\addtolength{\footskip}{10pt}
\pagestyle{plain}
\overfullrule=15pt % oznaci predlogo vrstico

% Ukazi za matematična okolja
\theoremstyle{definition} % tekst napisan pokončno
\newtheorem{definicija}{Definicija}[section]
\newtheorem{primer}[definicija]{Primer}
\newtheorem{opomba}[definicija]{Opomba}

\renewcommand\endprimer{\hfill$\diamondsuit$}


\theoremstyle{plain} % tekst napisan poševno
\newtheorem{lema}[definicija]{Lema}
\newtheorem{izrek}[definicija]{Izrek}
\newtheorem{trditev}[definicija]{Trditev}
\newtheorem{posledica}[definicija]{Posledica}

\begin{document}

%%%%%%%%%%%%%%%%%%%%%%%%%%%%%%%%%%%%%%%%%%%%%%%%%%%%%%%%%%%%%%%%%%%%%%%%%%%%%%%%%%%%%%%%%%%%%%%%%%%%%%%%%%%%%%%%%%%%%%%%%%%%%%%%%%%%%%%%%%%%%%
\title{Drugi pristopi}
\author{Neža Kržan}
\maketitle

%%%%%%%%%%%%%%%%%%%%%%%%%%%%%%%%%%%%%%%%%%%%%%%%%%%%%%%%%%%%%%%%%%%%%%%%%%%%%%%%%%%%%%%%%%%%%%%%%%%%%%%%%%%%%%%%%%%%%%%%%%%%%%%%%%%%%%%%%%%%%%
%%%%%%%%%%%%%%%%%%%%%%%%%%%%%%%%%%%%%%%%%%%%%%%%%%%%%%%%%%%%%%%%%%%%%%%%%%%%%%%%%%%%%%%%%%%%%%%%%%%%%%%%%%%%%%%%%%%%%%%%%%%%%%%%%%%%%%%%%%%%%%
\section{Drugi pristopi}
Da bi ocenili moč Bayesove analize dokazov DNK, jo je koristno primerjati z nekaterimi drugimi pristopi.

%%%%%%%%%%%%%%%%%%%%%%%%%%%%%%%%%%%%%%%%%%%%%%%%%%%%%%%%%%%%%%%%%%%%%%%%%%%%%%%%%%%%%%%%%%%%%%%%%%%%%%%%%%%%%%%%%%%%%%%%%%%%%%%%%%%%%%%%%%%%
\subsection{Frekvence}
Predlagano je bilo s strani mnogih avtorjev, da je bolj naraven način za obravnavo verjetnosti uporaba naravnih frekvenc(angl. natural 
frequencies). 
\begin{definicija}
    Frekvenca $f$ je posamezno število diskretnih statističnih enot iste vrednosti. Če je diskretnih podatkov zelo veliko ali če so 
    podatki zvezni, jih združujemo v frekvenčne razrede.
\end{definicija}

\begin{definicija}
    Absolutna frekvenca (oznaka $f_k$ za k-ti razred) je število vrednosti statistične spremenljivke v k-tem razredu.
\end{definicija}

\begin{definicija}
    Relativna frekvenca (oznaka $f_k'$) pa je delež absolutne frekvence $f_k$ glede na celoto. Če je $N$ število enot v populaciji 
    ali morda vzorcu, je 
    \[
        f_k' = \frac{f_k}{N}.
    \]
\end{definicija}
V forenzičnem kontekstu se navedene frekvence običajno nanašajo na pojavljanje dokazov za posamezen primer, medtem ko se frekvence za 
pojavljanje vprašanj običajno opisujejo kot osnovne stopnje(angl. base rates). Sklepanje o krivdi je lahko podprto s statistično analizo 
ustreznih podatkov in verjetnostnim sklepanjem z uporabo absolutnih ali relativnih frekvenc, pri čemer je verjetnost, da bi določene 
podatke (dokaze) pridobili zgolj po naključju, izjemno majhna . Relativne frekvence vedno navajajo ali predpostavljajo, da obstaja nek 
referenčni vzorec, na podlagi katerega se lahko oceni pogostost zadevnega dogodka. Nadaljnja predpostavka je, da je ta primerjava poučna 
in pomembna za obravnavano nalogo. V okviru kazenskega postopka bi na primer pričakovali, da bo relativna pogostost lahko podprla 
vmesno sklepanje o moči dokazov, ki se nanašajo na sporna dejstva, kar vodi do končnega sklepa, da je obtoženec nedolžen ali kriv. Relativne 
frekvence so rutinsko vključene v znanstvene dokaze, ki se predložijo v kazenskih postopkih.\\\\
Recimo, da je pogostost profila DNK $f$ 1 proti 10 milijonov, predpostavimo tudi, da ima obtoženec enak DNK in da je začetna populacija osumljencev 
100 milijonov ljudi. Zanima nas kakšna je verjetnost, da je obtoženec vir DNK-ja s kraja zločina. \\
Naj bo: \\
$f$ \dots pogostost profila DNK;\\
$m$ \dots velikost populacije osumljencev; \\
$n$ \dots število ljudi, ki imajo ustrezen DNK profil. \\
Po metodah, ki temeljijo na naravnih frekvencah, je potrebno izračunati, koliko ljudi z zadevnim profilom DNK je v populaciji osumljencem, tako 
da se $f$ pomnoži z $m$.  \\
Če je posameznikov s takim profilom $n > 1$, je verjetnost, da je obtoženi vir, $\frac{1}{n}$. V primeru, ko je $n < 1$ in so frekvence še 
posebaj majhne, je bolje raziskovati ali je profil DNK edinstven ali ne - ali poleg obtoženca obstajajo še drugi posamezniki z enakim 
profilom DNK. Temu pravimo metoda edinstvenosti(angl. uniqueness method). \\
S formulo binomske porazdelitve lahko izračunamo verjetnost, da se bo dogodek $X$, na primer profil DNK, pojavil $k$ - krat v $s$ - kratnem 
številu ponovitev, pri čemer ima dogodek $X$ frekvenco 4$f$. Želimo vedeti, kolikšna je verjetnost, da ima točno en posameznik ustrezen DNK 
profil, ob pogoju da ga ima vsaj en posameznik, torej:
\[
    P(n = 1 \lvert n \ge 1) = \frac{P(n=1 \cap n \ge 1)}{P(n \ge 1)} = \frac{m \times f \times (1 - f)^{m-1}}{1 - (1 - f)^m}.
\]

\begin{table}[h!]
    \centering
    \begin{tabular}{c c c c} 
        \hline
        $m$ & $f$ & $P(n = 1 \lvert n \ge 1)$ & $P(S \lvert M)$[Bayes]\\ 
        \hline
        10 milijonov & 1 v 100 milojonov & 0,9 & 0,9 \\
        100 milijonov & 1 v 100 milojonov & 0,62 & 0,5 \\
        1 milijarda & 1 v 100 milojonov & 0 & 0,09 \\ \hline
        10 milijonov & 1 v 1 milijardi & 1 & 0,99 \\
        100 milijonov & 1 v 1 milijardi & 0,9 & 0,9 \\
        1 milijarda & 1 v 1 milijardi & 0,62 & 0,5 \\ \hline
        10 milijonov & 1 v 10 milijardah & 1 & 0,999 \\
        100 milijonov & 1 v 10 milijardah & 1 & 0,99 \\
        1 milijarda & 1 v 10 milijardah & 0,9 & 0,9 \\ \hline
    \end{tabular}
\end{table}
Tabela vsebuje primerjavo rezultatov, ki jih daje Bayesovo pravilo. Obe metodi se dokaj ujemata, vendar se tudi razlikujeta, saj v nasprotju 
z Bayesovim pravilom pri metodi edinstvenosti ni mogoče enostavno upoštevati številnih zapletov, kot je bil naprimer vpliv stopnje 
laboratorijskih napak. 

%%%%%%%%%%%%%%%%%%%%%%%%%%%%%%%%%%%%%%%%%%%%%%%%%%%%%%%%%%%%%%%%%%%%%%%%%%%%%%%%%%%%%%%%%%%%%%%%%%%%%%%%%%%%%%%%%%%%%%%%%%%%%%%%%%%%%%%%%%%%
\subsection{Naključno ujemanje}
Metoda verjetnost naključnega ujemanja(angl. random match probability) izraža možnost, da bi imel naključni posameznik, ki ni povezan z obdolžencem, 
ustrezni DNK profil. Ta verjetnost je enaka pogostosti profila DNK. Težava tega pristopa je, da verjetnost naključnega ujemanja lahko predstavljena 
oziroma razumevana narobe. \\
Pogosto se to verjetnost interpretira na slednji način:
\begin{enumerate}
    \item če je verjetnost naključnega ujemanja na primer 1 proti 100 milijonom, potem je verjetnost, da ima profil DNK drug posameznik in ne 
    obdolženec 1 proti 100 milijonom;
    \item ker je to zelo majhna verjetnost, mora biti tudi verjetnost, da je sled DNK pustil nekdo drug na kraju zločina in ne obdolženec, zelo majhna;
    \item zato mora biti verjetnost, da je vir sledi DNK s kraja zločina obtoženec zelo velika, ampak znaša 1 proti 100 milijonov.
\end{enumerate}
Takšno sklepanje je napačno in je znano kot tožilčeva zmota(angl. Prosecutor’s fallacy). Sestavlja jo enačba
\begin{equation}\label{eq:tozilcevazmota}
    1 - f = P(S \lvert M).
\end{equation}
Zmota se pojavi v koraku (2), ko je zamenjano $P(M \lvert \neg S)$ s $P(\neg S \lvert M)$ in predpostavljeno, da sta obe verjetnosti enaki $f$. \\
Namesto verjetnosti naključnega ujemanja forenzični strokovnjaki pogosto pričajo o razmerju verjetnosti(angl. likelihood ratio) dokazov DNK, in 
sicer kot:
\[
    P(M \lvert S) = P(M \lvert \neg S).
\]
Bayesovo pravilo vključuje razmerje verjetnosti in predhodno oziroma pariorno verjetnost, torej je celovitejši način predstavitve dokazov DNK. 
Težavo ocenjevanja apriorne verjetnosti pri Bayesovem pravilu, bi lahko odpravili z osredotočanjem le na verjetnost. Metoda razmerja verjetnosi je 
koristna v državah kot sta naprimer Velika Britanija in Združene države Amerike, kjer se lahko Bayesovo pravilo šteje kot poseg v pravico porote 
do previdnosti: poroti naj ne bi bilo potrebno govoriti, kako naj razmišlja in presoja dokaze; Bayesovo pravilo pa je natanko metoda za 
tehtanje dokazov.

%%%%%%%%%%%%%%%%%%%%%%%%%%%%%%%%%%%%%%%%%%%%%%%%%%%%%%%%%%%%%%%%%%%%%%%%%%%%%%%%%%%%%%%%%%%%%%%%%%%%%%%%%%%%%%%%%%%%%%%%%%%%%%%%%%%%%%%%%%%%
\subsection{Predhodna verjetnost(angl. Prior Probability)}
Sodni izvedenec je naprošen, da opravi analizo profila DNK krvi, najdene na kraju kaznivega dejanja, in rezultat primerja s profilom DNK 
obdolženca. O krivdi ali nedolžnosti obtoženca bo odločala porota. Odločitev porotnikov bo delno odvisna od njihove ocene dveh interesnih 
možnosti\\
$H_1 \dots$ vir krvi je obtoženec,\\
$H_2 \dots$ vir krvi je druga oseba.\\
Porotniki bodo morda želeli, da jim izvedenec dokončno pove, katera hipoteza je resnična, ali da jim navede posebne vrednosti tako imenovanih 
verjetnosti vira. Za oceno verjetnosti vira mora forenzični znanstvenik upoštevati tudi druge dokaze v kazenskem primeru.\\\\
Recimo, da je izvedenec ugotovil, da imata obtoženec in kri s kraja zločina skupen niz genetskih označevalcev, ki jih najdemo pri eni osebi 
na 1 milijon prebivalcev v zadevni populaciji. Ne da bi upošteval druge dokaze v zadevi, lahko izvedenec poda izjavo o pogojni verjetnosti 
ugotovitve teh rezultatov pri dveh hipotezah o medsebojni povezanosti. Izvedenec lahko na primer izjavi, da so skupni genetski označevalci 
skoraj zagotovo najdeni v primeru H1 (vir je bil obtoženec), vendar imajo le 1 možnost na milijon, da bodo najdeni v primeru $H_2$ (vir je bil 
nekdo drug). Na podlagi te ocene lahko izvedenec poroti predloži razmerje verjetnosti - na primer, da so rezultati profiliranja DNK 1 milijonkrat 
bolj verjetni, če je bil vir krvi obtoženec in ne neka druga oseba. Vendar razmerje verjetnosti ni isto kot verjetnost vira. \\
Edini skladen način, kako na podlagi forenzičnih dokazov sklepati o verjetnosti virov, je uporaba Bayesovega pravila, ki zahteva, da začnemo s 
pripisom predhodnih verjetnosti za trditve, ki nas zanimajo. Bayesov pristop bo deloval le, če bo izvedenec lahko začel s predhodno verjetnostjo.\\\\
To nas pripelje do bistva vprašanja: ali naj forenziki sploh poskušajo določiti predhodne verjetnosti, in če da, kako. Občasno se predlaga, da 
bi morali forenziki predpostavljati enake predhodne verjetnosti, kar se včasih opisuje kot stališče nevtralnosti. Torej analitiki predpostavljajo, 
da sta predhodni verjetnosti $H_1$ in $H_2$ enaki, nato pa ju v skladu z Bayesovim pravilom pomnožijo z razmerjem verjetnosti, da določijo posteriorno 
verjetnost. \\\\
Ampak kakršnakoli privzeta predpostavka o predhodni verjetnosti je obravnavana kot kršitev obveznosti pravnega sistema, da zagotovi individualizirano 
pravico na podlagi dejstev vsakega primera, zato je drugi predlagani pristop, da forenzični znanstveniki prevzamejo odgovornost za oceno predhodne 
verjetnosti ustreznih hipotez, preden jih posodobijo na podlagi znanstvenih ugotovitev v skladu z Bayesovim pravilom. Glavni očitek temu pristopu v 
okviru kazenskega postopka je, da lahko forenzični znanstveniki presežejo svoje znanstveno znanje in si prisvojijo vlogo tistega, ki ugotavlja 
dejstva. Da bi izvedenec določil predhodne kontekstualno smiselne verjetnosti, bi moral upoštevati vse dokaze v zadevi. Zato je bilo predlagano, 
da sodni izvedenci nimajo nobene vloge pri ocenjevanju predhodnih verjetnosti. \\\\
Vprašanje, ali naj forenzični znanstveniki upoštevajo predhodno verjetnost hipotez, ki naj bi jih pomagali ovrednotiti, je zapleteno. Odgovor je odvisen od 
vloge, ki jo bo forenzični znanstvenik imel v pravnem sistemu. Če bodo forenzični znanstveniki za pravne namene sprejemali ultimativne ugotovitve v zvezi z 
določenim predlogom, ki jih zanima, potem bi morali in dejansko morajo upoštevati predhodne verjetnosti, da so hipoteze resnične. Če pa bo o 
resničnosti hipotez odločal nekdo drug, npr. sodnik ali porota, in je vloga forenzičnih znanstvenikov omejena na zagotavljanje strokovne pomoči, se 
morajo forenzični znanstveniki na splošno omejiti na določanje pogojne verjetnosti znanstvenih ugotovitev pri danih hipotezah, ki jih zanimajo, nalogo 
ocenjevanja predhodnih in poznejših verjetnosti pa morajo prepustiti nosilcu pravne odločitve.

\end{document}