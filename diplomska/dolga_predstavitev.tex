\documentclass[12pt,a4paper]{amsart}
\usepackage[slovene]{babel}
%\usepackage[cp1250]{inputenc}
\usepackage[T1]{fontenc}
\usepackage[utf8]{inputenc}
\usepackage{amsmath,amssymb,amsfonts}
\usepackage{url}
\usepackage[normalem]{ulem}
\usepackage[dvipsnames,usenames]{color}
\usepackage{graphicx}

% Oblika strani
\textwidth 15cm
\textheight 24cm
\oddsidemargin.5cm
\evensidemargin.5cm
\topmargin-5mm
\addtolength{\footskip}{10pt}
\pagestyle{plain}
\overfullrule=15pt % oznaci predlogo vrstico

% Ukazi za matematična okolja
\theoremstyle{definition} % tekst napisan pokončno
\newtheorem{definicija}{Definicija}[section]
\newtheorem{primer}[definicija]{Primer}
\newtheorem{opomba}[definicija]{Opomba}

\renewcommand\endprimer{\hfill$\diamondsuit$}


\theoremstyle{plain} % tekst napisan poševno
\newtheorem{lema}[definicija]{Lema}
\newtheorem{izrek}[definicija]{Izrek}
\newtheorem{trditev}[definicija]{Trditev}
\newtheorem{posledica}[definicija]{Posledica}

\begin{document}
%%%%%%%%%%%%%%%%%%%%%%%%%%%%%%%%%%%%%%%%%%%%%%%%%%%%%%%%%%%%%%%%%%%%%%%%%%%%%%%%%%%%%%%%%%%%%%%%%%%%%%%%%%%%%%%%%%%%%%%%%%%%%%%%%%%%%%%%%%%%%%
\title{Statistika v kazenskem pravu}
\author{Neža Kržan}
\maketitle
Statistične metode so temelj kazenskega pravosodja in kriminologije kot področij znanstvenega raziskovanja. Statistika omogoča oblikovanje in 
širjenje znanja o kriminaliteti in kazenskopravnem sistemu. Raziskave, ki preverjajo teorije ali preučujejo pojave v kazenskem pravosodju in so 
objavljene v znanstvenih revijah in knjigah, so podlaga za večino tega, kar vemo o kaznivih dejanjih in sistemu, ki je bil oblikovan za njihovo 
obravnavanje. Večina teh raziskav ne bi bila mogoča brez statističnih podatkov.\\\\
Raziskave na področju kazenskega pravosodja in kriminologije so različne po naravi in namenu. Velik del raziskav vključuje preverjanje teorije. 
Teorije so predlagane razlage določenih dogodkov. Hipoteze so majhni "deli" teorij, ki morajo biti resnični, da bi celotna teorija držala. Teorijo 
si lahko predstavljamo kot verigo, hipoteze pa kot člene, ki sestavljajo to verigo.\\\\
Raziskovalci na področju kazenskega pravosodja in kriminologije si običajno prizadevajo preučiti razmerja med dvema ali več spremenljivkami. Opazovani 
ali empirični pojavi sprožajo vprašanja o osnovnih silah, ki jih poganjajo. Vzemimo za primer umor. Empirični pojavi so umori in stopnje umorov 
na ravni mesta. endar pa raziskovalci običajno želijo več kot le zapisati empirične ugotovitve - želijo vedeti, zakaj so stvari takšne, kot so. Zato 
lahko poskušajo ugotoviti kriminogene dejavnike, ki so prisotni.\\\\
Raziskovalci, ki izvajajo kvantitativne študije, morajo določiti odvisne spremenljivke (DV) in neodvisne spremenljivke (IV). Odvisne spremenljivke 
so empirični dogodki, ki jih želi raziskovalec pojasniti. Primeri DV so stopnje umorov, stopnje premoženjskih kaznivih dejanj, povratništvo 
med nedavno izpuščenimi zaporniki in sodne odločitve o izreku kazni. Raziskovalci skušajo opredeliti spremenljivke, ki pomagajo napovedati ali 
pojasniti te dogodke. Neodvisne spremenljivke so dejavniki, za katere raziskovalec meni, da bi lahko vplivali na DV. Določitev IV in DV je 
odvisna od narave raziskovalne študije.
\begin{definicija}
    Odvisna spremenljivka(DV) je pojav, ki ga želi raziskovalec preučiti, razložiti ali napovedati.\\
    Neodvisna spremenljivka(IV) je dejavnik ali značilnost, s katero se poskuša pojasniti ali napovedati odvisno spremenljivko.
\end{definicija}

Pomembno je razumeti, da neodvisno in odvisno nista sinonima za vzrok in posledico. Določen IV je lahko povezan z določenim DV, vendar to še 
zdaleč ni dokončen dokaz, da je prvi vzrok drugega. Za dokazovanje vzročnosti morajo raziskovalci dokazati, da njihove študije izpolnjujejo 
tri merila. Prvo je časovno zaporedje, kar pomeni, da se mora IV pojaviti pred DV. Druga zahteva glede vzročnosti je, da obstaja empirična 
povezava med IV in DV. To je osnovna potreba - ni smiselno, da se poskušamo poglobiti v nianse neobstoječe povezave med dvema spremenljivkama. 
Zadnja zahteva je, da je razmerje med IV in DV nepristransko. To tretje merilo je v kriminologiji in kazenskopravnih raziskavah (pravzaprav v vseh 
družbenih vedah) pogosto najtežje premagati, saj je človeško vedenje zapleteno in ima vsako dejanje, ki ga oseba stori, več vzrokov. Razmejitev 
teh vzročnih dejavnikov je lahko težavna ali nemogoča. Razlog, zakaj je nepristranskost problem, je, da lahko obstaja tretja spremenljivka, 
ki pojasnjuje DV enako dobro ali celo bolje kot IV. Ta tretja spremenljivka lahko delno ali v celoti pojasni razmerje med IV in DV. Nenamerna 
izključitev ene ali več pomembnih spremenljivk lahko privede do napačnih zaključkov, saj lahko raziskovalec zmotno verjame, da IV močno 
napoveduje DV, medtem ko je v resnici razmerje dejansko delno ali v celoti posledica posrednih dejavnikov. Drug izraz za to težavo je 
pristranskost izpuščenih spremenljivk. Kadar je pristranskost izpuščenih spremenljivk (tj. spuriousness) prisotna v razmerju IV-DV, vendar 
se napačno ne prepozna, lahko ljudje pridejo do napačnega sklepa o pojavu. Torej vse vrste spremenljivk so med seboj povezane, vendar je 
pomembno, da na podlagi statističnih povezav ne prehitro sklepamo o vzročnih posledicah.\\\\

%%%%%%%%%%%%%%%%%%%%%%%%%%%%%%%%%%%%%%%%%%%%%%%%%%%%%%%%%%%%%%%%%%%%%%%%%%%%%%%%%%%%%%%%%%%%%%%%%%%%%%%%%%%%%%%%%%%%%%%%%%%%%%%%%%%%%%%%%%%%%%
%%%%%%%%%%%%%%%%%%%%%%%%%%%%%%%%%%%%%%%%%%%%%%%%%%%%%%%%%%%%%%%%%%%%%%%%%%%%%%%%%%%%%%%%%%%%%%%%%%%%%%%%%%%%%%%%%%%%%%%%%%%%%%%%%%%%%%%%%%%%%%
\section{Učinkovitost Bayesove teorije pri zmanjšanju zmote obrambnega odvetnika}
Zmota odvetnika je manj znana različica pogostejše tožilske zmote. Zmota odvetnika je različica tožilčeve zmote, pri kateri je porota spodbujena, 
da asociativnih dokazov ne upošteva kot nepomembnih, ne glede na to, kako redka je značilnost ujemanja.2 Za ponazoritev si predstavljajte 
sojenje z enakim ujemanjem krvi kot prej, vendar z dodatnimi dokazi proti obtožencu, kot sta očividec in posedovanje orodja za kaznivo dejanje. 
Zagovornik lahko kljub temu poroti pove, da je verjetnost obtoženčeve krivde le 1 proti 1000. To bi bilo sicer pravilno, če bi obtožnica utemeljevala 
primer samo na podlagi ujemanja krvi, vendar obrambni odvetnik dejansko prosi poroto, naj ne upošteva vseh dokazov razen ujemanja krvi. To je zmota 
obrambnega odvetnika. Če obrambnemu odvetniku uspe prepričati poroto, da verjame tej napačni logiki, lahko dodatni dokazi o ujemanju krvi napačno 
zmanjšajo občutek krivde pri poroti.\\
Čeprav so učinki tožilčeve zmote lahko hujši kot učinki zmote odvetnika (krivična obsodba v primerjavi s krivično oprostitvijo), so porote morda 
bolj dovzetne za slednjo kot za prvo. 


%%%%%%%%%%%%%%%%%%%%%%%%%%%%%%%%%%%%%%%%%%%%%%%%%%%%%%%%%%%%%%%%%%%%%%%%%%%%%%%%%%%%%%%%%%%%%%%%%%%%%%%%%%%%%%%%%%%%%%%%%%%%%%%%%%%%%%%%%%%%%%
%%%%%%%%%%%%%%%%%%%%%%%%%%%%%%%%%%%%%%%%%%%%%%%%%%%%%%%%%%%%%%%%%%%%%%%%%%%%%%%%%%%%%%%%%%%%%%%%%%%%%%%%%%%%%%%%%%%%%%%%%%%%%%%%%%%%%%%%%%%%%%
\section{Zasliševalčeva zmota}
To zmoto opredeljujemo kot zmoto, ki se kaže v tem, da izpovedni dokazi nikoli ne morejo zmanjšati verjetnosti krivde - v določenih okoliščinah 
lahko priznanje zmanjšal verjetnost krivde. Vprašanje je, kdaj natančno je priznanje znak krivde in kdaj ne. Priznanje poveča verjetnost krivde 
le, kadar je verjetnost priznanja, če je oseba kriva, večja od verjetnosti priznanja, če je oseba nedolžna. Če se $G$ nanaša na trditev, da je 
oseba kriva, $C$ pa na trditev, da oseba prizna, velja
\[
    P(G \lvert C) > P(G \lvert \neg C)
\]  
velja le pod pogojem, da je 
\[
    P(C \lvert G) > P(C \lvert \neg G).
\] 
\begin{trditev}
    Obstoj priznanja poveča verjetnost krivde, če in samo če je manj verjetno, da bo nedolžna oseba priznala kot kriva.
\end{trditev}
Slednjo trditev je mogoče izpeljati s pomočjo Bayesove teorije:
\begin{equation}
    \frac{P(G \lvert C)}{P(\neg G \lvert C)} = \frac{P(G)}{P(\neg G)}  \times \frac{P(C \lvert G)}{P(C \lvert \neg G)}
\end{equation}
Na levi strani je $\frac{P(G \lvert C)}{P(\neg G \lvert C)}$, razmerje posteriornih verjetnosti krivde in nedolžnosti (tj. verjetnosti po 
upoštevanju priznanja). Enak je zmnožku dveh drugih razmerij na desni strani. Razmerje $\frac{P(G)}{P(\neg G)}$ predstavlja razmerje predhodnih 
verjetnosti krivde in nedolžnosti (to je verjetnosti krivde in nedolžnosti pred upoštevanjem dejstva priznanja). Zadnje razmerje, 
$\frac{P(C \lvert G)}{P(C \lvert \neg G)}$, se imenuje razmerje verjetnosti in določa, kakšen vpliv bo imel nov dokaz (priznanje) na 
verjetnosti krivde in nedolžnosti. Zlasti bo posteriorna verjetnost krivde večja od predhodne verjetnosti krivde le, če bo razmerje verjetnosti 
večje od 1. Če pa je razmerje verjetnosti $\frac{P(C \lvert G)}{P(C \lvert \neg G)} < 1$, priznanje dejansko postane kazalnik nedolžnosti. \\\\
Ni nujno, da iz vsega tega sledi, da bi bil obstoj priznanja v tej situaciji dokaz proti krivdi. V nasprotju s tem, bi lahko priznanje še vedno 
povečalo verjetnost krivde, tudi če je bolj verjetno, da bo nedolžna oseba priznala kot kriva. Takšna zamisel se zdi logično nemogoča, če 
pogledamo (1). Vendar želim povedati, da enačba (1) morda ni prava formula za uporabo v tem primeru. Morda obstaja še eno empirično dejstvo, 
ki je pomembno za verjetnost krivde. V tem primeru bi morali uporabiti bolj zapleteno formulo, ki bi vključevala to dejstvo. Ker tu preučujemo 
dokazni učinek priznanja, ki je rezultat zaslišanja, verjetnost, ki jo iščemo, v resnici ni verjetnost, da je oseba X kriva, če je priznala, 
ampak verjetnost, da je oseba X kriva, če je priznala in če je bil zaslišana. Zato je treba razmerje posteriornih verjetnosti izraziti na 
naslednji bolj celovit način, pri čemer $I$ predstavlja trditev "je bil zaslišan":
\begin{equation}
    \frac{P(G \lvert C, I)}{P(\neg G \lvert C, I)} = \frac{P(G \lvert I)}{P(\neg G \lvert I)}  \times \frac{P(C \lvert G, I)}{P(C \lvert \neg G, I)}
\end{equation}
in z nadaljnjo razširitvijo prvega razmerja verjetnosti na desni strani (2)
\begin{equation}
    \frac{P(G \lvert C, I)}{P(\neg G \lvert C, I)} = \frac{P(G)}{P(\neg G)} \times \frac{P(I \lvert G)}{P(I \lvert \neg G)}  \times \frac{P(C \lvert G, I)}{P(C \lvert \neg G, I)}
\end{equation}
Da bi videli, zakaj je prvo razmerje na desni strani (2) enakovredno produktu prvih dveh razmerij na desni strani (3), najprej upoštevajte naslednji dve osnovni verjetnostni resnici:
\begin{equation}
    P(G \lvert I) = \frac{P(G) \times P(I \lvert G)}{P(I)}
\end{equation}
\begin{equation}
    P(\neg G \lvert I) = \frac{P(\neg G) \times P(I \lvert \neg G)}{P(I)}
\end{equation}
Če zdaj (4) delim z (5) dobim
\begin{equation}
    \frac{P(G \lvert I)}{P(\neg G \lvert I)} = \frac{P(G)}{P(\neg G)}  \times \frac{P(I \lvert G)}{P(I \lvert \neg G)}
\end{equation}
Enačba (6) dokazuje, da sta (2) in (3) enakovredni.\\\\
Primerjajmo enačbo (1) z enačbo (3), ki daje popolnejšo sliko stanja. Na levi strani je v vsakem primeru razmerje posteriornih verjetnosti krivde in nedolžnosti, s to 
razliko, da so v (1) te verjetnosti pogojene samo s priznanjem, medtem ko so v (3) pogojene tako s priznanjem kot z zaslišanjem. Prvo razmerje na desni strani je enako 
tako v (1) kot (3): razmerje predhodnih verjetnosti $G$ in $\neg G$. Tudi zadnje razmerje na desni strani je v obeh primerih podobno: verjetnost priznanja glede na 
krivdo (in zaslišanje), deljena z verjetnostjo priznanja glede na nedolžnost (in zaslišanje). V enačbi (3) imamo na desni strani razmerje 
$\frac{P(I \lvert G)}{P(I \lvert \neg G)}$, kar je verjetnost, da vas bo policija zaslišala, če je oseba kriva, deljena z verjetnostjo, da vas bo policija 
zaslišala, če je oseba nedolžna. To razmerje je večje od 1. Razumno je namreč pričakovati, da bo policija z večjo verjetnostjo izbrala za zaslišanje nekoga, 
ki je kriv, kot nekoga, ki je nedolžen. Prav to razmerje v (3) razkriva, kaj je narobe z zgornjo trditvijo, da se lahko verjetnost krivde po priznanju poveča le, 
če je $P(C \lvert G) > P(C \lvert \neg G)$. \\
To lahko pokažemo s preprostim protiprimerom. Sprejmimo domnevo, da kriminalci pod policijskim nadzorom redkeje priznajo kot nedolžni ljudje. Recimo, da prizna 
le 40 odstotkov krivcev, medtem ko prizna 60 odstotkov nedolžnih. Ker je zaradi tega verjetnost priznanja glede na krivdo manjša od verjetnosti priznanja glede 
na nedolžnost, se zdi, da sledi, da zdaj obstoj priznanja ne more povečati verjetnosti krivde od tiste, ki je bila pred priznanjem. Toda predpostavimo še, 
da je verjetnost, da bo policija zaslišala krivega, devetkrat večja kot verjetnost, da bo zaslišala nedolžnega posameznika. Nazadnje predpostavimo, da je 
predhodna verjetnost, da je oseba X kriva, na podlagi vseh dokazov pred priznanjem, 0,75. \\
Verjetnosti, ki veljajo za opisano situacijo:
\[
    P(C \lvert G, I) = 0,4
\]
\[
    P(C \lvert \neg G, I) = 0,6
\]
\[
    P(G) = 0,75
\]
\[
    P(\neg G) = 0.25
\]
\[
    \frac{P(I \lvert G)}{P(I \lvert \neg G)} = 9.
\]
Sledi po enačbi (3)
\begin{equation}
    \frac{P(G \lvert C, I)}{P(\neg G \lvert C, I)}  = \frac{P(G \lvert I)}{P(\neg G \lvert I)}  \times \frac{P(C \lvert G, I)}{P(C \lvert \neg G, I)} = \frac{0,75}{0,25} \times 9 \times \frac{0,4}{0,6} = 18
\end{equation}
Enačba (7) nam pove, da je verjetnost, da je oseba X kriva, ob upoštevanju vseh okoliščin, 18-krat večja od verjetnosti, da je nedolžena. Ker mora biti 
bodisi kriva bodisi nedolžena, lahko to razmerje uporabimo za pridobitev posteriorne verjetnosti krivde osebe X. Verjetnost, da je kriva, glede na to, 
da je priznala (in da je bil zaslišana), je 18/19 ali 0,95. Pred priznanjem je bila verjetnost krivde 0,75. Po priznanju se verjetnost krivde poveča 
na 0,95 kljub temu, da je verjetnost, da bo priznala, če je kriva, manjša od verjetnosti, da bo priznala, če je nedolžena. \\
S tem je dokaz, da je priznanje lahko dokaz krivde le, če je verjetnost priznanja glede na krivdo večja od verjetnosti priznanja glede na nedolžnost, popoln.
\end{document}