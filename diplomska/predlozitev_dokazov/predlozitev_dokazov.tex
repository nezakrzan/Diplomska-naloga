\documentclass[12pt,a4paper]{amsart}
\usepackage[slovene]{babel}
%\usepackage[cp1250]{inputenc}
\usepackage[T1]{fontenc}
\usepackage[utf8]{inputenc}
\usepackage{amsmath,amssymb,amsfonts}
\usepackage{url}
\usepackage[normalem]{ulem}
\usepackage[dvipsnames,usenames]{color}
\usepackage{graphicx}

% Oblika strani
\textwidth 15cm
\textheight 24cm
\oddsidemargin.5cm
\evensidemargin.5cm
\topmargin-5mm
\addtolength{\footskip}{10pt}
\pagestyle{plain}
\overfullrule=15pt % oznaci predlogo vrstico

% Ukazi za matematična okolja
\theoremstyle{definition} % tekst napisan pokončno
\newtheorem{definicija}{Definicija}[section]
\newtheorem{primer}[definicija]{Primer}
\newtheorem{opomba}[definicija]{Opomba}

\renewcommand\endprimer{\hfill$\diamondsuit$}


\theoremstyle{plain} % tekst napisan poševno
\newtheorem{lema}[definicija]{Lema}
\newtheorem{izrek}[definicija]{Izrek}
\newtheorem{trditev}[definicija]{Trditev}
\newtheorem{posledica}[definicija]{Posledica}

\begin{document}
%%%%%%%%%%%%%%%%%%%%%%%%%%%%%%%%%%%%%%%%%%%%%%%%%%%%%%%%%%%%%%%%%%%%%%%%%%%%%%%%%%%%%%%%%%%%%%%%%%%%%%%%%%%%%%%%%%%%%%%%%%%%%%%%%%%%%%%%%%%%%%
\title{Dokazni standardi, ki se uporabljajo v sodnem postopku}
\author{Neža Kržan}
\maketitle

%%%%%%%%%%%%%%%%%%%%%%%%%%%%%%%%%%%%%%%%%%%%%%%%%%%%%%%%%%%%%%%%%%%%%%%%%%%%%%%%%%%%%%%%%%%%%%%%%%%%%%%%%%%%%%%%%%%%%%%%%%%%%%%%%%%%%%%%%%%%%%
%%%%%%%%%%%%%%%%%%%%%%%%%%%%%%%%%%%%%%%%%%%%%%%%%%%%%%%%%%%%%%%%%%%%%%%%%%%%%%%%%%%%%%%%%%%%%%%%%%%%%%%%%%%%%%%%%%%%%%%%%%%%%%%%%%%%%%%%%%%%%%
V vsakem trenutku običajno obstaja en sam standard dokazovanja, ki je sprejet v kateri koli veji znanosti. Ampak pravo pa uporablja različne 
standarde dokazovanja za različne vrste primerov; v primerih, ki vključujejo najhujše kazni, se običajno zahteva večja stopnja dokazovanja. Čeprav 
so ti pravni standardi formulirani v verjetnostni terminologiji, jih je le približno mogoče prenesti v številčne količine. 
%V tem razdelku pregledamo koncepte dokazovanja, ki jih uporabljajo sodišča, in navedemo, kako so povezani s pravili za predložitev dokazov v sodnem postopku.
Da bi bila oseba obsojena za kaznivo dejanje, mora tožilec prepričati poroto, da je obtoženec kriv; v večini civilnih zadev, sodnik ali 
porota sprejme odločitev na podlagi prepričljivosti dokazov. V primeru, da to ne gre, se poslužijo uporabi verjetnosti. Obstajajo štiri opredelitve 
dokaznih standardov, ki jih uporabljajo sodišča, ki jim matematiki pripišejo verjetnost. Te verjetnosti so pogojne, saj se nanašajo na oceno dokazov, 
ki jo opravi sodnik.\\
Štiri opredelitve dokazov:
\begin{itemize}
    \item premoč dokaza,
    \item jasni in prepričljivi dokazi,
    \item jasen, nedvoumen in prepričljiv dokaz,
    \item onkraj razumnega dvoma.
\end{itemize}
Pojavi se že prva težava - kako pravna merila prenesti v natančen matematični okvir. Ugotovila sem tudi, da pri branju pravnih mnenj, ki uporabljajo 
statistične dokaze, je pomembno poznati vrsto zadeve in točno določeno fazo pravnega postopka, na katero se mnenje nanaša.

%V pravu Združenih držav Amerike je Fryejev standard, Fryejev test ali test splošne sprejemljivosti sodni test, ki se uporablja na ameriških sodiščih za 
%določanje dopustnosti znanstvenih dokazov. Določa, da je izvedensko mnenje, ki temelji na znanstveni tehniki, dopustno le, če je tehnika splošno 
%sprejeta kot zanesljiva v ustrezni znanstveni skupnosti.

%%%%%%%%%%%%%%%%%%%%%%%%%%%%%%%%%%%%%%%%%%%%%%%%%%%%%%%%%%%%%%%%%%%%%%%%%%%%%%%%%%%%%%%%%%%%%%%%%%%%%%%%%%%%%%%%%%%%%%%%%%%%%%%%%%%%%%%%%%%%%%
%%%%%%%%%%%%%%%%%%%%%%%%%%%%%%%%%%%%%%%%%%%%%%%%%%%%%%%%%%%%%%%%%%%%%%%%%%%%%%%%%%%%%%%%%%%%%%%%%%%%%%%%%%%%%%%%%%%%%%%%%%%%%%%%%%%%%%%%%%%%%%
\section{Uporabnost statističnega strokovnega znanja pri pomoči pravnem postopku}
Sodišča potrebujejo statistično strokovno znanje ne le za izračun rezultata statističnega postopka, temveč tudi za zagotovitev, da je metodologija 
primerna za podatke in da analiza osvetljuje pomembno pravno vprašanje. Pri skoraj vseh uporabah podatkov se sodni postopek zanaša na pričanje 
strokovnjakov, ki ocenijo zanesljivost podatkovne baze in pravilno razlagajo rezultate statistične analize. Pred pričanjem na sodišču moramo vedeti, 
na kaj točno se podatki nanašajo, kako so bili zbrani in kakšen del manjka ali je neuporaben, da se lahko odločimo za ustrezen postopek analize podatkov.
Potrebujemo osnovne informacije odvetnika in drugih strokovnjakov, da lahko oblikujemo ustrezne primerjalne skupine. Ta postopek prevajanja vključuje 
določitev ustrezne populacije (populacij), ki jo (jih) je treba preučiti, parametrov, ki nas zanimajo, in statističnega postopka, ki ga je treba uporabiti. 
Statistiki ne morejo določiti, katere vrednosti parametra so pravno pomembne, včasih pa je sam parameter interesa pravno določen.\\\\
Statistične informacije, ki jih dobi sodnik, so filtrirane prek odvetnikov. Skoraj vsak statistik, ki ga poznam in je sodeloval v več sodnih postopkih, 
lahko pove primer, ko je želel, da mu odvetnik postavi dodatna vprašanja, da bi pojasnil pomen analize ali predstavil več podatkov, vendar se je odvetnik 
odločil drugače.\\\\
Kombinirane postopke, ki so bistveni za pridobivanje informacij iz podatkov in razlago rezultatov več podobnih študij, so strokovnjaki včasih zlorabili, 
zato se sodišča morda ne zavedajo, da je predstavljena analiza nepopolna.\\\\
Uporaba statističnih podatkov na sodiščih prinaša tudi nove težave za statistiko. Zagotovo je prispevala k zanimanju za napake merjenja in njihov vpliv na 
regresijske analize, povečala pomen skrbno izvedenih retrospektivnih študij, nas spomnila na potrebo po študijah moči, zlasti kadar so velikosti 
vzorcev neenake, in na potrebo po skrbnem pregledu predpostavk, na katerih temeljijo naše metode. \\\\
%%%%%%%%%%%%%%%%%%%%%%%%%%%%%%%%%%%%%%%%%%%%%%%%%%%%%%%%%%%%%%%%%%%%%%%%%%%%%%%%%%%%%%%%%%%%%%%%%%%%%%%%%%%%%%%%%%%%%%%%%%%%%%%%%%%%%%%%%%%%%%
%%%%%%%%%%%%%%%%%%%%%%%%%%%%%%%%%%%%%%%%%%%%%%%%%%%%%%%%%%%%%%%%%%%%%%%%%%%%%%%%%%%%%%%%%%%%%%%%%%%%%%%%%%%%%%%%%%%%%%%%%%%%%%%%%%%%%%%%%%%%%%
\section{Raziskovalni proces}
Raziskovalni proces v kazenskem pravosodju je običajno namenjen preučevanju problemov kriminala. Proces se izvaja po naslednjih točkah.\\\\
\textit{1. Identifikacija problema.\\} 
Na tej stopnji je treba navesti, zakaj je raziskava potrebna za reševanje določenega problema. Problem je treba jasno opredeliti in opisati. 
Navesti je potrebno koncept(e), hipotezo in spremenljivko(e), ki jih preučujejo.\\
Koncepti so abstrakcije (npr. socialno-ekonomski status), ki jih ni mogoče neposredno opazovati, vendar jih želijo izmeriti. Meritve morajo biti 
veljavne in zanesljive. Veljavnost je stopnja, do katere merilo natančno meri spremenljivko in njen osnovni koncept. Po mojem mnenju bi se na em mestu 
že lahko vprašali oziroma soočili s problemom ali je izbrana mera jasen kazalnik zadevnega koncepta. Zanesljivost je stopnja, do katere je 
spremenljivka dosleden in zanesljiv kazalnik koncepta. Spremenljivka je namenjena merjenju teh opazovanj ali konceptov. Običajno ima več kot 
eno možno vrednost. Merjenje spremenljivke mora biti jasno opredeljeno oziroma mora imeti operativno opredelitev. Hipoteza je izražena v 
obliki odnosa med spremenljivkami. V smislu reševanja problemov hipoteza opisuje način, na katerega je mogoče rešiti problem. V raziskavi 
neodvisna spremenljivka ($X$) povzroča učinek ali vpliv na odvisno spremenljivko ($Y$). Odvisna spremenljivka ($Y$) se lahko spremeni zaradi prisotnosti 
neodvisne spremenljivke. Hipoteza je torej napoved. Pričakujemo, da bo neodvisna spremenljivka povzročila učinek na odvisno spremenljivko.\\\\
\textit{2. Zasnova raziskave.\\} 
Več elementov raziskovalne zasnove se nanaša zlasti na postopek statistične analize. Vsi imajo ključno vlogo v logiki statistike. Zasnova raziskave 
nam pomaga ugotoviti, ali bi bil program ali metoda učinkovita, če bi jo poskušali izvajati na drugih mestih in v drugih časih. To opavimo 
z t.i. klasičnim eksperimentom. Cilj raziskave je dokazati, ali je imela intervencija želeni učinek ali ne. Da bi to ugotovili, poskušamo z 
raziskovalno zasnovo izolirati učinek ukrepa na problem. Klasični eksperiment vključuje razvrstitev udeležencev v eksperimentalno in kontrolno 
skupino. Ključni element postopka je naključna izbira, ki zagotavlja, da sta skupini primerljivi v vseh pomembnih vidikih. V statističnem smislu 
naključna dodelitev pomeni, da ima vsak član ciljne populacije enake možnosti, da bo izbran v eksperimentalno skupino. Drugi element je verjetnostno 
vzorčenje. Raziskovalec običajno ne more preučiti vseh elementov populacije. Večina raziskav se izvaja z izbiro vzorca iz populacije. Zbiranje podatkov 
vključuje opredelitev in izbiro virov podatkov. Vir podatkov je lahko raziskava, uradne evidence ali uradni statistični podatki.\\\\
\textit{3. Analiza podatkov.\\}
Ko so podatki zbrani, se začne analiza s pravilno izbiro in uporabo statističnih metod. Zadnji vidik je interpretacija in predstavitev rezultatov 
raziskave. Pri tem je pomembno upoštevati občinstvo, ki mu bodo raziskovalni rezultati namenjeni in ravno pri tem se pojavi največ napak. Poleg tega 
morajo biti raziskovalci s področja kazenskega pravosodja pozorni na politične posledice rezultatov. 

%%%%%%%%%%%%%%%%%%%%%%%%%%%%%%%%%%%%%%%%%%%%%%%%%%%%%%%%%%%%%%%%%%%%%%%%%%%%%%%%%%%%%%%%%%%%%%%%%%%%%%%%%%%%%%%%%%%%%%%%%%%%%%%%%%%%%%%%%%%%%%
%%%%%%%%%%%%%%%%%%%%%%%%%%%%%%%%%%%%%%%%%%%%%%%%%%%%%%%%%%%%%%%%%%%%%%%%%%%%%%%%%%%%%%%%%%%%%%%%%%%%%%%%%%%%%%%%%%%%%%%%%%%%%%%%%%%%%%%%%%%%%%
\section{Povzemanje podatkov in predstavitev rezultatov}
Najpogostejši način razvrščanja podatkov je izdelava frekvenčne porazdelitve. Da bi prikazali, kako so podatki porazdeljeni, se ustvarijo 
neprekrivajoče se kategorije, ki vsebujejo število opazovanj v vsaki kategoriji. \\\\

%%%%%%%%%%%%%%%%%%%%%%%%%%%%%%%%%%%%%%%%%%%%%%%%%%%%%%%%%%%%%%%%%%%%%%%%%%%%%%%%%%%%%%%%%%%%%%%%%%%%%%%%%%%%%%%%%%%%%%%%%%%%%%%%%%%%%%%%%%%%%%
%%%%%%%%%%%%%%%%%%%%%%%%%%%%%%%%%%%%%%%%%%%%%%%%%%%%%%%%%%%%%%%%%%%%%%%%%%%%%%%%%%%%%%%%%%%%%%%%%%%%%%%%%%%%%%%%%%%%%%%%%%%%%%%%%%%%%%%%%%%%%%
Obravnavala sem primere, v katerih se matematične metode uporabljajo za odločanje o tem, kaj se je zgodilo ob posebni, edinstveni priložnosti, 
v nasprotju s primeri, v katerih je naloga, opredeljena v veljavnem pravu, merjenje statističnih značilnosti ali verjetnih učinkov nekega procesa 
ali statističnih značilnosti neke populacije ljudi ali dogodkov.\\
Zaradi tega je smiselno matematične dokaze razdeliti na tri različne, vendar delno prekrivajoče se kategorije:\\
\begin{enumerate}
    \item tiste v katerih je tak dokaz usmerjen na pojav ali neobstoj dogodka, dejanja ali vrste ravnanja, na katerem temelji sodni spor;
    \item tiste, pri katerih je tak dokaz usmerjen na identiteto posameznika, odgovornega za določeno dejanje ali niz dejanj;
    \item tiste, pri katerih je tak dokaz usmerjen v namero ali kakšen drug duševni element odgovornosti, kot je znanje ali provokacija.
\end{enumerate}
Pomen, ustreznost in nevarnosti matematičnega dokaza so močno odvisni od tega, ali naj bi takšen dokaz vplival na dogodek, identiteto ali miselnost.\\\\

%%%%%%%%%%%%%%%%%%%%%%%%%%%%%%%%%%%%%%%%%%%%%%%%%%%%%%%%%%%%%%%%%%%%%%%%%%%%%%%%%%%%%%%%%%%%%%%%%%%%%%%%%%%%%%%%%%%%%%%%%%%%%%%%%%%%%%%%%%%%%%
%%%%%%%%%%%%%%%%%%%%%%%%%%%%%%%%%%%%%%%%%%%%%%%%%%%%%%%%%%%%%%%%%%%%%%%%%%%%%%%%%%%%%%%%%%%%%%%%%%%%%%%%%%%%%%%%%%%%%%%%%%%%%%%%%%%%%%%%%%%%%%
\section{Teorije vrednotenja dokazov}
Ena od najbolj obravnavanih in kontroverznih tem med pravniki je bila vloga verjetnosti pri ocenjevanju pravnih dokazov, izvedenih v postopku 
ugotavljanja dejstev, ki je značilen za sodišča. To vprašanje je še posebej pomembno v kazenskih postopkih, kjer je dokazni standard določen kot 
onkraj vsakega razumnega dvoma. Na mednarodnih kazenskih sodiščih je vodilno načelo prosta ocena dokazov, kar pomeni, da sodniki niso dolžni 
spoštovati nobenega pravila o tem, kako ocenjevati dokaze, in zato lahko izberejo pristop, za katerega menijo, da je najprimernejši za oceno.\\\\
Postopek ugotavljanja dejstev zahteva oceno vseh dokazov, ki so jih predložile stranke, da bi se odločilo, ali se obdolženec lahko šteje za krivega 
ali ne. Uvede se pojem dokazni standard, ki je pravno vprašanje, tj. gre za abstraktno normo, ki je (podobno kot obstoj določenih predpostavk za 
določeno kaznivo dejanje) opredeljena s pravnim pravilom. Vrednotenje dokazov pa je dejansko vprašanje, tj. gre za odločitev, kako se dokazi v določenem 
primeru nanašajo na normo. \\
Frekvenčni (ali matematični) pristopi k ocenjevanju dokazov določajo različne odstotke za dokazni standard, ki je manjši od popolne gotovosti, zato se 
predložene informacije (ali njihovo pomanjkanje) pretvorijo v številčno vrednost (običajno približno 90-95-odstotna stopnja verjetnosti), ki se nato 
primerja z zahtevanim dokaznim standardom.\\
Frekvencistično tradicionalno verjetnost lahko opredelimo kot sistem razmišljanja, ki povezuje verjetnost nastanka dogodka s trenutnim številom. Po 
tem sistemu so možnosti dogodka za nastanek ugodne, če spada med večino opazovanih dogodkov. Nasprotno pa možnosti dogodka niso ugodne, če spada 
v manjšino opazovanih dogodkov. Verjetnost dogodka je torej enaka številu primerov, v katerih se je dogodek zgodil, deljenemu s številom vseh 
relevantnih primerov.\\\\
V tem eseju bodo opisane glavne značilnosti metode dokazne vrednosti in modela verjetnosti teme. Metoda dokazne vrednosti temelji na vrednosti, ki 
jo ima dokaz za dokazno temo, njen namen pa je ugotoviti, ali med dokazom in zadevno dokazno temo obstaja naključna povezava. Njena glavna skrb je 
dokazovanje določenega omejenega nabora dokazov, njen cilj pa je oceniti verjetnost, da dokazi dokazujejo hipotezo. Po drugi strani je cilj modela 
verjetnosti teme oceniti verjetnost hipoteze glede na dokaze. Njegov cilj je ugotoviti, kako verjetno je, da je zadevna zadeva, za katero dokazi 
lahko zagotavljajo določeno stopnjo podpore ali ne, resnična. Glavna razlika z metodo dokazne vrednosti je, da predpostavlja, da obstaja začetna 
verjetnost za temo pred obravnavo dokazov. Tako metoda dokazne vrednosti kot model verjetnosti teme temeljita na konceptu teoretične pogostosti.\\\\
Res je, da so lahko frekvenčni modeli dragoceni v primerih, kot so tisti, kjer so na voljo DNK ali druge vrste dokazov, in jih je mogoče uporabiti za 
prikaz, kako verjetno je, da bi se naključno izbrana oseba iz določene populacije ujemala z vzorcem, ali v primerih množičnih grozodejstev z velikim 
številom žrtev za izbiro statistično ustreznih vzorcev celotne populacije žrtev. Po drugi strani pa imajo ti modeli notranje pomanjkljivosti, ki jih 
ni mogoče zanemariti. Bistvena teoretična pomanjkljivost frekvenčnih teorij je, da zahtevajo statistične dokaze, ki sodišču niso na voljo: sodišča ne 
morejo številčno ovrednotiti nekega dejanskega dokaza, saj se ne zavedajo pogostosti različnih priložnostnih razmerij, ki jih uporabljajo. Na primer, 
ne morejo oceniti verjetnosti, da je opazovanje določene priče v skladu s tem, kar se je dejansko zgodilo. Poleg tega frekvenčne teorije temeljijo na 
predpostavki, da visoka vrednost verjetnosti, ki opisuje razmerje med obstoječimi dokazi in tipičnim primerom, pomeni, da je vrednost tega dokaza 
visoka. To pomeni, da se meri skladnost med dejanskimi dokazi in tistim, kar se je resnično zgodilo. Tako to merjenje temelji na predpostavki, da 
obstajata reprezentativna populacija in skladen rezultat, ta pogoj pa v kazenskem primeru nikoli ni izpolnjen. Poleg tega se najmočnejši argument nanaša 
na dejstvo, da z izračunom verjetnosti ni mogoče upoštevati posameznih primerov. Osredotoča se na statistiko, medtem ko se namesto tega v kazenskih 
zadevah največkrat obravnavajo edinstvene in redke situacije.

\end{document}