\documentclass[12pt,a4paper]{amsart}
\usepackage[slovene]{babel}
%\usepackage[cp1250]{inputenc}
\usepackage[T1]{fontenc}
\usepackage[utf8]{inputenc}
\usepackage{amsmath,amssymb,amsfonts}
\usepackage{url}
\usepackage[normalem]{ulem}
\usepackage[dvipsnames,usenames]{color}

% Oblika strani
\textwidth 15cm
\textheight 24cm
\oddsidemargin.5cm
\evensidemargin.5cm
\topmargin-5mm
\addtolength{\footskip}{10pt}
\pagestyle{plain}
\overfullrule=15pt % oznaci predlogo vrstico


% Ukazi za matematična okolja
\theoremstyle{definition} % tekst napisan pokončno
\newtheorem{definicija}{Definicija}[section]
\newtheorem{primer}[definicija]{Primer}
\newtheorem{opomba}[definicija]{Opomba}

\renewcommand\endprimer{\hfill$\diamondsuit$}


\theoremstyle{plain} % tekst napisan poševno
\newtheorem{lema}[definicija]{Lema}
\newtheorem{izrek}[definicija]{Izrek}
\newtheorem{trditev}[definicija]{Trditev}
\newtheorem{posledica}[definicija]{Posledica}


% Za številske mnozice uporabi naslednje simbole
\newcommand{\R}{\mathbb R}
\newcommand{\N}{\mathbb N}
\newcommand{\Z}{\mathbb Z}
\newcommand{\C}{\mathbb C}
\newcommand{\Q}{\mathbb Q}

% Ukaz za slovarsko geslo
\newlength{\odstavek}
\setlength{\odstavek}{\parindent}
\newcommand{\geslo}[2]{\noindent\textbf{#1}\hspace*{3mm}\hangindent=\parindent\hangafter=1 #2}


\newcommand{\program}{Finančna matematika} 
\newcommand{\imeavtorja}{Neža Kržan} 
\newcommand{\imementorja}{izr. prof. Jaka Smrekar} 
\newcommand{\naslovdela}{Statistika v kazenskem pravu}
\newcommand{\letnica}{2022} %letnica diplome


\begin{document}

%%%%%%%%%%%%%%%%%%%%%%%%%%%%%%%%%%%%%%%%%%%%%%%%%%%%%%%%%%%%%%%%%%%%%%%%%%%%%%%%%%%%%%%%%%%%%%%%%%%%%%%%%%%%%%%%%%%%%%%%%%%%%%%%%%%%%%%%%%%%
\thispagestyle{empty}
\noindent{\large
UNIVERZA V LJUBLJANI\\[1mm]
FAKULTETA ZA MATEMATIKO IN FIZIKO\\[5mm]
\program\ -- 1.~stopnja}
\vfill

\begin{center}{\large
\imeavtorja\\[2mm]
{\bf \naslovdela}\\[10mm]
Delo diplomskega seminarja\\[1cm]
Mentor: \imementorja}
\end{center}
\vfill

\noindent{\large
Ljubljana, \letnica}
\pagebreak

%%%%%%%%%%%%%%%%%%%%%%%%%%%%%%%%%%%%%%%%%%%%%%%%%%%%%%%%%%%%%%%%%%%%%%%%%%%%%%%%%%%%%%%%%%%%%%%%%%%%%%%%%%%%%%%%%%%%%%%%%%%%%%%%%%%%%%%%%%%%
\thispagestyle{empty}
\tableofcontents
\pagebreak

%%%%%%%%%%%%%%%%%%%%%%%%%%%%%%%%%%%%%%%%%%%%%%%%%%%%%%%%%%%%%%%%%%%%%%%%%%%%%%%%%%%%%%%%%%%%%%%%%%%%%%%%%%%%%%%%%%%%%%%%%%%%%%%%%%%%%%%%%%%%
%\thispagestyle{empty}
%\begin{center}
%{\bf \naslovdela}\\[3mm]
%{\sc Povzetek}
%\end{center}
% tekst povzetka v slovenscini
%V povzetku na kratko opišite vsebinske rezultate dela. Sem ne sodi razlaga organizacije dela -- v katerem poglavju/razdelku je kaj, pač pa le opis vsebine.
%\vfill
%\begin{center}
%{\bf Angle"ski naslov dela}\\[3mm] % prevod slovenskega naslova dela 
%{\sc Abstract}
%\end{center}
% tekst povzetka v anglescini
%Prevod zgornjega povzetka v angleščino.

%\vfill\noindent
%{\bf Math. Subj. Class. (2010):} navedite vsaj eno klasifikacijsko oznako -- dostopne so na \url{www.ams.org/mathscinet/msc/msc2010.html}  \\[1mm]  
%{\bf Ključne besede:} navedite nekaj ključnih pojmov, ki nastopajo v delu  \\[1mm]  
%{\bf Keywords:} angleški prevod ključnih besed
%\pagebreak

%%%%%%%%%%%%%%%%%%%%%%%%%%%%%%%%%%%%%%%%%%%%%%%%%%%%%%%%%%%%%%%%%%%%%%%%%%%%%%%%%%%%%%%%%%%%%%%%%%%%%%%%%%%%%%%%%%%%%%%%%%%%%%%%%%%%%%%%%%%%
%%%%%%%%%%%%%%%%%%%%%%%%%%%%%%%%%%%%%%%%%%%%%%%%%%%%%%%%%%%%%%%%%%%%%%%%%%%%%%%%%%%%%%%%%%%%%%%%%%%%%%%%%%%%%%%%%%%%%%%%%%%%%%%%%%%%%%%%%%%%
\section{Bayesova statistika}
Bayesova statistika je statistična veja, ki nam s pomočjo matematičnih pristopov omogoča uporabo verjetnosti pri reševanju statističnih 
problemov. V svoje modele vključuje pogojno verjetnost, katero izračunamo z uporabo Bayesovega pravila. \\

Zlasti Bayesovo sklepanje razlaga verjetnost kot merilo verjetnosti ali zaupanja, ki ga lahko ima posameznik glede nastanka določenega dogodka. 
O nekem dogodku lahko že imamo predhodno prepričanje oziroma apriorno prepričanje, ki pa se lahko spremeni, ko se pojavijo novi dokazi. Bayesova 
statistika nam daje matematične modele za vključevanje naših apriornih prepričanj in dokazov za ustvarjenje novih prepričanj oziroma za 
pridobitev aposteriornega prepričanja, ki se lahko uporabi za kasnejše odločitve.

%%%%%%%%%%%%%%%%%%%%%%%%%%%%%%%%%%%%%%%%%%%%%%%%%%%%%%%%%%%%%%%%%%%%%%%%%%%%%%%%%%%%%%%%%%%%%%%%%%%%%%%%%%%%%%%%%%%%%%%%%%%%%%%%%%%%%%%%%%%%
\subsection{Bayesovo pravilo} 
Bayesovo sklepanje temelji na Bayesovim pravilom, ki izraža verjetnost nekega dogodka z verjetnostjo dveh dogodkov in obrnejnje pogojne 
verjetnosti. Pogojna verjetnost predstavlja verjetnost dogodka, glede na drug dogodek. 

\begin{definicija}
Pogojna verjetnost dogodka H, glede na dogodek E, je
\begin{equation}\label{eq:pogojna}
P(H \mid E) = \frac{P(H \cap E)}{P(E)},
\end{equation}
ob predpostavki, da je $P(E) > 0$.
\end{definicija}

Formula \eqref{eq:pogojna} pove, da je verjetnost dogodka H ob pogoju, da se je zgodil dogodek E, enaka razmerju verjetnosti, da se 
zgodita oba dogodka in verjetnosti, da se je zgodil dogodek E.\\

Potem pogojno verjetnost uporabimo še v števcu formule \eqref{eq:pogojna} in dobimo Bayesovo pravilo:
\begin{equation}\label{eq:bpravilo}
    P(H \mid E) = \frac{P(E \mid H) \times P(H)}{P(E)},
\end{equation}

Verjetnost dogodka E lahko še razpišemo in dobimo:
\begin{equation}\label{eq:b_pravilo}
    P(H \mid E) = \frac{P(E \mid H) \times P(H)}{P(E \mid H)P(H) + P(E \mid \neg H)P(\neg H)}.
\end{equation} \\

Obstaja še ena formulacija Bayesovega pravila, ki olajša izračune in je pogosto uporabljena pri Bayesovi analizi DNK dokazov:
\begin{equation}\label{eq:b_pravilo_DNK}
    \frac{P(H \mid E)}{P(\neg H \mid E)} = \frac{P(E \mid H)}{P(E \mid \neg H)} \times \frac{P(H)}{P(\neg H)}.
\end{equation}\\

%%%%%%%%%%%%%%%%%%%%%%%%%%%%%%%%%%%%%%%%%%%%%%%%%%%%%%%%%%%%%%%%%%%%%%%%%%%%%%%%%%%%%%%%%%%%%%%%%%%%%%%%%%%%%%%%%%%%%%%%%%%%%%%%%%%%%%%%%%%%
\subsection{Bayesovo posodabljanje}
Bayesovo pravilo se razlikuje od Bayesovega posodabljanja. Prvo je matematični izrek, drugo pa logična trditev, kako se sčasoma posodabljajo 
apriorne verjetnosti dokazov glede na novo zbrane dokaze oziroma prepričanja. \\

Bayesovo posodabljanje pravi:
\begin{trditev}
Če se dogodek E zgodi ob času $t_1 > t_0$, potem je $P_1(H) = P_0(H \mid E)$.
\end{trditev}

Ob času $t_0$ dogodku H dodelimo verjetnost $P_0(H)$; to se imenuje predhodna verjetnost oziroma apriorna verjetnost. Ko se zgodi dogodek E 
ob času $t_1$, ki vpliva na naša prepričanja o dogodku H, Bayesovo posodabljanje pravi, da je potrebno apriorno verjetnost dogodka H v času $t_1$ 
enačiti z pogojno verjetnostjo dogodka H glede na dogodek E v času $t_0$. \\

Recimo, da je dogodek H neka hipoteza oziroma prepričanje o zločinu in dogodek E dokazi, zbrani za ta zložin. Pri Bayesovem posodabljanju je videti, 
kot da je dokaz E nesporno resničen. Z drugimi besedami, predpostavka je, da moramo imeti po zbiranju dokazov E stopnjo zaupanja v E enako 1, 
torej če so dokazi zbarni v času $t_1$, je $P_1(E)=1$.

%%%%%%%%%%%%%%%%%%%%%%%%%%%%%%%%%%%%%%%%%%%%%%%%%%%%%%%%%%%%%%%%%%%%%%%%%%%%%%%%%%%%%%%%%%%%%%%%%%%%%%%%%%%%%%%%%%%%%%%%%%%%%%%%%%%%%%%%%%%%
%%%%%%%%%%%%%%%%%%%%%%%%%%%%%%%%%%%%%%%%%%%%%%%%%%%%%%%%%%%%%%%%%%%%%%%%%%%%%%%%%%%%%%%%%%%%%%%%%%%%%%%%%%%%%%%%%%%%%%%%%%%%%%%%%%%%%%%%%%%%
\section{}


%%%%%%%%%%%%%%%%%%%%%%%%%%%%%%%%%%%%%%%%%%%%%%%%%%%%%%%%%%%%%%%%%%%%%%%%%%%%%%%%%%%%%%%%%%%%%%%%%%%%%%%%%%%%%%%%%%%%%%%%%%%%%%%%%%%%%%%%%%%%
%%%%%%%%%%%%%%%%%%%%%%%%%%%%%%%%%%%%%%%%%%%%%%%%%%%%%%%%%%%%%%%%%%%%%%%%%%%%%%%%%%%%%%%%%%%%%%%%%%%%%%%%%%%%%%%%%%%%%%%%%%%%%%%%%%%%%%%%%%%%
% Seznam uporabljene literature
%https://www.quantstart.com/articles/Bayesian-Statistics-A-Beginners-Guide/
\end{document}