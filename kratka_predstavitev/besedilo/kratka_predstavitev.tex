\documentclass[12pt,a4paper]{amsart}
\usepackage[slovene]{babel}
%\usepackage[cp1250]{inputenc}
\usepackage[T1]{fontenc}
\usepackage[utf8]{inputenc}
\usepackage{amsmath,amssymb,amsfonts}
\usepackage{url}
\usepackage[normalem]{ulem}
\usepackage[dvipsnames,usenames]{color}
\usepackage{graphicx}

% Oblika strani
\textwidth 15cm
\textheight 24cm
\oddsidemargin.5cm
\evensidemargin.5cm
\topmargin-5mm
\addtolength{\footskip}{10pt}
\pagestyle{plain}
\overfullrule=15pt % oznaci predlogo vrstico

% Ukazi za matematična okolja
\theoremstyle{definition} % tekst napisan pokončno
\newtheorem{definicija}{Definicija}[section]
\newtheorem{primer}[definicija]{Primer}
\newtheorem{opomba}[definicija]{Opomba}

\renewcommand\endprimer{\hfill$\diamondsuit$}


\theoremstyle{plain} % tekst napisan poševno
\newtheorem{lema}[definicija]{Lema}
\newtheorem{izrek}[definicija]{Izrek}
\newtheorem{trditev}[definicija]{Trditev}
\newtheorem{posledica}[definicija]{Posledica}

\begin{document}

%%%%%%%%%%%%%%%%%%%%%%%%%%%%%%%%%%%%%%%%%%%%%%%%%%%%%%%%%%%%%%%%%%%%%%%%%%%%%%%%%%%%%%%%%%%%%%%%%%%%%%%%%%%%%%%%%%%%%%%%%%%%%%%%%%%%%%%%%%%%%%
\title{Statistika v kazenskem pravu}
\author{Neža Kržan}
\maketitle

V diplomski nalogi se bom osredotočila na statistiko v kazenskem pravu in na zmote, ki se pojavljajo pri uporabi le te, zaradi pomanjkanja
znanja statistike pri odvetnikih, sodnikih in poroti. Osredotočila se bom na uporabo Bayesove statistike v kazenskih postopkih oziroma na izračune,
ki izhajajo iz Bayesove statistike in jo primerjala z drugimi metodami. V nadaljevanju bom opisala in razložila dve najpogostejši zmoti, prva
je Tožilčeva zmota, ki je dobro znan statistični problem, druga večja, ki pa izhaja iz prve, pa je Zmota obrambnega odvetnika. Ker je uporaba
statistike in verjetnostnega računa čedalje pogostejša v sodnih postopkih, bom na koncu pregledala resničen primer sodbe medicinski sestri Lucii de Berk.

%%%%%%%%%%%%%%%%%%%%%%%%%%%%%%%%%%%%%%%%%%%%%%%%%%%%%%%%%%%%%%%%%%%%%%%%%%%%%%%%%%%%%%%%%%%%%%%%%%%%%%%%%%%%%%%%%%%%%%%%%%%%%%%%%%%%%%%%%%%%
%%%%%%%%%%%%%%%%%%%%%%%%%%%%%%%%%%%%%%%%%%%%%%%%%%%%%%%%%%%%%%%%%%%%%%%%%%%%%%%%%%%%%%%%%%%%%%%%%%%%%%%%%%%%%%%%%%%%%%%%%%%%%%%%%%%%%%%%%%%%
\section{Bayesova statistika}
Bayesova statistika je statistična veja, ki nam s pomočjo matematičnih pristopov omogoča uporabo verjetnosti pri reševanju statističnih
problemov. V svoje modele vključuje pogojno verjetnost, katero izračunamo z uporabo Bayesovega pravila. \\
 
Zlasti Bayesovo sklepanje razlaga verjetnost kot merilo verjetnosti ali zaupanja, ki ga lahko ima posameznik glede nastanka določenega dogodka.
O nekem dogodku lahko že imamo predhodno prepričanje oziroma apriorno prepričanje, ki pa se lahko spremeni, ko se pojavijo novi dokazi. Bayesova
statistika nam daje matematične modele za vključevanje naših apriornih prepričanj in dokazov za ustvarjanje novih prepričanj oziroma za
pridobitev aposteriornega prepričanja, ki se lahko uporabi za kasnejše odločitve.

%%%%%%%%%%%%%%%%%%%%%%%%%%%%%%%%%%%%%%%%%%%%%%%%%%%%%%%%%%%%%%%%%%%%%%%%%%%%%%%%%%%%%%%%%%%%%%%%%%%%%%%%%%%%%%%%%%%%%%%%%%%%%%%%%%%%%%%%%%%%
\subsection{Bayesovo pravilo} 
Bayesovo sklepanje temelji na Bayesovem pravilu, ki izraža verjetnost nekega dogodka z verjetnostjo dveh dogodkov in obrnejnje pogojne
verjetnosti. Pogojna verjetnost predstavlja verjetnost dogodka, glede na drug dogodek. \\
 
S pomočjo formule za pogojno verjetnost dobimo Bayesovo pravilo:
\begin{equation}\label{eq:bpravilo}
    P(H \lvert E) = \frac{P(E \lvert H) \times P(H)}{P(E)},
\end{equation}
verjetnost dogodka E lahko še razpišemo in dobimo:
\begin{equation}\label{eq:b_pravilo}
    P(H \lvert E) = \frac{P(E \lvert H) \times P(H)}{P(E \lvert H)P(H) + P(E \lvert \neg H)P(\neg H)\vspace{3mm}}.
\end{equation}

%%%%%%%%%%%%%%%%%%%%%%%%%%%%%%%%%%%%%%%%%%%%%%%%%%%%%%%%%%%%%%%%%%%%%%%%%%%%%%%%%%%%%%%%%%%%%%%%%%%%%%%%%%%%%%%%%%%%%%%%%%%%%%%%%%%%%%%%%%%%
%%%%%%%%%%%%%%%%%%%%%%%%%%%%%%%%%%%%%%%%%%%%%%%%%%%%%%%%%%%%%%%%%%%%%%%%%%%%%%%%%%%%%%%%%%%%%%%%%%%%%%%%%%%%%%%%%%%%%%%%%%%%%%%%%%%%%%%%%%%%
\section{Bayesova analiza}
V kazenskih zadevah želimo vedeti, ali je obtoženec kriv ali ne. Če imamo torej na voljo dokaz $E$, nas zanima pogojna verjetnost \vspace{3mm}
\[P(kriv \lvert E)\], pri čemer nam je lahko v pomoč Bayesovo pravilo. To v teoriji drži, čeprav je v praksi izračun verjetnostne krivde lahko
preveč zapleten. Ampak z Bayesovim pravilom lahko ocenimo verjetnosti vmesnih trditev oziroma dokazov, ki so ključnega pomena za ugotavljanje
obtoženčeve krivde. Najbolj pogosta uporaba Bayesovega pravila je pri ugotavljanju, ali je obtoženec vir sledi DNK-ja s kraja zločina. \\

%%%%%%%%%%%%%%%%%%%%%%%%%%%%%%%%%%%%%%%%%%%%%%%%%%%%%%%%%%%%%%%%%%%%%%%%%%%%%%%%%%%%%%%%%%%%%%%%%%%%%%%%%%%%%%%%%%%%%%%%%%%%%%%%%%%%%%%%%%%%
\subsection{Poenostavljena Bayesova analiza}
Naj bo:\\
$S$ \dots trditev, da je obtoženec vir sledi DNK s kraja zločina; \\
$M$ \dots trditev, da se obtoženčev DNK ujema z DNK-jem s kraja zločina; \\
$f$ \dots funkcija pogostosti ujemanja DNK z DNK-jem s kraja zločina. \\
Želimo vedeti, kakšna je verjetnost S glede na M, tj. $P(S \lvert M)$. \\

Bayesovo pravilo lahko uporabimo na naslednji način:
\[\frac{P(S \lvert M)}{P(\neg S \lvert M)} = \frac{P(M \lvert S)}{P(M \lvert \neg S)} \times \frac{P(S)}{P(\neg S)}.\] \\
Verjetnosti $P(S)$ in $P(\neg S)$ je težko oceniti, ker ne vemo kakšna je množica osumljencev. Smiselno bi bilo, da zanju upoštevamo interval 
predhodnih verjetnosti in ocenimo njihov vpliv na verjetnost trditve $S$ in njene negacije. Nato moramo določiti vrednost $P(M \lvert S)$, ki 
je običajno enaka ena - če bi obtoženec dejansko pustil sledove, bi laboratorijske analize pokazale ujemanje(to imenujemo lažno 
negativni rezultat); to je sicer poenostavitev, saj se lahko zgodi, da analize ne pokažejo ujemanja, čeprav je obtoženec pustil sledi. 
Potrebujemo še verjetnost $P(M \lvert \neg S)$ (verjetnost, da se bo našlo ujemanje, če obtoženec ni vir sledi na kraju zločina). To je 
običajno enakovredno pogostosti ujemnja DNK-ja z DNK-jem s kraja zložina (tj. $f$); tudi to je poenostavitev, saj se lahko zgodi, da 
obtoženec nima enakega DNK profila, vendar so laboratorijske analize pokazale, da ga ima(to imenujemo lažno 
pozitivni rezultat).
Sledi:
\[\frac{P(S \lvert M)}{P(\neg S \lvert M)} = \frac{1}{f} \times \frac{P(S)}{P(\neg S)}.\] \\

Ker poenostavljena Bayesova analiza ne upošteva možnosti lažno pozitivne in negativne laboratorijske analize si poglejmo še izpopolnjeno 
Bayesovo analizo.

%%%%%%%%%%%%%%%%%%%%%%%%%%%%%%%%%%%%%%%%%%%%%%%%%%%%%%%%%%%%%%%%%%%%%%%%%%%%%%%%%%%%%%%%%%%%%%%%%%%%%%%%%%%%%%%%%%%%%%%%%%%%%%%%%%%%%%%%%%%%
\subsection{Izpopolnjena Bayesova analiza}
Za upoštevanje možnosti laboratorijskih napak, bomo namesto $M$ uvedli spremenljivko $M_p$.\\
$M_p$ \dots poročano ujemanje laboratorijske analize; \\
$M_t$ \dots trditev, da obstaja dejansko ujemanje v DNK-ju;\\
$\neg M_t$ \dots trditev, da obstaja neujemanje v DNK-ju.  \\
 
Sledi:
\[P(M_p \lvert \neg S) = P(M_p \lvert M_t)P(M_t \lvert \neg S) + P(M_p \lvert \neg M_t)P(\neg M_t \lvert \neg S). \vspace{3mm}\]
Sedaj je $P(M_t \lvert \neg S)$ enako $f$ in zato $P(\neg M_t \lvert \neg S)$ enako $1-f$. $P(M_p \lvert \neg M_t)$ opisuje verjetnost lažno
pozitivnih rezultatov laboratorija(oznaka $FP$) in $P(M_p \lvert M_t)$ verjetnost resničnih pozitivnih rezultatov laboratorija(oznaka $FN$).
Sledi: \vspace{2mm}
\[P(M \lvert \neg S) = [(1 - FN) \times f] + [FP \times (1 - f)]. \vspace{3mm}\]
 
Formula pokaže, da za pravilno oceno verjetnosti $P(M_p \lvert \neg S)$ potrebujemo statistično oceno pogostosti profila DNK in stopnje napak
laboratorijskih analiz, ki pa so redko na voljo. \vspace{3mm}
 
%%%%%%%%%%%%%%%%%%%%%%%%%%%%%%%%%%%%%%%%%%%%%%%%%%%%%%%%%%%%%%%%%%%%%%%%%%%%%%%%%%%%%%%%%%%%%%%%%%%%%%%%%%%%%%%%%%%%%%%%%%%%%%%%%%%%%%%%%%%%
%%%%%%%%%%%%%%%%%%%%%%%%%%%%%%%%%%%%%%%%%%%%%%%%%%%%%%%%%%%%%%%%%%%%%%%%%%%%%%%%%%%%%%%%%%%%%%%%%%%%%%%%%%%%%%%%%%%%%%%%%%%%%%%%%%%%%%%%%%%%
\section{Drugi pristopi}
Da bi ocenila moč Bayesove anlaize dokazov DNK, jo bom primerjala z nekaterimi drugimi pristopi kot so frekvence in edinstvenost ter naključno 
ujemanje in razmerja verjetnosti.

%%%%%%%%%%%%%%%%%%%%%%%%%%%%%%%%%%%%%%%%%%%%%%%%%%%%%%%%%%%%%%%%%%%%%%%%%%%%%%%%%%%%%%%%%%%%%%%%%%%%%%%%%%%%%%%%%%%%%%%%%%%%%%%%%%%%%%%%%%%%
\subsection{Frekvence in edinstvenost}
Predlagano je bilo s strani mnogih avtorjev, da je bolj naraven način za obravnavo verjetnosti uporaba naravnih frekvenc(angl. natural
frequencies). \\
 
Po metodah, ki temeljijo na naravnih frekvencah, je potrebno izračunati, koliko ljudi z zadevnim profilom DNK je v populaciji osumljencem, tako
da pogostost profila DNK pomnožimo z velikostjo populacije osumljencev.  \\
V primeru ko ne na primer pogostost profila DNK oz. frekvenca še posebaj majhne, je bolje raziskovati ali je profil DNK edinstven ali ne - ali 
poleg obtoženca obstajajo še drugi posamezniki z enakim profilom DNK. Temu pravimo metoda edinstvenosti(angl. uniqueness method). \\

Videli bomo, da se obe metodi dokaj ujemata, vendar se tudi razlikujeta, saj v nasprotju z Bayesovim pravilom pri metodi edinstvenosti 
ni mogoče enostavno upoštevati številnih zapletov, kot je bil naprimer vpliv stopnje laboratorijskih napak. \\

%%%%%%%%%%%%%%%%%%%%%%%%%%%%%%%%%%%%%%%%%%%%%%%%%%%%%%%%%%%%%%%%%%%%%%%%%%%%%%%%%%%%%%%%%%%%%%%%%%%%%%%%%%%%%%%%%%%%%%%%%%%%%%%%%%%%%%%%%%%%
\subsection{Naključno ujemanje in razmerja verjetnosti}
Metoda verjetnost naključnega ujemanja(angl. random match probability) izraža možnost, da bi imel naključni posameznik, ki ni povezan z obdolžencem,
ustrezni DNK profil. Ta verjetnost je enaka pogostosti profila DNK. Težava tega pristopa je, da verjetnost naključnega ujemanja lahko predstavljena
oziroma razumevanja narobe. \\
 
Bayesovo pravilo vključuje razmerje verjetnosti in predhodno oziroma apriorno verjetnost, torej je celovitejši način predstavitve dokazov DNK.
Težavo ocenjevanja apriorne verjetnosti pri Bayesovem pravilu bi lahko odpravili z osredotočanjem le na verjetnost. Metoda razmerja verjetnosti je
koristna v državah kot sta na primer Velika Britanija in Združene države Amerike, kjer se lahko Bayesovo pravilo šteje kot poseg v pravico porote
do previdnosti: poroti naj ne bi bilo potrebno govoriti, kako naj razmišlja in presoja dokaze; Bayesovo pravilo pa je natanko metoda za
tehtanje dokazov. \vspace{3mm}

%%%%%%%%%%%%%%%%%%%%%%%%%%%%%%%%%%%%%%%%%%%%%%%%%%%%%%%%%%%%%%%%%%%%%%%%%%%%%%%%%%%%%%%%%%%%%%%%%%%%%%%%%%%%%%%%%%%%%%%%%%%%%%%%%%%%%%%%%%%%
%%%%%%%%%%%%%%%%%%%%%%%%%%%%%%%%%%%%%%%%%%%%%%%%%%%%%%%%%%%%%%%%%%%%%%%%%%%%%%%%%%%%%%%%%%%%%%%%%%%%%%%%%%%%%%%%%%%%%%%%%%%%%%%%%%%%%%%%%%%%
\section{Zmote}
Ker večina ljudi pri razmišljanju o verjetnosti dela osnovne napake, obstaja mnogo zmot, ki izhajajo iz osnovnega razumevanja pravil
teorije verjetnosti. Številne od teh zmot so zlasti posledica napačnega razumevanja pogojne verjetnosti. Bolj znana primera takih zmot sta
Tožilčeva zmota(angl. Prosecutor’s fallacy) in Zmota obrambnega odvetnika(angl. Defense attorney's fallacy). Čeprav so posledice tožilčeve zmote 
lahko hujše kot posledice zmote obrambnega odvetnika, so porote morda bolj dovzetne za slednjo kot za prvo. \\
 
Manj znana zmota je tudi Zasliševalčeva zmota(angl. Interrogator’s fallacy), ki razloži zakaj priznanja, pridobljena med zaslišanjem, ponujajo
dvomljive dokaze v prid krivdi oziroma dokazi med sodbo ne morejo zmanjšati verjetnosti krivde.

%%%%%%%%%%%%%%%%%%%%%%%%%%%%%%%%%%%%%%%%%%%%%%%%%%%%%%%%%%%%%%%%%%%%%%%%%%%%%%%%%%%%%%%%%%%%%%%%%%%%%%%%%%%%%%%%%%%%%%%%%%%%%%%%%%%%%%%%%%%%
\subsection{Tožilčeva zmota}
Tožilčeva zmota(angl. Prosecutor’s fallacy) se pogosto pojavlja v kazenskem pravu, vendar jo pogosto neprepoznajo, deloma zato, ker preiskovalci 
nimajo močne intuicije o tem, kaj zmota sploh pomeni. Tožilčeva zmota je dobro znana statistična zmota, ki izhaja iz napačnega razumevanja 
pogojnih verjetnosti in vprašanj večkratnega testiranja. Napaka temelji na predpostavki, da je $P(H \lvert E) = P(E \lvert H)$, pri čemer $H$ 
predstavlja primer, da se najdejo dokazi o obtožencu, $E$ pa primer, da je obtoženec nedolžen. Vendar ta enakost ne drži: čeprav je $P(H \lvert E)$ 
običajno zelo majhen, je lahko $P(E \lvert H)$ še vedno veliko večji. \\ 

Za lažjo predstavo si oglejmo primer. Na primer, da ima storilec zločina enako krvno skupino kot obtoženec. Ta krvna skupina je tako redka, da je 
verjetnost, da jo bo nekdo imel, le 1 proti 1000. Tožilčeva zmota bi bila, da je verjetnost, da je obtoženec nedolžen enaka 1 proti 1000. Verjetnost
naključnega ujemanja, torej pogostost krvnega profila, smo neustrezno združili z verjetnostjo izvora, torej verjetnost, da je vir krvi s kraja zločina 
nekdo drug kot obtoženec. V mestu z milijon prebivalci bi torej bilo 1000 prebivalcev z ustrezno krvno skupino. Čeprav obstaja le 1 proti 1000 možnosti, 
da se kri naključne osebe ujema s krvjo zločinca, je verjetnost, da je oseba, ki ustreza krvni skupini, nedolžna na podlagi ujemanja dokazov 999 proti 1000.

%%%%%%%%%%%%%%%%%%%%%%%%%%%%%%%%%%%%%%%%%%%%%%%%%%%%%%%%%%%%%%%%%%%%%%%%%%%%%%%%%%%%%%%%%%%%%%%%%%%%%%%%%%%%%%%%%%%%%%%%%%%%%%%%%%%%%%%%%%%%
\subsection{Zmota obrambnega odvetnika}
Zmota zagovornika oziroma zmota obrambnega odvetnika(angl. Defense attorney's fallacy ) se pojavi, ko se poroča o tem, koliko ljudi z 
določeno značilnostjo, se pojavi v določeni populaciji. Predpostavlja se, da je storilec del neke poljubno velike populacije in da ni na voljo 
drugih informacij, torej je za vse enako verjetno, da so storilci. Na podlagi teh predpostavk lahko sklepamo, da obstaja majhna verjetnost, 
da je osumljenec storilec kaznivega dejanja. \\

Za ponazoritev se spomnimo enakega primera sojenja kot prej, vendar z dodatnimi dokazi proti obtožencu. Zagovornik lahko kljub tem dodatnim 
dokazom poroti pove, da je verjetnost obtoženčeve krivde le 1 proti 1000, ampak to pomeni, da porota ne upošteva dodatnih dokazov. Če obrambnemu 
odvetniku uspe prepričati poroto, da verjame temu napačnemu izračunu, lahko dodatni dokazi torej napačno zmanjšajo prepričanje porote o krivdi. \vspace{3mm}

%%%%%%%%%%%%%%%%%%%%%%%%%%%%%%%%%%%%%%%%%%%%%%%%%%%%%%%%%%%%%%%%%%%%%%%%%%%%%%%%%%%%%%%%%%%%%%%%%%%%%%%%%%%%%%%%%%%%%%%%%%%%%%%%%%%%%%%%%%%%
%%%%%%%%%%%%%%%%%%%%%%%%%%%%%%%%%%%%%%%%%%%%%%%%%%%%%%%%%%%%%%%%%%%%%%%%%%%%%%%%%%%%%%%%%%%%%%%%%%%%%%%%%%%%%%%%%%%%%%%%%%%%%%%%%%%%%%%%%%%%
\section{Primer - Lucia de B.}
Za podroben vpogled statistike v kazenskem pravu se bom v svoji diplomski nalogi osredotočila na sojenje medicinski sestri Lucii, ki je bila
obsojena na dosmrtno kazen zaradi domnevnega uboja ali poskusa uboja več bolnikov v dveh bolnišnicah, kjer je v tistem času delala. V obeh
bolnišnicah, kjer je delala, so se v času njenih izmen dogajali incidenti, zato se je postavilo vprašanje ali je Luciina prisotnost pri
toliko incidentih zgolj naključna. Predpostavili so dve hipotezi, in sicer: \\
$H_0 =$ verjetnost, da se osumljenki med izmeno zgodi incident, je enaka verjetnosti za katero koli drugo medicinsko sestro; \\
$H_1=$ pojavljanje incidentov je nedovisno od izmene.\\
 
Izračunali so, da je verjetnost, da je osumljenka doživela toliko incidentov, manjša od 1 proti 342 milijonom. Po mnenju statistikov, ki so
delali izračune, je ta verjetnost tako majhna, da je po standardni statistični metodologiji potrebno ničeno hipotezo zavrniti. Izračuni so bili
pomankljivi na večih področjih, saj je bil verjetnostni model preveč poenostavljen. Poleg tega je sodišče ocenjeno verjetnost 1 proti 342 milijonom
interpretiralo napačno in se je zgodila tako imenovana Tožilčeva zmota.

\end{document}