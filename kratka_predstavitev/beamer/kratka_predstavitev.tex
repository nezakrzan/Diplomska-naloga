\documentclass{beamer}
\mode<presentation>

\usepackage[utf8]{inputenc}
\usepackage[T1]{fontenc}
\usepackage[slovene]{babel}
\usepackage{lmodern}
\usepackage{array} 
\usepackage{tikz}

\usetheme{Berlin}
\usecolortheme{default}
\useinnertheme[shadows]{rounded}
\useoutertheme{infolines}
\beamertemplatenavigationsymbolsempty

\usepackage{times}
\newcommand{\ds}{\displaystyle}
\newcommand{\ts}{\textstyle}
\newcommand{\presledek}{\vspace{3mm}}
\newcommand{\ph}{\phantom{1}} 
\newcommand{\tr}{{\rm tr \,}}

\newtheorem{definicija}{Definicija}
\newtheorem{izrek}{Izrek}

\begin{document}

%%%%%%%%%%%%%%%%%%%%%%%%%%%%%%%%%%%%%%%%%%%%%%%%%%%%%%%%%%%%%%%%%%%%%%%%%%%%%%%%%%%%%%%%%%%%%%%%%%%%%%%%%%%%%%%%%%%%%%%%%%%%%%%%%%%%%%%%%%%%
%%%%%%%%%%%%%%%%%%%%%%%%%%%%%%%%%%%%%%%%%%%%%%%%%%%%%%%%%%%%%%%%%%%%%%%%%%%%%%%%%%%%%%%%%%%%%%%%%%%%%%%%%%%%%%%%%%%%%%%%%%%%%%%%%%%%%%%%%%%%
\title{Statistika v kazenskem pravu}
\subtitle{Kratka predstavitev diplomske naloge}
\author[Neža Kržan]{Neža Kržan}
 
\institute[FMF]{Mentor: izr. prof. Jaka Smrekar \\ Fakulteta za matematiko in fiziko \\ \vspace{10mm} Ljubljana, 28.november 2022}
\date[28. november 2022] {}

\subject{Talks}

\begin{frame}
   \titlepage
\end{frame}

%%%%%%%%%%%%%%%%%%%%%%%%%%%%%%%%%%%%%%%%%%%%%%%%%%%%%%%%%%%%%%%%%%%%%%%%%%%%%%%%%%%%%%%%%%%%%%%%%%%%%%%%%%%%%%%%%%%%%%%%%%%%%%%%%%%%%%%%%%%%
%%%%%%%%%%%%%%%%%%%%%%%%%%%%%%%%%%%%%%%%%%%%%%%%%%%%%%%%%%%%%%%%%%%%%%%%%%%%%%%%%%%%%%%%%%%%%%%%%%%%%%%%%%%%%%%%%%%%%%%%%%%%%%%%%%%%%%%%%%%%
\begin{frame}
   \frametitle{Kratek pregled}
   \tableofcontents[pausesections]
\end{frame}

%%%%%%%%%%%%%%%%%%%%%%%%%%%%%%%%%%%%%%%%%%%%%%%%%%%%%%%%%%%%%%%%%%%%%%%%%%%%%%%%%%%%%%%%%%%%%%%%%%%%%%%%%%%%%%%%%%%%%%%%%%%%%%%%%%%%%%%%%%%%
%%%%%%%%%%%%%%%%%%%%%%%%%%%%%%%%%%%%%%%%%%%%%%%%%%%%%%%%%%%%%%%%%%%%%%%%%%%%%%%%%%%%%%%%%%%%%%%%%%%%%%%%%%%%%%%%%%%%%%%%%%%%%%%%%%%%%%%%%%%%
\section{Bayesova statistika}

\begin{frame}
  \frametitle{Bayesova statistika}
  Bayesovo sklepanje razlaga verjetnost kot merilo verjetnosti ali zaupanja, ki ga lahko ima posameznik glede nastanka določenega dogodka.
  \begin{itemize}
    \item O dogodku lahko že imamo predhodno prepričanje oziroma apriorno prepričanje.
    \item Ta se lahko spremeni, ko se pojavijo novi dokazi.
  \end{itemize} \vspace{3mm}
  Bayesova statistika nam daje matematične modele za vključevanje naših apriornih prepričanj in dokazov za ustvarjanje novih prepričanj.
\end{frame}

%%%%%%%%%%%%%%%%%%%%%%%%%%%%%%%%%%%%%%%%%%%%%%%%%%%%%%%%%%%%%%%%%%%%%%%%%%%%%%%%%%%%%%%%%%%%%%%%%%%%%%%%%%%%%%%%%%%%%%%%%%%%%%%%%%%%%%%%%%%%
\begin{frame}
  \frametitle{Bayesovo pravilo}
Bayesovo sklepanje temelji na Bayesovem pravilu, ki izraža verjetnost nekega dogodka z verjetnostjo dveh dogodkov in obrnejnje pogojne
verjetnosti. Pogojna verjetnost predstavlja verjetnost dogodka, glede na drug dogodek. \vspace{3mm}
S pomočjo formule za pogojno verjetnost dobimo Bayesovo pravilo:
\begin{equation}\label{eq:bpravilo}
    P(H \lvert E) = \frac{P(E \lvert H) \times P(H)}{P(E)},
\end{equation}
verjetnost dogodka E lahko še razpišemo in dobimo:
\begin{equation}\label{eq:b_pravilo}
  P(H \lvert E) = \frac{P(E \lvert H) \times P(H)}{P(E \lvert H)P(H) + P(E \lvert \neg H)P(\neg H)}.
\end{equation} \\
\end{frame}

%%%%%%%%%%%%%%%%%%%%%%%%%%%%%%%%%%%%%%%%%%%%%%%%%%%%%%%%%%%%%%%%%%%%%%%%%%%%%%%%%%%%%%%%%%%%%%%%%%%%%%%%%%%%%%%%%%%%%%%%%%%%%%%%%%%%%%%%%%%%
%%%%%%%%%%%%%%%%%%%%%%%%%%%%%%%%%%%%%%%%%%%%%%%%%%%%%%%%%%%%%%%%%%%%%%%%%%%%%%%%%%%%%%%%%%%%%%%%%%%%%%%%%%%%%%%%%%%%%%%%%%%%%%%%%%%%%%%%%%%%
\section{Bayesova analiza}

\begin{frame}
  \frametitle{Bayesova analiza}
Na voljo imamo dokaz $E$, nas pa zanima pogojna verjetnost
\begin{equation}
  P(kriv \lvert E),
\end{equation}
pri čemer nam je lahko v pomoč Bayesovo pravilo. To v teoriji drži, čeprav je v praksi izračun verjetnostne krivde lahko
preveč zapleten. \vspace{3mm}

Z Bayesovim pravilom lahko ocenimo verjetnosti vmesnih trditev oziroma dokazov, ki so ključnega pomena za ugotavljanje
obtoženčeve krivde.
\end{frame}

%%%%%%%%%%%%%%%%%%%%%%%%%%%%%%%%%%%%%%%%%%%%%%%%%%%%%%%%%%%%%%%%%%%%%%%%%%%%%%%%%%%%%%%%%%%%%%%%%%%%%%%%%%%%%%%%%%%%%%%%%%%%%%%%%%%%%%%%%%%%
\begin{frame}
  \frametitle{Poenostavljena Bayesova analiza}
Naj bo:\\
$S$ \dots trditev, da je obtoženec vir sledi DNK s kraja zločina; \\
$M$ \dots trditev, da se obtoženčev DNK ujema z DNK-jem s kraja zločina; \\
$f$ \dots funkcija pogostosti ujemanja DNK z DNK-jem s kraja zločina. \\
Želimo vedeti, kakšna je verjetnost S glede na M, tj. $P(S \lvert M)$. \\ \vspace{3mm}
Bayesovo pravilo lahko uporabimo na naslednji način:
\[\frac{P(S \lvert M)}{P(\neg S \lvert M)} = \frac{P(M \lvert S)}{P(M \lvert \neg S)} \times \frac{P(S)}{P(\neg S)}.\] \\
\end{frame}

\begin{frame}
  \frametitle{Poenostavljena Bayesova analiza}
\[\frac{P(S \lvert M)}{P(\neg S \lvert M)} = \frac{P(M \lvert S)}{P(M \lvert \neg S)} \times \frac{P(S)}{P(\neg S)}.\] \\
\begin{itemize}
  \item Verjetnosti $P(S)$ in $P(\neg S)$ je težko oceniti, ker ne vemo kakšna je množica osumljencev;
  \item $P(M \lvert S) = 1$ - \textbf{lažno negativni rezultat};
  \item $P(M \lvert \neg S) = f$ - \textbf{lažno pozitivni rezultat}; \vspace{3mm}
\end{itemize}
Ob upoštevanju lažno negativnega in pozitivnega rezultata sledi:
  \[\frac{P(S \lvert M)}{P(\neg S \lvert M)} = \frac{1}{f} \times \frac{P(S)}{P(\neg S)}.\] \\  
\end{frame}

%%%%%%%%%%%%%%%%%%%%%%%%%%%%%%%%%%%%%%%%%%%%%%%%%%%%%%%%%%%%%%%%%%%%%%%%%%%%%%%%%%%%%%%%%%%%%%%%%%%%%%%%%%%%%%%%%%%%%%%%%%%%%%%%%%%%%%%%%%%%
\begin{frame}
  \frametitle{Izpopolnjena Bayesova analiza}
Za upoštevanje možnosti laboratorijskih napak, bomo namesto $M$ uvedli spremenljivko $M_p$.\\
$M_p$ \dots poročano ujemanje laboratorijske analize; \\
$M_t$ \dots trditev, da obstaja dejansko ujemanje v DNK-ju;\\
$\neg M_t$ \dots trditev, da obstaja neujemanje v DNK-ju.  \\ \vspace{3mm}

Sledi:
\[P(M_p \lvert \neg S) = P(M_p \lvert M_t)P(M_t \lvert \neg S) + P(M_p \lvert \neg M_t)P(\neg M_t \lvert \neg S). \vspace{3mm}\]
\end{frame}

\begin{frame}
  \frametitle{Izpopolnjena Bayesova analiza}
\[P(M_p \lvert \neg S) = P(M_p \lvert M_t)P(M_t \lvert \neg S) + P(M_p \lvert \neg M_t)P(\neg M_t \lvert \neg S). \vspace{3mm}\]
  
Sedaj je:
\begin{itemize}
  \item $P(M_t \lvert \neg S) = f$ in zato
  \item $P(\neg M_t \lvert \neg S) = 1-f$; \vspace{2mm}
\end{itemize}  
$P(M_p \lvert \neg M_t) = FP$ \dots verjetnost lažno pozitivnih rezultatov laboratorija\\
$P(M_p \lvert M_t) =  FN$  \dots verjetnost lažno negativnih rezultatov laboratorija. \vspace{3mm}

Sledi: \vspace{2mm}
  \[P(M \lvert \neg S) = [(1 - FN) \times f] + [FP \times (1 - f)]. \vspace{3mm}\]
\end{frame}

%%%%%%%%%%%%%%%%%%%%%%%%%%%%%%%%%%%%%%%%%%%%%%%%%%%%%%%%%%%%%%%%%%%%%%%%%%%%%%%%%%%%%%%%%%%%%%%%%%%%%%%%%%%%%%%%%%%%%%%%%%%%%%%%%%%%%%%%%%%%
%%%%%%%%%%%%%%%%%%%%%%%%%%%%%%%%%%%%%%%%%%%%%%%%%%%%%%%%%%%%%%%%%%%%%%%%%%%%%%%%%%%%%%%%%%%%%%%%%%%%%%%%%%%%%%%%%%%%%%%%%%%%%%%%%%%%%%%%%%%%
\section{Drugi pristopi}

\begin{frame}
  \frametitle{Frekvence in edinstvenost}
\textbf{Naravne frekvence (angl. natural frequencies)}:
\begin{itemize}
  \item bolj naraven način za obravnavo verjetnosti;
  \item potrebno je izračunati, koliko ljudi z zadevnim profilom DNK je v populaciji osumljencem, tako da pogostost profila DNK pomnožimo z 
  velikostjo populacije osumljencev.  \\ \vspace{3mm}
\end{itemize}

\textbf{Edinstvenost (angl. uniqueness method)} \\
V primeru ko je na primer pogostost profila DNK oz. frekvenca še posebaj majhna, je bolje raziskovati ali je profil DNK edinstven ali ne. 
\end{frame}

%%%%%%%%%%%%%%%%%%%%%%%%%%%%%%%%%%%%%%%%%%%%%%%%%%%%%%%%%%%%%%%%%%%%%%%%%%%%%%%%%%%%%%%%%%%%%%%%%%%%%%%%%%%%%%%%%%%%%%%%%%%%%%%%%%%%%%%%%%%%
\begin{frame}
  \frametitle{Naključno ujemanje in razmerja verjetnosti}
\textbf{Metoda verjetnost naključnega ujemanja (angl. random match probability)}:
\begin{itemize}
  \item izraža možnost, da bi imel naključni posameznik, ki ni povezan z obdolžencem, ustrezni DNK profil;
  \item težava tega pristopa je, da verjetnost naključnega ujemanja lahko predstavljena oziroma razumevana narobe. \vspace{3mm}
\end{itemize}

Bayesovo pravilo vključuje razmerje verjetnosti in predhodno oziroma apriorno verjetnost. \vspace{3mm}

\textbf{Metoda razmerja verjetnosti}:
\begin{itemize}
  \item koristna v Veliki Britaniji in ZDA;
  \item Bayesovo pravilo se lahko šteje kot poseg v pravice porote;
  \item poroti naj ne bi bilo potrebno govoriti, kako naj presoja in razmišlja.
\end{itemize}
\end{frame}

%%%%%%%%%%%%%%%%%%%%%%%%%%%%%%%%%%%%%%%%%%%%%%%%%%%%%%%%%%%%%%%%%%%%%%%%%%%%%%%%%%%%%%%%%%%%%%%%%%%%%%%%%%%%%%%%%%%%%%%%%%%%%%%%%%%%%%%%%%%%
%%%%%%%%%%%%%%%%%%%%%%%%%%%%%%%%%%%%%%%%%%%%%%%%%%%%%%%%%%%%%%%%%%%%%%%%%%%%%%%%%%%%%%%%%%%%%%%%%%%%%%%%%%%%%%%%%%%%%%%%%%%%%%%%%%%%%%%%%%%%
\section{Zmote}

\begin{frame}
  \frametitle{Zmote v kazenskem pravu}
Številne zmote so zlasti posledica napačnega razumevanja pogojne verjetnosti.
\begin{itemize}
  \item \textbf{Tožilčeva zmota (angl. Prosecutor’s fallacy)},
  \item \textbf{Zmota obrambnega odvetnika (angl. Defense attorney's fallacy)} in 
  \item \textbf{Zasliševalčeva zmota (angl. Interrogator’s fallacy)}. \vspace{3mm}
\end{itemize}
\end{frame}

%%%%%%%%%%%%%%%%%%%%%%%%%%%%%%%%%%%%%%%%%%%%%%%%%%%%%%%%%%%%%%%%%%%%%%%%%%%%%%%%%%%%%%%%%%%%%%%%%%%%%%%%%%%%%%%%%%%%%%%%%%%%%%%%%%%%%%%%%%%%
\begin{frame}
  \frametitle{Tožilčeva zmota}
\begin{itemize}
  \item Pogosto se pojavlja v kazenskem pravu;
  \item izhaja iz napačnega razumevanja pogojnih verjetnosti in vprašanj večkratnega testiranja; \vspace{3mm}
\end{itemize}
temelji na predpostavki, da je 
\begin{equation}
  P(H \lvert E) = P(E \lvert H)
\end{equation}
$H$ \dots primer, da se najdejo dokazi o obtožencu, \\
$E$ \dots primer, da je obtoženec nedolžen. \vspace{3mm}

Ta enakost ne drži: čeprav je $P(H \lvert E)$ običajno zelo majhna, je lahko $P(E \lvert H)$ še vedno veliko večja. \\ 
\end{frame}

%%%%%%%%%%%%%%%%%%%%%%%%%%%%%%%%%%%%%%%%%%%%%%%%%%%%%%%%%%%%%%%%%%%%%%%%%%%%%%%%%%%%%%%%%%%%%%%%%%%%%%%%%%%%%%%%%%%%%%%%%%%%%%%%%%%%%%%%%%%%
\begin{frame}
  \frametitle{Zmota obrambnega odvetnika}
\begin{itemize}
  \item pojavi se, ko se poroča o tem, koliko ljudi z določeno značilnostjo, se pojavi v določeni populaciji;
  \item predpostavlja se, da je storilec del neke poljubno velike populacije in da ni na voljo drugih informacij, torej je za vse enako verjetno, 
  da so storilci. \vspace{3mm}
\end{itemize}
Na podlagi teh predpostavk lahko sklepamo, da obstaja majhna verjetnost, da je osumljenec storilec kaznivega dejanja. \\
\end{frame}

%%%%%%%%%%%%%%%%%%%%%%%%%%%%%%%%%%%%%%%%%%%%%%%%%%%%%%%%%%%%%%%%%%%%%%%%%%%%%%%%%%%%%%%%%%%%%%%%%%%%%%%%%%%%%%%%%%%%%%%%%%%%%%%%%%%%%%%%%%%%
%%%%%%%%%%%%%%%%%%%%%%%%%%%%%%%%%%%%%%%%%%%%%%%%%%%%%%%%%%%%%%%%%%%%%%%%%%%%%%%%%%%%%%%%%%%%%%%%%%%%%%%%%%%%%%%%%%%%%%%%%%%%%%%%%%%%%%%%%%%%
\section{Primer - Lucia De Berk}

\begin{frame}
  \frametitle{Lucia De Berk}
\begin{itemize}
  \item Medicinska sestra Lucia De Berk obsojena na dosmrtno kazen, zaradi domnevnega ubija več bolnikov v dveh bolnišnicah;
  \item postavilo se je vprašanje: \textbf{Ali je Luciina prisotnost pri toliko smrtih bolnikov naključna?}
  \item postavili so dve hipotezi;
  \item izračuni so bili pomankljivi na večih področjih;
  \item verjetnostni model je bil preveč poenostavljen;
  \item verjetnost, da je osumljenka doživela toliko smrti bolnikov, je bila napačno interpretirana. \vspace{4mm}
\end{itemize}

Zgodila se je tako imenovana \textbf{Tožilčeva zmota}.
\end{frame}

\end{document}