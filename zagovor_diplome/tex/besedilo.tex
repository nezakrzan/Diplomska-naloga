\documentclass[12pt,a4paper]{amsart}
\usepackage[slovene]{babel}
%\usepackage[cp1250]{inputenc}
\usepackage[T1]{fontenc}
\usepackage[utf8]{inputenc}
\usepackage{amsmath,amssymb,amsfonts}
\usepackage{url}
\usepackage[normalem]{ulem}
\usepackage[dvipsnames,usenames]{color}
\usepackage{graphicx}

% Oblika strani
\textwidth 15cm
\textheight 24cm
\oddsidemargin.5cm
\evensidemargin.5cm
\topmargin-5mm
\addtolength{\footskip}{10pt}
\pagestyle{plain}
\overfullrule=15pt % oznaci predlogo vrstico

% Ukazi za matematična okolja
\theoremstyle{definition} % tekst napisan pokončno
\newtheorem{definicija}{Definicija}[section]
\newtheorem{primer}[definicija]{Primer}
\newtheorem{opomba}[definicija]{Opomba}

\renewcommand\endprimer{\hfill$\diamondsuit$}


\theoremstyle{plain} % tekst napisan poševno
\newtheorem{lema}[definicija]{Lema}
\newtheorem{izrek}[definicija]{Izrek}
\newtheorem{trditev}[definicija]{Trditev}
\newtheorem{posledica}[definicija]{Posledica}

\begin{document}

%%%%%%%%%%%%%%%%%%%%%%%%%%%%%%%%%%%%%%%%%%%%%%%%%%%%%%%%%%%%%%%%%%%%%%%%%%%%%%%%%%%%%%%%%%%%%%%%%%%%%%%%%%%%%%%%%%%%%%%%%%%%%%%%%%%%%%%%%%%%%%
\title{Statistika v kazenskem pravu}
\author{Neža Kržan}
\maketitle

%%%%%%%%%%%%%%%%%%%%%%%%%%%%%%%%%%%%%%%%%%%%%%%%%%%%%%%%%%%%%%%%%%%%%%%%%%%%%%%%%%%%%%%%%%%%%%%%%%%%%%%%%%%%%%%%%%%%%%%%%%%%%%%%%%%%%%%%%%%%%%
%%%%%%%%%%%%%%%%%%%%%%%%%%%%%%%%%%%%%%%%%%%%%%%%%%%%%%%%%%%%%%%%%%%%%%%%%%%%%%%%%%%%%%%%%%%%%%%%%%%%%%%%%%%%%%%%%%%%%%%%%%%%%%%%%%%%%%%%%%%%%%
\section{Zmote v kazenskem pravu}
Ker večina ljudi pri razmišljanju o verjetnosti dela osnovne napake, obstaja mnogo zmot, ki izhajajo iz osnovnega razumevanja pravil
teorije verjetnosti. Številne od teh zmot so posledica napačnega razumevanja pogojne verjetnosti. Bolj znana primera takih zmot sta
tožilčeva zmota in zmota obrambnega odvetnika.

%%%%%%%%%%%%%%%%%%%%%%%%%%%%%%%%%%%%%%%%%%%%%%%%%%%%%%%%%%%%%%%%%%%%%%%%%%%%%%%%%%%%%%%%%%%%%%%%%%%%%%%%%%%%%%%%%%%%%%%%%%%%%%%%%%%%%%%%%%%%%%
\subsection{Tožilčeva zmota}
Tožilčeva zmota je bila motivacija za mojo diplomsko nalogo. Izhaja iz napačnega razumevanja
pogojnih verjetnosti, matematično sicer nič posebno, vendar lahko pusti hude posledice v sodnem procesu.\\\\
Če je $E$ dokaz in $H$ trditev,  da je obtoženi nedolžen. \\
Tožilčeva zmota je zamenjava verjetnost dokaza $E$ glede na
hipotezo $H$ z verjetnostjo hipoteze $H$ glede na dokaze $E$ oziroma $P(E \lvert H)$ z $P(H \lvert E)$.\\\\
Pa si poglejmo to na realnem primeru - sodni proces medicinski sestri Lucii de Berg.  Lucia de Berk je bila obsojena na dosmrtni zapor zaradi obtožbe 
uboja več bolnikov v dveh bolnišnicah, v katerih je delala v bližnji preteklosti. \\\\
Sodišče je bilo mnenja, da podan verjetnostni izračun, pomeni, da je osumljenka vse dogodke, navedene v obtožnici, doživela naključno. Ti 
izračuni naj bi poseldično prikazovali, da je velika verjetnost, da obstaja povezava med izmeno osumljenke in pojavom zadevnega dogodka. Ti sklepi sodišča 
bi morali statistikom vzbujati dvome, saj je sodba dvoumna in lahko bi rekla, da je sodišče storilo znano tožilčevo zmoto. Po sklepu sodišča bi lahko rekli, 
da govorijo o verjetnosti, da se je nekaj zgodilo ob predpostavki, da je vse popolnoma naključno ali pa si sodbo sodišča razlagamo kot verjetnost, da se 
je nekaj naključno zgodilo. Ti dve trditvi pa sta različni, kar lahko pokažem z naslednjimi formulami. Naj bo\\
$F$ \dots opazovani dogodek;\\
$H_0$ \dots trditev, da se dogodek zgodi naključno.\\
Statistik je izračunal verjetnost $P(F \lvert H_0)$, medtem ko je sodišče mislilo, da je izračunana  verjetnost v bistvu $P(H_0 \lvert F)$, 
kar pa je definicija tožilčeve zmote.

%%%%%%%%%%%%%%%%%%%%%%%%%%%%%%%%%%%%%%%%%%%%%%%%%%%%%%%%%%%%%%%%%%%%%%%%%%%%%%%%%%%%%%%%%%%%%%%%%%%%%%%%%%%%%%%%%%%%%%%%%%%%%%%%%%%%%%%%%%%%%%
%%%%%%%%%%%%%%%%%%%%%%%%%%%%%%%%%%%%%%%%%%%%%%%%%%%%%%%%%%%%%%%%%%%%%%%%%%%%%%%%%%%%%%%%%%%%%%%%%%%%%%%%%%%%%%%%%%%%%%%%%%%%%%%%%%%%%%%%%%%%%%
\section{Statistika v kazenskem pravu}
Statistiki si običajno prizadevajo preučiti razmerja med dvema ali več spremenljivkami.
\begin{definicija}
    \textit{Odvisna spremenljivka} je pojav, ki ga želi statistik preučiti, razložiti ali napovedati.
\end{definicija}
\begin{definicija}
    \textit{Neodvisna spremenljivka} je dejavnik ali značilnost, s katero se poskuša pojasniti ali napovedati odvisno spremenljivko.
\end{definicija}
Pomembno je razumeti, da neodvisno in odvisno nista sinonima za vzrok in posledico. Določene neodvisne spremenljivke so lahko povezane z
določenimi odvisnimi spremenljivkami, vendar to še zdaleč ni dokončen dokaz, da so prve vzrok drugih. Za dokazovanje vzročnosti morajo
statistiki dokazati, da njihove študije izpolnjujejo tri merila. Prvo je časovno zaporedje, druga zahteva glede vzročnosti je, da obstaja empirična povezava med neodvisno in odvisno spremenljivko.
Zadnja zahteva je, da je razmerje med neodvisno spremenljivko in odvisno spremenljivko nepristransko.\\\\
Statistiki se že na začetku sodnega procesa soočajo s prvimi težavami - določitvijo odvisnih in neodvisnih spremenljivk za
modeliranje. V proces določanja spremenljivk pa pogosto v preveliki meri posegajo odvetniki, ki se sklicujejo na pravne zakone in načela. To lahko
postane sporno, saj lahko takšni pretirani posegi ovirajo statistične znanstvenike pri izračunu verjetnostnega vpliva spremenljivk. Zagotovo 
določitev spremenljivk ne sme biti naloga le odvetnikov ali le statistikov, ampak menim, da je sodelovanje med statistiki in odvetniki pomembno.  S tem se lahko zagotovi pravilno opredelitev spremenljivk in pravilne verjetnostne 
izračune, ki bodo prispevali k pravičnim odločitvam v sodnih postopkih.

%%%%%%%%%%%%%%%%%%%%%%%%%%%%%%%%%%%%%%%%%%%%%%%%%%%%%%%%%%%%%%%%%%%%%%%%%%%%%%%%%%%%%%%%%%%%%%%%%%%%%%%%%%%%%%%%%%%%%%%%%%%%%%%%%%%%%%%%%%%%%%
%%%%%%%%%%%%%%%%%%%%%%%%%%%%%%%%%%%%%%%%%%%%%%%%%%%%%%%%%%%%%%%%%%%%%%%%%%%%%%%%%%%%%%%%%%%%%%%%%%%%%%%%%%%%%%%%%%%%%%%%%%%%%%%%%%%%%%%%%%%%%%
\section{Koncept verjetnosti}
Pogosto se opravlja primerjava verjetnosti dokazov na podlagi dveh konkurenčnih predlogov, in sicer predloga tožilca in predloga obrambe.\\\\
$H_p \dots$ trditev, ki jo predlaga tožilstvo;\\
$H_d \dots$ trditev, ki jo predlaga obramba;\\\\
V splošnem nas zanima vpliv dokazov na verjetnost krivde($H_p$) in nedolžnosti($H_d$) osumljenca. Gre za dopolnjujoča se dogodka in razmerje verjetij 
teh dveh dogodkov,
\begin{equation}
   \frac{P(H_p)}{P(H_d)}, \vspace{2mm}
\end{equation}
je verjetnost proti nedolžnosti ali verjetnost za krivdo. Ob upoštevanju dodatnih informacij $E$ oziroma dokazov, je razmerje
\begin{equation}
   \frac{P(H_p \lvert E)}{P(H_d \lvert E)} \vspace{2mm},
\end{equation}
verjetnost v prid krivdi ob upoštevanju dokazov $E$.\\\\
Ali je obtoženec kriv glede na znan doka $E$, je glavna stvar, ki nas pri sojenju zanima. Če imamo torej na voljo dokaz $E$, nas zanima pogojna 
verjetnost
\[
    P(kriv \lvert E), \vspace{2mm}
\]
pri čemer nam je lahko v pomoč Bayesovo pravilo. To v teoriji drži, čeprav je v praksi izračun verjetnostne krivde lahko preveč zapleten. Ampak 
z Bayesovim pravilom lahko ocenimo verjetnosti vmesnih trditev oziroma dokazov, ki so ključnega pomena za ugotavljanje obtoženčeve krivde.

%%%%%%%%%%%%%%%%%%%%%%%%%%%%%%%%%%%%%%%%%%%%%%%%%%%%%%%%%%%%%%%%%%%%%%%%%%%%%%%%%%%%%%%%%%%%%%%%%%%%%%%%%%%%%%%%%%%%%%%%%%%%%%%%%%%%%%%%%%%%%%
%%%%%%%%%%%%%%%%%%%%%%%%%%%%%%%%%%%%%%%%%%%%%%%%%%%%%%%%%%%%%%%%%%%%%%%%%%%%%%%%%%%%%%%%%%%%%%%%%%%%%%%%%%%%%%%%%%%%%%%%%%%%%%%%%%%%%%%%%%%%%%
\section{Bayesova statistika}

%%%%%%%%%%%%%%%%%%%%%%%%%%%%%%%%%%%%%%%%%%%%%%%%%%%%%%%%%%%%%%%%%%%%%%%%%%%%%%%%%%%%%%%%%%%%%%%%%%%%%%%%%%%%%%%%%%%%%%%%%%%%%%%%%%%%%%%%%%%%%%
\subsection{Opredelitev}
Bayesova analiza je standardna metoda za posodabljanje verjetnosti po opazovanju več dokazov, zato je zelo primerna za obravnavo in vrednotenje 
dokazov.
Začnemo z nekim predhodnim prepričanjem o hipotezi in ga posodabljamo, ko se dokazi ponovno pojavijo. Pri uporabi Bayesovega sklepanja morajo 
statistiki utemeljiti predhodne predpostavke.

%%%%%%%%%%%%%%%%%%%%%%%%%%%%%%%%%%%%%%%%%%%%%%%%%%%%%%%%%%%%%%%%%%%%%%%%%%%%%%%%%%%%%%%%%%%%%%%%%%%%%%%%%%%%%%%%%%%%%%%%%%%%%%%%%%%%%%%%%%%%%%
\subsection{Bayesovo pravilo}
Bayesovo sklepanje temelji na Bayesovem pravilu, ki izraža verjetnost nekega dogodka z verjetnostjo dveh dogodkov in obrnjene pogojne
verjetnosti.
\begin{izrek}
    (Bayesovo pravilo)
    \begin{equation}\label{eq:bpravilo}
        P(H \lvert E) = \frac{P(E \lvert H) \times P(H)}{P(E)}. \vspace{2mm}
     \end{equation}
\end{izrek}

%%%%%%%%%%%%%%%%%%%%%%%%%%%%%%%%%%%%%%%%%%%%%%%%%%%%%%%%%%%%%%%%%%%%%%%%%%%%%%%%%%%%%%%%%%%%%%%%%%%%%%%%%%%%%%%%%%%%%%%%%%%%%%%%%%%%%%%%%%%%%%
\subsection{Bayesovo posodabljanje}
Bayesovo posodabljanje je logična trditev, kako se sčasoma posodabljajo
apriorne oziroma predhodne verjetnosti dokazov glede na novo zbrane dokaze oziroma prepričanja.
\begin{definicija}
    (Bayesovo posodabljanje)
    Če se dogodek E zgodi ob času $t_1 > t_0$, potem je $P_1(H) = P_0(H \lvert E)$.
\end{definicija}
Ob času $t_0$ dogodku H dodelimo verjetnost $P_0(H)$; to se imenuje predhodna verjetnost oziroma apriorna verjetnost. Ko se zgodi dogodek E
ob času $t_1$, ki vpliva na naša prepričanja o dogodku H, Bayesovo posodabljanje pravi, da je potrebno apriorno verjetnost dogodka H v času $t_1$
enačiti s pogojno verjetnostjo dogodka H glede na dogodek E v času $t_0$. \\

%%%%%%%%%%%%%%%%%%%%%%%%%%%%%%%%%%%%%%%%%%%%%%%%%%%%%%%%%%%%%%%%%%%%%%%%%%%%%%%%%%%%%%%%%%%%%%%%%%%%%%%%%%%%%%%%%%%%%%%%%%%%%%%%%%%%%%%%%%%%%%
\subsection{Bayesova teorija v kazenskem pravu}
Gre za postopek posodabljanja verjetnosti tožilčeve hipoteze na podlagi predhodnih verjetnosti.\\

%%%%%%%%%%%%%%%%%%%%%%%%%%%%%%%%%%%%%%%%%%%%%%%%%%%%%%%%%%%%%%%%%%%%%%%%%%%%%%%%%%%%%%%%%%%%%%%%%%%%%%%%%%%%%%%%%%%%%%%%%%%%%%%%%%%%%%%%%%%%%%
\subsection{Predhodna verjetnost in določitev aposteriorne verjetnosti}
\begin{definicija}
    Predhodna verjetnost, ki je uporabljena v vsaki posodobitvi verjetnosti s pomočjo Bayesove teorije, je začetna verjetnost hipoteze 
    oziroma tožilčeve domneve o obtožencu oziroma storilcu kaznivega dejanja.
\end{definicija}
Določitev predhodnih oziroma apriornih verjetnosti je resen problem pri Bayesovemu pristopu v kazenskih postopkih. Različne metode za določitev 
in izračun teh verjetnosti lahko dajejo rezultate, ki se med seboj precej razlikujejo, kar pa je problematično, ker celotna Bayesova teorija 
temelji ravno na teh začetnih izračunih.\\\\
Bistveno vprašanje, ki se postavlja, je, ali naj analitiki sploh poskušajo določiti predhodne verjetnosti in če ja, kako naj jih določijo. Nekateri 
strokovnjaki predlagajo, da bi analitiki morali predpostaviti enake predhodne verjetnosti za vse hipoteze v primeru, kar se imenuje nevtralno stanje. 
To bi se lahko izkazalo za praktičen pristop, saj se lahko statistik s tem izogne vplivu lastnih in odvetniških predsodkov ter mnenj, ki bi lahko 
vplivali na predpostavke o verjetnosti. Na ta način se lahko zagotovi objektivnost analize, saj ne poskušamo prikazati ene hipoteze bolj verjetne 
od druge. Kljub temu pa mislim, da moramo biti do tega pristopa nekoliko kritični, saj je predpostavljanje enake verjetnosti za vse možnosti 
problematično - v realnosti se različne hipoteze razlikujejo po svoji verjetnosti.\\
Ker obstajajo pravila in smernice, kako upoštevati zakonodajo, pravila in postopke sodnega procesa, naj statistik uporabi svoje strokovno znanje za 
izračun apriorne verjetnosti na podlagi razpoložljivih podatkov in brez nepotrebnega vplivanja odvetnikov ali drugih udeležencev postopka. \\\\
Tekom sodnega procesa se pojavljajo nove domneve o obtožencu in novi dokazi s kraja zločina. Smiselno je, da vse to v postopku 
izračuna tudi upoštevamo. Zasledila sem, da se tekom sodnega procesa marsikateri dokaz najprej prizna in je znan sodniku, ki presoja tožilčevo domnevo 
o obtožencu, torej ga upoštevajo v svojih izračunih za posodobitve predhodnih verjetnosti hipotez. Potem pa dokaz iz sodnega procesa umaknejo, ampak 
presnetilo me je, da dokaz največkrat ni umaknjen iz verjetnostnih računov. Mnenja sem, da bi morali statistiki, ko se določen dokaz iz 
sodnega procesa, zaradi tehtnega razloga, umakne, posodobiti vse račune za nazaj in nato nadaljevati posodabljanje verjetnosti. Tako bi dobili primeren 
izračun posteriornih verjetnosti.

%%%%%%%%%%%%%%%%%%%%%%%%%%%%%%%%%%%%%%%%%%%%%%%%%%%%%%%%%%%%%%%%%%%%%%%%%%%%%%%%%%%%%%%%%%%%%%%%%%%%%%%%%%%%%%%%%%%%%%%%%%%%%%%%%%%%%%%%%%%%%%
%%%%%%%%%%%%%%%%%%%%%%%%%%%%%%%%%%%%%%%%%%%%%%%%%%%%%%%%%%%%%%%%%%%%%%%%%%%%%%%%%%%%%%%%%%%%%%%%%%%%%%%%%%%%%%%%%%%%%%%%%%%%%%%%%%%%%%%%%%%%%%
\section{Razmerje verjetij}
Občasno se zgodi, da predloga tožilstva in obrambe nista komplementarna in v takih primerih ni mogoče določiti $P(H_p)$ ali $P(H_d)$, 
ampak samo vpliv statistike, znane kot razmerje verjetij.

%%%%%%%%%%%%%%%%%%%%%%%%%%%%%%%%%%%%%%%%%%%%%%%%%%%%%%%%%%%%%%%%%%%%%%%%%%%%%%%%%%%%%%%%%%%%%%%%%%%%%%%%%%%%%%%%%%%%%%%%%%%%%%%%%%%%%%%%%%%%%%
\subsection{Razmerje verjetij v kazenskem pravu}
Sedaj si poglejmo obliko Bayesovega izreka o razmerju verjetij v forenzičnem kontekstu, tj. ocenjevanje vrednosti nekaterih dokazov.\\\\
Naj bo:\\
$H_p \dots$ obtoženec je resnično kriv - nadomestimo $H$;\\
$H_d \dots$ obtoženec je resnično nedolžen - nadomestimo $\bar{H}$;\\
$Ev \dots$ obravnavani dokaz - nadomestimo dogodek $E$;\\\\
Ker ima vsak kazenski primer specifične posebnosti, moramo v enačbo vpeljati še t.i. informacije o ozadju, $I$. Če jih ne upoštevamo, lahko posamezno 
vrednotenje dokazov postane preveč splošno. Ob upoštevanju informacij o ozadju $I$ dobimo zapis
\[
   \frac{P(H_p \lvert Ev, I)}{P(H_d \lvert Ev, I)} = \frac{P(Ev \lvert H_p, I)}{P(Ev \lvert H_d, I)} \times \frac{P(H_p \lvert I)}{P(H_d \lvert I)}. \vspace{2mm}
\]
Da lahko ocenimo oziroma določimo vrednost dokaza, potrebujemo razmerje verjetij.
\begin{definicija}
    Naj bosta  $H_p$ in $H_d$ dve konkurenčni hipotezi ter $I$ informacije o ozadju. Vrednost $V$ dokaza $Ev$ je podana z
    \[
        V = \frac{P(Ev \lvert H_p, I)}{P(Ev \lvert H_d, I)}, \vspace{2mm}
    \]
    to je razmerje verjetij, ki pretvori predhodno razmerje
    \[
        \frac{P(H_p \lvert I)}{P(H_d \lvert I)} \vspace{2mm}
    \]
    v aposteriorno razmerje
    \[
        \frac{P(H_p \lvert Ev, I)}{P(H_d \lvert Ev, I)}.
    \]
\end{definicija}
Razmerje verjetij nam pove, katera hipoteza je bolje podprta z dokazi. Iz Bayesovega izreka neposredno izhaja, da če je razmerje verjetij večje od 1, potem dokaz povečuje
»verjetnost« krivde (pri čemer višje vrednosti pomenijo večjo verjetnost krivde), če pa je manjše od 1, zmanjšuje »verjetnost« krivde
(in bolj ko se približuje ničli, manjša je »verjetnost« krivde).\\\\
Ker je ocena vrednosti razmerja verjetij lahko podvržena številnim virom negotovosti, mora poročilo o vrednosti razmerja verjetij vključevati merilo 
njegove natančnosti, na primer  na primer z navedbo številčnega razpona vrednosti za verjetnost dokazov na podlagi konkurenčnih predlogov in s tem 
številčnega razpona vrednosti za razmerje verjetij.

%%%%%%%%%%%%%%%%%%%%%%%%%%%%%%%%%%%%%%%%%%%%%%%%%%%%%%%%%%%%%%%%%%%%%%%%%%%%%%%%%%%%%%%%%%%%%%%%%%%%%%%%%%%%%%%%%%%%%%%%%%%%%%%%%%%%%%%%%%%%%%
%%%%%%%%%%%%%%%%%%%%%%%%%%%%%%%%%%%%%%%%%%%%%%%%%%%%%%%%%%%%%%%%%%%%%%%%%%%%%%%%%%%%%%%%%%%%%%%%%%%%%%%%%%%%%%%%%%%%%%%%%%%%%%%%%%%%%%%%%%%%%%
\section{Druge metode}
Da sem lahko ocenila ustreznost Bayesove statistike, sem jo primerjala z drugimi metodatmi.\\\\
Velikokrat uporabljena metoda temelji na relativnih frekvencah. Relativne frekvence vedno navajajo ali predpostavljajo, da obstaja nek
referenčni vzorec, na podlagi katerega se lahko oceni pogostost zadevnega dogodka. V okviru kazenskega postopka relativna frekvenca lahko podpre
vmesno sklepanje o moči dokazov, kar nam pomaga pri končnem sklepu.\\\\
Metoda verjetnosti naključnega ujemanja izraža možnost, da bi imel naključni posameznik, ki ni povezan z obdolžencem,
ustrezni DNK profil. Težava tega pristopa je, da je verjetnost naključnega ujemanja lahko predstavljena
oziroma razumevana narobe. Gre za zamenjavo dveh pogojnih verjetnosti oziroma tožilčevo zmoto.

%%%%%%%%%%%%%%%%%%%%%%%%%%%%%%%%%%%%%%%%%%%%%%%%%%%%%%%%%%%%%%%%%%%%%%%%%%%%%%%%%%%%%%%%%%%%%%%%%%%%%%%%%%%%%%%%%%%%%%%%%%%%%%%%%%%%%%%%%%%%%%
%%%%%%%%%%%%%%%%%%%%%%%%%%%%%%%%%%%%%%%%%%%%%%%%%%%%%%%%%%%%%%%%%%%%%%%%%%%%%%%%%%%%%%%%%%%%%%%%%%%%%%%%%%%%%%%%%%%%%%%%%%%%%%%%%%%%%%%%%%%%%%

%%%%%%%%%%%%%%%%%%%%%%%%%%%%%%%%%%%%%%%%%%%%%%%%%%%%%%%%%%%%%%%%%%%%%%%%%%%%%%%%%%%%%%%%%%%%%%%%%%%%%%%%%%%%%%%%%%%%%%%%%%%%%%%%%%%%%%%%%%%%%%
%%%%%%%%%%%%%%%%%%%%%%%%%%%%%%%%%%%%%%%%%%%%%%%%%%%%%%%%%%%%%%%%%%%%%%%%%%%%%%%%%%%%%%%%%%%%%%%%%%%%%%%%%%%%%%%%%%%%%%%%%%%%%%%%%%%%%%%%%%%%%%
\section{Izogib zmotam z uporabo Bayesovih omrežij}
Ker je največkrat težava v tem, da se večina odvetnikov in sodnikov ob pogledu na verjetnostne izračune in statistično analizo dokazov ustraši, se mi 
zdijo Bayeosova omrežja dober predlog za predstavitev verjetnostnih izračunov.

%%%%%%%%%%%%%%%%%%%%%%%%%%%%%%%%%%%%%%%%%%%%%%%%%%%%%%%%%%%%%%%%%%%%%%%%%%%%%%%%%%%%%%%%%%%%%%%%%%%%%%%%%%%%%%%%%%%%%%%%%%%%%%%%%%%%%%%%%%%%%%
\subsection{Opredelitev}
Bayesova omrežja pomagajo določiti ustrezne verjetnostne formule, ne da bi prikazali njihovo polno algebrsko obliko, in omogočajo
skoraj popolno avtomatizacijo potrebnih verjetnostnih izračunov.
\begin{definicija}
   Bayesovo omrežje je verjetnostni grafični model, ki predstavlja množico spremenljivk in njihovih pogojnih odvisnosti prek usmerjenega
   acikličnega grafa.
\end{definicija}

%%%%%%%%%%%%%%%%%%%%%%%%%%%%%%%%%%%%%%%%%%%%%%%%%%%%%%%%%%%%%%%%%%%%%%%%%%%%%%%%%%%%%%%%%%%%%%%%%%%%%%%%%%%%%%%%%%%%%%%%%%%%%%%%%%%%%%%%%%%%%%
\subsection{Uporaba Bayesovih omrežij na sodišču}
Tem metodam se daje poseben poudarek, kadar je treba med konkurenčnimi hipotezami izbrati najverjetnejšo, izbira pa mora biti podprta z znanstveno utemeljeno
argumentacijo. Primerna so za analizo dogodka, ki se je zgodil, in napovedovanje verjetnosti, da je k temu prispeval katerikoli od več možnih
znanih vzrokov. Prednosti Bayesovih mrež se najbolj izrazito pokažejo na zapletenih področjih z več spremenljivkami.\\\\
Pomembno je, da za izdelavo uporabimo dosleden okvir, drugače lahko Bayesova omrežja, ki jih oblikujejo različne stranke za en
primer, kažejo različne rezultate.

%%%%%%%%%%%%%%%%%%%%%%%%%%%%%%%%%%%%%%%%%%%%%%%%%%%%%%%%%%%%%%%%%%%%%%%%%%%%%%%%%%%%%%%%%%%%%%%%%%%%%%%%%%%%%%%%%%%%%%%%%%%%%%%%%%%%%%%%%%%%%%
%%%%%%%%%%%%%%%%%%%%%%%%%%%%%%%%%%%%%%%%%%%%%%%%%%%%%%%%%%%%%%%%%%%%%%%%%%%%%%%%%%%%%%%%%%%%%%%%%%%%%%%%%%%%%%%%%%%%%%%%%%%%%%%%%%%%%%%%%%%%%%
\section{Zaključek}
V diplomski nalogi ugotovim, na podlagi pregleda nekaterih drugih metod, da je Bayesov pristop na splošno najboljši za vrednotenje dokazov, še posebej v primeru,
ko se dokazi nanašajo na DNK ali laboratorijske analize. V delu pridem do ugotovitve, da je mogoče najbolje, da za določitev in izračun prehodnih verjetnosti poskrbijo
statistiki, saj znajo v svoje modele vključevati tudi pravno teorijo. Težave pa nastanejo pri vključevanju novih dokazov, ki so sodišču priloženi
tekom sodnega procesa in izključevanju nepomembnih oziroma nepotrebnih dokazov za sodni proces.\\
Ker so me predvsem zanimale zmote, do katerih prihaja, sem veliko časa posvetila ugotavljanju najboljšega načina za izogib zmotam. In glavna ugotovitev dela je, da se lahko 
s pomočjo Bayesovih omrežij izognemo zmotam, saj ne prikažejo polne algebrske oblike verjetnostne formule, omogočajo pa skoraj popolno
avtomatizacijo potrebnih verjetnostnih izračunov in več prefinjenih možnosti.

\end{document}