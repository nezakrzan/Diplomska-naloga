\documentclass[12pt,a4paper]{amsart}
\usepackage[slovene]{babel}
%\usepackage[cp1250]{inputenc}
\usepackage[T1]{fontenc}
\usepackage[utf8]{inputenc}
\usepackage{amsmath,amssymb,amsfonts}
\usepackage{url}
\usepackage[normalem]{ulem}
\usepackage[dvipsnames,usenames]{color}
\usepackage{graphicx}

% Oblika strani
\textwidth 15cm
\textheight 24cm
\oddsidemargin.5cm
\evensidemargin.5cm
\topmargin-5mm
\addtolength{\footskip}{10pt}
\pagestyle{plain}
\overfullrule=15pt % oznaci predlogo vrstico

% Ukazi za matematična okolja
\theoremstyle{definition} % tekst napisan pokončno
\newtheorem{definicija}{Definicija}[section]
\newtheorem{primer}[definicija]{Primer}
\newtheorem{opomba}[definicija]{Opomba}

\renewcommand\endprimer{\hfill$\diamondsuit$}


\theoremstyle{plain} % tekst napisan poševno
\newtheorem{lema}[definicija]{Lema}
\newtheorem{izrek}[definicija]{Izrek}
\newtheorem{trditev}[definicija]{Trditev}
\newtheorem{posledica}[definicija]{Posledica}

\begin{document}

%%%%%%%%%%%%%%%%%%%%%%%%%%%%%%%%%%%%%%%%%%%%%%%%%%%%%%%%%%%%%%%%%%%%%%%%%%%%%%%%%%%%%%%%%%%%%%%%%%%%%%%%%%%%%%%%%%%%%%%%%%%%%%%%%%%%%%%%%%%%%%
\title{Statistika v kazenskem pravu}
\author{Neža Kržan}
\maketitle

%%%%%%%%%%%%%%%%%%%%%%%%%%%%%%%%%%%%%%%%%%%%%%%%%%%%%%%%%%%%%%%%%%%%%%%%%%%%%%%%%%%%%%%%%%%%%%%%%%%%%%%%%%%%%%%%%%%%%%%%%%%%%%%%%%%%%%%%%%%%%%
%%%%%%%%%%%%%%%%%%%%%%%%%%%%%%%%%%%%%%%%%%%%%%%%%%%%%%%%%%%%%%%%%%%%%%%%%%%%%%%%%%%%%%%%%%%%%%%%%%%%%%%%%%%%%%%%%%%%%%%%%%%%%%%%%%%%%%%%%%%%%%
\section{Statistika v kazenskem pravu}
Velik del raziskav vključuje preverjanje teorije in hipotez. Statistiki si običajno prizadevajo preučiti razmerja med dvema ali več spremenljivkami.
\begin{definicija}
    \textit{Odvisna spremenljivka} je pojav, ki ga želi statistik preučiti, razložiti ali napovedati.
\end{definicija}
\begin{definicija}
    \textit{Neodvisna spremenljivka} je dejavnik ali značilnost, s katero se poskuša pojasniti ali napovedati odvisno spremenljivko.
\end{definicija}
Pomembno je razumeti, da neodvisno in odvisno nista sinonima za vzrok in posledico. Določene neodvisne spremenljivke so lahko povezane z
določenimi odvisnimi spremenljivkami, vendar to še zdaleč ni dokončen dokaz, da so prve vzrok drugih. Za dokazovanje vzročnosti morajo
statistiki dokazati, da njihove študije izpolnjujejo tri merila. Prvo je časovno zaporedje, druga zahteva glede vzročnosti je, da obstaja empirična povezava med neodvisno in odvisno spremenljivko.
Zadnja zahteva je, da je razmerje med neodvisno spremenljivko in odvisno spremenljivko nepristransko.\\\\
Statistiki se že na začetku sodnega procesa soočajo s prvimi težavami - določitvijo odvisnih in neodvisnih spremenljivk za
modeliranje. V proces določanja spremenljivk pa pogosto v preveliki meri posegajo odvetniki, ki se sklicujejo na pravne zakone in načela. To lahko
postane sporno, saj lahko takšni pretirani posegi ovirajo statistične znanstvenike pri izračunu verjetnostnega vpliva spremenljivk. Zagotovo 
določitev spremenljivk ne sme biti naloga le odvetnikov ali le statistikov, ampak menim, da je sodelovanje med statistiki in odvetniki pomembno.  S tem se lahko zagotovi pravilno opredelitev spremenljivk in pravilne verjetnostne 
izračune, ki bodo prispevali k pravičnim odločitvam v sodnih postopkih.

%%%%%%%%%%%%%%%%%%%%%%%%%%%%%%%%%%%%%%%%%%%%%%%%%%%%%%%%%%%%%%%%%%%%%%%%%%%%%%%%%%%%%%%%%%%%%%%%%%%%%%%%%%%%%%%%%%%%%%%%%%%%%%%%%%%%%%%%%%%%%%
%%%%%%%%%%%%%%%%%%%%%%%%%%%%%%%%%%%%%%%%%%%%%%%%%%%%%%%%%%%%%%%%%%%%%%%%%%%%%%%%%%%%%%%%%%%%%%%%%%%%%%%%%%%%%%%%%%%%%%%%%%%%%%%%%%%%%%%%%%%%%%
\section{Raziskovalni proces}
Raziskovalni proces se izvaja po naslednjih točkah.\\
\textbf{1. Identifikacija problema.\\} 
\textbf{2. Zasnova raziskave.\\} 
\textbf{3. Analiza podatkov.\\}

%%%%%%%%%%%%%%%%%%%%%%%%%%%%%%%%%%%%%%%%%%%%%%%%%%%%%%%%%%%%%%%%%%%%%%%%%%%%%%%%%%%%%%%%%%%%%%%%%%%%%%%%%%%%%%%%%%%%%%%%%%%%%%%%%%%%%%%%%%%%%%
%%%%%%%%%%%%%%%%%%%%%%%%%%%%%%%%%%%%%%%%%%%%%%%%%%%%%%%%%%%%%%%%%%%%%%%%%%%%%%%%%%%%%%%%%%%%%%%%%%%%%%%%%%%%%%%%%%%%%%%%%%%%%%%%%%%%%%%%%%%%%%
\section{Vrednotenje dokazov}
Najpogostejši uporabljeni metodi za ocenjevanje dokazov sta metoda dokazne vrednosti in model verjetnosti hipoteze.\\
Metoda dokazne vrednosti temelji na vrednosti, ki jo ima dokaz za dokazno temo, njen namen pa je ugotoviti, ali med dokazom in zadevno dokazno temo 
obstaja naključna povezava, pri čemer je dokazna tema glavna obtožba v kazenskem primeru. S to metodo dokazujemo določen omejen nabor dokazov, njen cilj pa 
je oceniti verjetnost, da dokazi dokazujejo hipotezo.\\
Z modelom verjetnosti hipoteze pa ocenjujemo verjetnost hipoteze glede na dokaze. Cilj je ugotoviti, kako verjetno je, da je hipoteza, za katero dokazi 
zagotavljajo določeno stopnjo podpore, resnična. Glavna razlika z metodo dokazne vrednosti je, da predpostavlja, da obstaja začetna verjetnost za hipotezo 
pred obravnavo dokazov, t.i. predhodna verjetnost.\\\\

%%%%%%%%%%%%%%%%%%%%%%%%%%%%%%%%%%%%%%%%%%%%%%%%%%%%%%%%%%%%%%%%%%%%%%%%%%%%%%%%%%%%%%%%%%%%%%%%%%%%%%%%%%%%%%%%%%%%%%%%%%%%%%%%%%%%%%%%%%%%%%
%%%%%%%%%%%%%%%%%%%%%%%%%%%%%%%%%%%%%%%%%%%%%%%%%%%%%%%%%%%%%%%%%%%%%%%%%%%%%%%%%%%%%%%%%%%%%%%%%%%%%%%%%%%%%%%%%%%%%%%%%%%%%%%%%%%%%%%%%%%%%%
\section{Koncept verjetnosti}
Pogosto se opravlja primerjava verjetnosti dokazov na podlagi dveh konkurenčnih predlogov, in sicer predloga tožilca in predloga obrambe.\\\\
$H_p \dots$ trditev, ki jo predlaga tožilstvo;\\
$H_d \dots$ trditev, ki jo predlaga obramba;\\\\
V splošnem nas zanima vpliv dokazov na verjetnost krivde($H_p$) in nedolžnosti($H_d$) osumljenca. Gre za dopolnjujoča se dogodka in razmerje verjetij 
teh dveh dogodkov,
\begin{equation}
   \frac{P(H_p)}{P(H_d)}, \vspace{2mm}
\end{equation}
je verjetnost proti nedolžnosti ali verjetnost za krivdo. Ob upoštevanju dodatnih informacij $E$ oziroma dokazov, je razmerje
\begin{equation}
   \frac{P(H_p \lvert E)}{P(H_d \lvert E)} \vspace{2mm},
\end{equation}
verjetnost v prid krivdi ob upoštevanju dokazov $E$.\\\\
Ali je obtoženec kriv glede na znan doka $E$, je glavna stvar, ki nas pri sojenju zanima. Če imamo torej na voljo dokaz $E$, nas zanima pogojna 
verjetnost
\[
    P(kriv \lvert E), \vspace{2mm}
\]
pri čemer nam je lahko v pomoč Bayesovo pravilo. To v teoriji drži, čeprav je v praksi izračun verjetnostne krivde lahko preveč zapleten. Ampak 
z Bayesovim pravilom lahko ocenimo verjetnosti vmesnih trditev oziroma dokazov, ki so ključnega pomena za ugotavljanje obtoženčeve krivde.

%%%%%%%%%%%%%%%%%%%%%%%%%%%%%%%%%%%%%%%%%%%%%%%%%%%%%%%%%%%%%%%%%%%%%%%%%%%%%%%%%%%%%%%%%%%%%%%%%%%%%%%%%%%%%%%%%%%%%%%%%%%%%%%%%%%%%%%%%%%%%%
%%%%%%%%%%%%%%%%%%%%%%%%%%%%%%%%%%%%%%%%%%%%%%%%%%%%%%%%%%%%%%%%%%%%%%%%%%%%%%%%%%%%%%%%%%%%%%%%%%%%%%%%%%%%%%%%%%%%%%%%%%%%%%%%%%%%%%%%%%%%%%
\section{Bayesova statistika}

%%%%%%%%%%%%%%%%%%%%%%%%%%%%%%%%%%%%%%%%%%%%%%%%%%%%%%%%%%%%%%%%%%%%%%%%%%%%%%%%%%%%%%%%%%%%%%%%%%%%%%%%%%%%%%%%%%%%%%%%%%%%%%%%%%%%%%%%%%%%%%
\subsection{Opredelitev}
Bayesova analiza je standardna metoda za posodabljanje verjetnosti po opazovanju več dokazov, zato je zelo primerna za obravnavo in vrednotenje 
dokazov.
Začnemo z nekim predhodnim prepričanjem o hipotezi in ga posodabljamo, ko se dokazi ponovno pojavijo. Pri uporabi Bayesovega sklepanja morajo 
statistiki utemeljiti predhodne predpostavke.

%%%%%%%%%%%%%%%%%%%%%%%%%%%%%%%%%%%%%%%%%%%%%%%%%%%%%%%%%%%%%%%%%%%%%%%%%%%%%%%%%%%%%%%%%%%%%%%%%%%%%%%%%%%%%%%%%%%%%%%%%%%%%%%%%%%%%%%%%%%%%%
\subsection{Bayesovo pravilo}
Bayesovo sklepanje temelji na Bayesovem pravilu, ki izraža verjetnost nekega dogodka z verjetnostjo dveh dogodkov in obrnjene pogojne
verjetnosti.
\begin{izrek}
    (Bayesovo pravilo)
    \begin{equation}\label{eq:bpravilo}
        P(H \lvert E) = \frac{P(E \lvert H) \times P(H)}{P(E)}. \vspace{2mm}
     \end{equation}
\end{izrek}

%%%%%%%%%%%%%%%%%%%%%%%%%%%%%%%%%%%%%%%%%%%%%%%%%%%%%%%%%%%%%%%%%%%%%%%%%%%%%%%%%%%%%%%%%%%%%%%%%%%%%%%%%%%%%%%%%%%%%%%%%%%%%%%%%%%%%%%%%%%%%%
\subsection{Bayesovo posodabljanje}
Bayesovo posodabljanje je logična trditev, kako se sčasoma posodabljajo
apriorne oziroma predhodne verjetnosti dokazov glede na novo zbrane dokaze oziroma prepričanja.
\begin{definicija}
    (Bayesovo posodabljanje)
    Če se dogodek E zgodi ob času $t_1 > t_0$, potem je $P_1(H) = P_0(H \lvert E)$.
\end{definicija}
Ob času $t_0$ dogodku H dodelimo verjetnost $P_0(H)$; to se imenuje predhodna verjetnost oziroma apriorna verjetnost. Ko se zgodi dogodek E
ob času $t_1$, ki vpliva na naša prepričanja o dogodku H, Bayesovo posodabljanje pravi, da je potrebno apriorno verjetnost dogodka H v času $t_1$
enačiti s pogojno verjetnostjo dogodka H glede na dogodek E v času $t_0$. \\

%%%%%%%%%%%%%%%%%%%%%%%%%%%%%%%%%%%%%%%%%%%%%%%%%%%%%%%%%%%%%%%%%%%%%%%%%%%%%%%%%%%%%%%%%%%%%%%%%%%%%%%%%%%%%%%%%%%%%%%%%%%%%%%%%%%%%%%%%%%%%%
\subsection{Bayesova teorija v kazenskem pravu}
Gre za postopek posodabljanja verjetnosti tožilčeve hipoteze na podlagi predhodnih verjetnosti.\\

%%%%%%%%%%%%%%%%%%%%%%%%%%%%%%%%%%%%%%%%%%%%%%%%%%%%%%%%%%%%%%%%%%%%%%%%%%%%%%%%%%%%%%%%%%%%%%%%%%%%%%%%%%%%%%%%%%%%%%%%%%%%%%%%%%%%%%%%%%%%%%
\subsection{Predhodna verjetnost in določitev aposteriorne verjetnosti}
\begin{definicija}
    Predhodna verjetnost, ki je uporabljena v vsaki posodobitvi verjetnosti s pomočjo Bayesove teorije, je začetna verjetnost hipoteze 
    oziroma tožilčeve domneve o obtožencu oziroma storilcu kaznivega dejanja.
\end{definicija}
Določitev predhodnih oziroma apriornih verjetnosti je resen problem pri Bayesovemu pristopu v kazenskih postopkih. Različne metode za določitev 
in izračun teh verjetnosti lahko dajejo rezultate, ki se med seboj precej razlikujejo, kar pa je problematično, ker celotna Bayesova teorija 
temelji ravno na teh začetnih izračunih.\\\\
Bistveno vprašanje, ki se postavlja, je, ali naj analitiki sploh poskušajo določiti predhodne verjetnosti in če ja, kako naj jih določijo. Nekateri 
strokovnjaki predlagajo, da bi analitiki morali predpostaviti enake predhodne verjetnosti za vse hipoteze v primeru, kar se imenuje nevtralno stanje. 
To bi se lahko izkazalo za praktičen pristop, saj se lahko statistik s tem izogne vplivu lastnih in odvetniških predsodkov ter mnenj, ki bi lahko 
vplivali na predpostavke o verjetnosti. Na ta način se lahko zagotovi objektivnost analize, saj ne poskušamo prikazati ene hipoteze bolj verjetne 
od druge. Kljub temu pa mislim, da moramo biti do tega pristopa nekoliko kritični, saj je predpostavljanje enake verjetnosti za vse možnosti 
problematično - v realnosti se različne hipoteze razlikujejo po svoji verjetnosti.\\
Ker obstajajo pravila in smernice, kako upoštevati zakonodajo, pravila in postopke sodnega procesa, naj statistik uporabi svoje strokovno znanje za 
izračun apriorne verjetnosti na podlagi razpoložljivih podatkov in brez nepotrebnega vplivanja odvetnikov ali drugih udeležencev postopka. \\\\
Tekom sodnega procesa se pojavljajo nove domneve o obtožencu in novi dokazi s kraja zločina. Smiselno je, da vse to v postopku 
izračuna tudi upoštevamo. Zasledila sem, da se tekom sodnega procesa marsikateri dokaz najprej prizna in je znan sodniku, ki presoja tožilčevo domnevo 
o obtožencu, torej ga upoštevajo v svojih izračunih za posodobitve predhodnih verjetnosti hipotez. Potem pa dokaz iz sodnega procesa umaknejo, ampak 
presnetilo me je, da dokaz največkrat ni umaknjen iz verjetnostnih računov. Mnenja sem, da bi morali statistiki, ko se določen dokaz iz 
sodnega procesa, zaradi tehtnega razloga, umakne, posodobiti vse račune za nazaj in nato nadaljevati posodabljanje verjetnosti. Tako bi dobili primeren 
izračun posteriornih verjetnosti.

%%%%%%%%%%%%%%%%%%%%%%%%%%%%%%%%%%%%%%%%%%%%%%%%%%%%%%%%%%%%%%%%%%%%%%%%%%%%%%%%%%%%%%%%%%%%%%%%%%%%%%%%%%%%%%%%%%%%%%%%%%%%%%%%%%%%%%%%%%%%%%
%%%%%%%%%%%%%%%%%%%%%%%%%%%%%%%%%%%%%%%%%%%%%%%%%%%%%%%%%%%%%%%%%%%%%%%%%%%%%%%%%%%%%%%%%%%%%%%%%%%%%%%%%%%%%%%%%%%%%%%%%%%%%%%%%%%%%%%%%%%%%%
\section{Razmerje verjetij}
Občasno se zgodi, da predloga tožilstva in obrambe nista komplementarna in v takih primerih ni mogoče določiti $P(H_p)$ ali $P(H_d)$, 
ampak samo vpliv statistike, znane kot razmerje verjetij.

%%%%%%%%%%%%%%%%%%%%%%%%%%%%%%%%%%%%%%%%%%%%%%%%%%%%%%%%%%%%%%%%%%%%%%%%%%%%%%%%%%%%%%%%%%%%%%%%%%%%%%%%%%%%%%%%%%%%%%%%%%%%%%%%%%%%%%%%%%%%%%
\subsection{Opredelitev}
Bayesov izrek v obliki razmerja verjetij
\begin{equation}
   \frac{P(H \lvert E)}{P(\bar{H} \lvert E)} = \frac{P(E \lvert H)}{P(E \lvert \bar{H})} \times \frac{P(H)}{P(\bar{H})}. \vspace{2mm}
\end{equation}
\begin{definicija}
    Razmerje
    \begin{equation}
        \frac{P(E \lvert H)}{P(E \lvert \bar{H})} \vspace{2mm}
    \end{equation}
     se imenuje razmerje verjetij. \\
\end{definicija}
Oglejmo si dogodka $E$ in $H$, ter njuna komplementa. Razmerje verjetij je tu razmerje verjetij $E$, ko je $H$ resničen in verjetnosti $E$,
ko je $H$ neresničen. Da bi upoštevali učinek $E$ na verjetnost $H$, tj. da bi
\[
   \frac{P(H)}{P(\bar{H})} \vspace{2mm}
\]
spremenili v
\[
   \frac{P(H \lvert E)}{P(\bar{H} \lvert E)}, \vspace{2mm}
\]
prvo pomnožimo z razmerjem verjetij. Razmerje
\[
   \frac{P(H)}{P(\bar{H})} \vspace{2mm}
\]
je znano kot apriorno razmerje v korist H, razmerje
\[
   \frac{P(H \lvert E)}{P(\bar{H} \lvert E)} \vspace{2mm}
\]
pa je znano kot aposteriorno razmerje v korist $H$.\\\\
Razlika med $P(E \lvert H)$ in $P(H \lvert E)$ je bistvena. Pri proučevanju vpliva
$E$ na $H$ je treba upoštevati tako verjetnost $E$, ko je $H$ resničen in ko je $H$ neresničen. Pogosta napaka, tj. zmota prenesene pogojne
verjetnosti, je, da dogodek $E$, ki je malo verjeten, če je $\bar{H}$ resničen, pomeni dokaz v prid $H$. Da bi bilo tako, je treba dodatno
zagotoviti, da E ni tako malo verjeten, če je H resničen. Razmerje verjetij je potem večje od 1 in pozitivna verjetnost je večja od
predhodne verjetnosti. Torej iz Bayesovega izreka neposredno izhaja, da če je razmerje verjetij večje od 1, potem dokaz povečuje
»verjetnost« krivde (pri čemer višje vrednosti pomenijo večjo verjetnost krivde), če pa je manjše od 1, zmanjšuje »verjetnost« krivde
(in bolj ko se približuje ničli, manjša je »verjetnost« krivde).

%%%%%%%%%%%%%%%%%%%%%%%%%%%%%%%%%%%%%%%%%%%%%%%%%%%%%%%%%%%%%%%%%%%%%%%%%%%%%%%%%%%%%%%%%%%%%%%%%%%%%%%%%%%%%%%%%%%%%%%%%%%%%%%%%%%%%%%%%%%%%%
\subsection{Razmerje verjetij v kazenskem pravu}
Sedaj si poglejmo obliko Bayesovega izreka o razmerju verjetij v forenzičnem kontekstu, tj. ocenjevanje vrednosti nekaterih dokazov.\\\\
Naj bo:\\
$H_p \dots$ obtoženec je resnično kriv - nadomestimo $H$;\\
$H_d \dots$ obtoženec je resnično nedolžen - nadomestimo $\bar{H}$;\\
$Ev \dots$ obravnavani dokaz - nadomestimo dogodek $E$;\\\\
Ker ima vsak kazenski primer specifične posebnosti, moramo v enačbo vpeljati še t.i. informacije o ozadju, $I$. Če jih ne upoštevamo, lahko posamezno 
vrednotenje dokazov postane preveč splošno. Ob upoštevanju informacij o ozadju $I$ dobimo zapis
\[
   \frac{P(H_p \lvert Ev, I)}{P(H_d \lvert Ev, I)} = \frac{P(Ev \lvert H_p, I)}{P(Ev \lvert H_d, I)} \times \frac{P(H_p \lvert I)}{P(H_d \lvert I)}. \vspace{2mm}
\]
Da lahko ocenimo oziroma določimo vrednost dokaza, potrebujemo razmerje verjetij.
\begin{definicija}
    Naj bosta  $H_p$ in $H_d$ dve konkurenčni hipotezi ter $I$ informacije o ozadju. Vrednost $V$ dokaza $Ev$ je podana z
    \[
        V = \frac{P(Ev \lvert H_p, I)}{P(Ev \lvert H_d, I)}, \vspace{2mm}
    \]
    to je razmerje verjetij, ki pretvori predhodno razmerje
    \[
        \frac{P(H_p \lvert I)}{P(H_d \lvert I)} \vspace{2mm}
    \]
    v aposteriorno razmerje
    \[
        \frac{P(H_p \lvert Ev, I)}{P(H_d \lvert Ev, I)}.
    \]
\end{definicija}
Razmerje verjetij nam pove, katera hipoteza je bolje podprta z dokazi.\\\\
Ker je ocena vrednosti razmerja verjetij lahko podvržena številnim virom negotovosti, mora poročilo o vrednosti razmerja verjetij vključevati merilo 
njegove natančnosti, na primer  na primer z navedbo številčnega razpona vrednosti za verjetnost dokazov na podlagi konkurenčnih predlogov in s tem 
številčnega razpona vrednosti za razmerje verjetij.

%%%%%%%%%%%%%%%%%%%%%%%%%%%%%%%%%%%%%%%%%%%%%%%%%%%%%%%%%%%%%%%%%%%%%%%%%%%%%%%%%%%%%%%%%%%%%%%%%%%%%%%%%%%%%%%%%%%%%%%%%%%%%%%%%%%%%%%%%%%%%%
%%%%%%%%%%%%%%%%%%%%%%%%%%%%%%%%%%%%%%%%%%%%%%%%%%%%%%%%%%%%%%%%%%%%%%%%%%%%%%%%%%%%%%%%%%%%%%%%%%%%%%%%%%%%%%%%%%%%%%%%%%%%%%%%%%%%%%%%%%%%%%
\section{Zmote v kazenskem pravu}
Ker večina ljudi pri razmišljanju o verjetnosti dela osnovne napake, obstaja mnogo zmot, ki izhajajo iz osnovnega razumevanja pravil
teorije verjetnosti. Številne od teh zmot so zlasti posledica napačnega razumevanja pogojne verjetnosti. Bolj znana primera takih zmot sta
tožilčeva zmota in zmota obrambnega odvetnika.

%%%%%%%%%%%%%%%%%%%%%%%%%%%%%%%%%%%%%%%%%%%%%%%%%%%%%%%%%%%%%%%%%%%%%%%%%%%%%%%%%%%%%%%%%%%%%%%%%%%%%%%%%%%%%%%%%%%%%%%%%%%%%%%%%%%%%%%%%%%%%%
\subsection{Tožilčeva zmota}
Tožilčeva zmota je dobro znana statistična zmota, ki izhaja iz napačnega razumevanja
pogojnih verjetnosti, pogosto imenuje napaka prenesenega pogojnika oziroma tudi 
tožilčeva zmota.\\\\
Če je $E$ dokaz in $H$ trditev,  da je obtoženi nedolžen, upoštevamo pogojne verjetnosti: \\
$P(E \lvert H)$ \dots verjetnost resničnosti dokaz $E$, kljub temu da je obtoženi nedolžen; \\
$P(H \lvert E)$ \dots verjetnost, da je obtoženi nedolžen kljub dokazu $E$. \\\\
Tožilčeva zmota je zamenjava verjetnost dokaza $E$ glede na
hipotezo $H$ z verjetnostjo hipoteze $H$ glede na dokaze $E$ oziroma $P(E \lvert H)$ z $P(H \lvert E)$, torej ko napačno verjamemo, da je verjetnost 
naključnega znanstvenega ujemanja enaka verjetnosti, da je obtoženec nedolžen.\\\\
Pri forenzičnih dokazih je ponavadi verjetnost $P(E \lvert H)$ majhna. Tožilec pa velikokrat sklepa, da je tudi verjetnost
$P(H \lvert E)$ majhna.
Zgoraj napisani pogojni verjetnosti pa sta lahko precej različni; uporabimo Bayesovo pravilo:
\[
   P(H \lvert E) = P(E \lvert H) \times \frac{P(H)}{P(E)}, \vspace{2mm}
\]
kjer je $P(H)$ verjetnost nedolžnosti in $P(E)$ verjetnost dokaza. Enačba kaže, da majhna pogojna verjetnost $P(E \lvert H)$ ne pomeni majhne
pogojne verjetnosti $P(H \lvert E)$ v primeru velike verjetnosti nedolžnosti in majhne verjetnosti dokaza. \\

%%%%%%%%%%%%%%%%%%%%%%%%%%%%%%%%%%%%%%%%%%%%%%%%%%%%%%%%%%%%%%%%%%%%%%%%%%%%%%%%%%%%%%%%%%%%%%%%%%%%%%%%%%%%%%%%%%%%%%%%%%%%%%%%%%%%%%%%%%%%%%
\subsection{Zmota obrambnega odvetnika}
Se zgodi, ko se dokazi obtoženca ujemajo z dokazi kaznivega dejanja štejejo za nepomembne, ker visoka predhodna verjetnost, da obotženec ni storil kaznivega dejanja (kar
se zgodi, če je na primer potencialno število osumljencev zelo veliko) še vedno povzroči visoko verjetnost, da oseba ustreza dokazom kaznivega dejanja, kljub temu, da ni 
vpletena v kaznivo dejanje. \\

%%%%%%%%%%%%%%%%%%%%%%%%%%%%%%%%%%%%%%%%%%%%%%%%%%%%%%%%%%%%%%%%%%%%%%%%%%%%%%%%%%%%%%%%%%%%%%%%%%%%%%%%%%%%%%%%%%%%%%%%%%%%%%%%%%%%%%%%%%%%%%
%%%%%%%%%%%%%%%%%%%%%%%%%%%%%%%%%%%%%%%%%%%%%%%%%%%%%%%%%%%%%%%%%%%%%%%%%%%%%%%%%%%%%%%%%%%%%%%%%%%%%%%%%%%%%%%%%%%%%%%%%%%%%%%%%%%%%%%%%%%%%%
\section{Izogib zmotam z uporabo Bayesovih omrežij}
Ker je največkrat težava v tem, da se večina odvetnikov in sodnikov ob pogledu na verjetnostne izračune in statistično analizo dokazov ustraši, se mi 
zdijo Bayeosova omrežja dober predlog za predstavitev verjetnostnih izračunov.

%%%%%%%%%%%%%%%%%%%%%%%%%%%%%%%%%%%%%%%%%%%%%%%%%%%%%%%%%%%%%%%%%%%%%%%%%%%%%%%%%%%%%%%%%%%%%%%%%%%%%%%%%%%%%%%%%%%%%%%%%%%%%%%%%%%%%%%%%%%%%%
\subsection{Opredelitev}
Bayesova omrežja pomagajo določiti ustrezne verjetnostne formule, ne da bi prikazali njihovo polno algebrsko obliko, in omogočajo
skoraj popolno avtomatizacijo potrebnih verjetnostnih izračunov.
\begin{definicija}
   Bayesovo omrežje je verjetnostni grafični model, ki predstavlja množico spremenljivk in njihovih pogojnih odvisnosti prek usmerjenega
   acikličnega grafa.
\end{definicija}

%%%%%%%%%%%%%%%%%%%%%%%%%%%%%%%%%%%%%%%%%%%%%%%%%%%%%%%%%%%%%%%%%%%%%%%%%%%%%%%%%%%%%%%%%%%%%%%%%%%%%%%%%%%%%%%%%%%%%%%%%%%%%%%%%%%%%%%%%%%%%%
\subsection{Uporaba Bayesovih omrežij na sodišču}
Tem metodam se daje poseben poudarek, kadar je treba med konkurenčnimi hipotezami izbrati najverjetnejšo, izbira pa mora biti podprta z znanstveno utemeljeno
argumentacijo. Primerna so za analizo dogodka, ki se je zgodil, in napovedovanje verjetnosti, da je k temu prispeval katerikoli od več možnih
znanih vzrokov. Prednosti Bayesovih mrež se najbolj izrazito pokažejo na zapletenih področjih z več spremenljivkami.\\\\
Pomembno je, da za izdelavo uporabimo dosleden okvir, drugače lahko Bayesova omrežja, ki jih oblikujejo različne stranke za en
primer, kažejo različne rezultate.

%%%%%%%%%%%%%%%%%%%%%%%%%%%%%%%%%%%%%%%%%%%%%%%%%%%%%%%%%%%%%%%%%%%%%%%%%%%%%%%%%%%%%%%%%%%%%%%%%%%%%%%%%%%%%%%%%%%%%%%%%%%%%%%%%%%%%%%%%%%%%%
%%%%%%%%%%%%%%%%%%%%%%%%%%%%%%%%%%%%%%%%%%%%%%%%%%%%%%%%%%%%%%%%%%%%%%%%%%%%%%%%%%%%%%%%%%%%%%%%%%%%%%%%%%%%%%%%%%%%%%%%%%%%%%%%%%%%%%%%%%%%%%
\section{Zaključek}
V diplomski nalogi ugotovim, na podlagi pregleda nekaterih drugih metod, da je Bayesov pristop na splošno najboljši za vrednotenje dokazov, še posebej v primeru,
ko se dokazi nanašajo na DNK ali laboratorijske analize. V delu pridem do ugotovitve, da je mogoče najbolje, da za določitev in izračun prehodnih verjetnosti poskrbijo
statistiki, saj znajo v svoje modele vključevati tudi pravno teorijo. Težave pa nastanejo pri vključevanju novih dokazov, ki so sodišču priloženi
tekom sodnega procesa in izključevanju nepomembnih oziroma nepotrebnih dokazov za sodni proces.\\
Ker so me predvsem zanimale zmote, do katerih prihaja, sem veliko časa posvetila ugotavljanju najboljšega načina za izogib zmotam. In glavna ugotovitev dela je, da se lahko 
s pomočjo Bayesovih omrežij izognemo zmotam, saj ne prikažejo polne algebrske oblike verjetnostne formule, omogočajo pa skoraj popolno
avtomatizacijo potrebnih verjetnostnih izračunov in več prefinjenih možnosti.




\end{document}