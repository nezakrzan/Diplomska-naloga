\documentclass[12pt,a4paper]{amsart}
\usepackage[slovene]{babel}
%\usepackage[cp1250]{inputenc}
\usepackage[T1]{fontenc}
\usepackage[utf8]{inputenc}
\usepackage{amsmath,amssymb,amsfonts}
\usepackage{url}
\usepackage[normalem]{ulem}
\usepackage[dvipsnames,usenames]{color}
\usepackage{graphicx}

% Oblika strani
\textwidth 15cm
\textheight 24cm
\oddsidemargin.5cm
\evensidemargin.5cm
\topmargin-5mm
\addtolength{\footskip}{10pt}
\pagestyle{plain}
\overfullrule=15pt % oznaci predlogo vrstico

% Ukazi za matematična okolja
\theoremstyle{definition} % tekst napisan pokončno
\newtheorem{definicija}{Definicija}[section]
\newtheorem{primer}[definicija]{Primer}
\newtheorem{opomba}[definicija]{Opomba}

\renewcommand\endprimer{\hfill$\diamondsuit$}


\theoremstyle{plain} % tekst napisan poševno
\newtheorem{lema}[definicija]{Lema}
\newtheorem{izrek}[definicija]{Izrek}
\newtheorem{trditev}[definicija]{Trditev}
\newtheorem{posledica}[definicija]{Posledica}

\begin{document}

%%%%%%%%%%%%%%%%%%%%%%%%%%%%%%%%%%%%%%%%%%%%%%%%%%%%%%%%%%%%%%%%%%%%%%%%%%%%%%%%%%%%%%%%%%%%%%%%%%%%%%%%%%%%%%%%%%%%%%%%%%%%%%%%%%%%%%%%%%%%%%
\title{Statistika v kazenskem pravu}
\author{Neža Kržan}
\maketitle

%%%%%%%%%%%%%%%%%%%%%%%%%%%%%%%%%%%%%%%%%%%%%%%%%%%%%%%%%%%%%%%%%%%%%%%%%%%%%%%%%%%%%%%%%%%%%%%%%%%%%%%%%%%%%%%%%%%%%%%%%%%%%%%%%%%%%%%%%%%%%%
%%%%%%%%%%%%%%%%%%%%%%%%%%%%%%%%%%%%%%%%%%%%%%%%%%%%%%%%%%%%%%%%%%%%%%%%%%%%%%%%%%%%%%%%%%%%%%%%%%%%%%%%%%%%%%%%%%%%%%%%%%%%%%%%%%%%%%%%%%%%%%
\section{Statistika v kazenskem pravu}
Raziskave na področju kazenskega pravosodja in kriminologije so različne po naravi in namenu. Velik del raziskav vključuje preverjanje teorije 
in hipotez. Statistiki si na področju kazenskega pravosodja in kriminologije običajno prizadevajo preučiti razmerja med dvema ali več spremenljivkami.
\begin{definicija}
    \textit{Odvisna spremenljivka} je pojav, ki ga želi statistik preučiti, razložiti ali napovedati.
\end{definicija}
\begin{definicija}
    \textit{Neodvisna spremenljivka} je dejavnik ali značilnost, s katero se poskuša pojasniti ali napovedati odvisno spremenljivko.
\end{definicija}
Pomembno je razumeti, da neodvisno in odvisno nista sinonima za vzrok in posledico. Določene neodvisne spremenljivke so lahko povezane z
določenimi odvisnimi spremenljivkami, vendar to še zdaleč ni dokončen dokaz, da so prve vzrok drugih. Za dokazovanje vzročnosti morajo
statistiki dokazati, da njihove študije izpolnjujejo tri merila. Prvo je časovno zaporedje, kar pomeni, da se mora neodvisna spremenljivka
pojaviti pred odvisno spremenljivko. Druga zahteva glede vzročnosti je, da obstaja empirična povezava med neodvisno in odvisno spremenljivko.
Zadnja zahteva je, da je razmerje med neodvisno spremenljivko in odvisno spremenljivko nepristransko. Nepristranskost je v kriminologiji in
kazenskopravnih raziskavah pogosto najtežje dokazati, saj je človeško vedenje zapleteno in ima vsako dejanje, ki ga oseba stori, več vzrokov.
Razmejitev teh vzročnih dejavnikov je lahko težavna ali nemogoča. Razlog, zakaj je nepristranskost problem, je, da lahko obstaja tretja
spremenljivka, ki pojasnjuje odvisno spremenljivko enako dobro ali celo bolje kot neodvisne spremenljivke. Ta tretja spremenljivka lahko delno
ali v celoti pojasni razmerje med neodvisno in odvisno spremenljivko. Nenamerna izključitev ene ali več pomembnih spremenljivk lahko privede do
napačnih zaključkov, saj lahko statistik zmotno verjame, da neodvisna spremenljivka močno napoveduje odvisno spremenljivko, medtem ko je v
resnici razmerje dejansko delno ali v celoti posledica posrednih dejavnikov. Drug izraz za to težavo je pristranskost izpuščenih spremenljivk.
Kadar je pristranskost izpuščenih spremenljivk prisotna v razmerju neodvisne spremenljivke - odvisne spremenljivke, vendar se ne prepozna, lahko
pridemo do napačnega sklepa o sodnem procesu.\\\\
Statistiki se že na začetku sodnega procesa soočajo s prvimi težavami - določitvijo odvisnih in neodvisnih spremenljivk za
modeliranje. V proces določanja spremenljivk pa pogosto v preveliki meri posegajo odvetniki, ki se sklicujejo na pravne zakone in načela. To lahko
postane sporno, saj lahko takšni pretirani posegi ovirajo statistične znanstvenike pri izračunu verjetnostnega vpliva spremenljivk. Zagotovo 
določitev spremenljivk ne sme biti naloga le odvetnikov ali le statistikov, ampak menim, da je sodelovanje med statistiki in odvetniki pomembno, 
saj vsak prispeva svoj del poznavanja teorije, ki je v ozadju.  S tem se lahko zagotovi pravilno opredelitev spremenljivk in pravilne verjetnostne 
izračune, ki bodo prispevali k pravičnim odločitvam v sodnih postopkih.

%%%%%%%%%%%%%%%%%%%%%%%%%%%%%%%%%%%%%%%%%%%%%%%%%%%%%%%%%%%%%%%%%%%%%%%%%%%%%%%%%%%%%%%%%%%%%%%%%%%%%%%%%%%%%%%%%%%%%%%%%%%%%%%%%%%%%%%%%%%%%%
%%%%%%%%%%%%%%%%%%%%%%%%%%%%%%%%%%%%%%%%%%%%%%%%%%%%%%%%%%%%%%%%%%%%%%%%%%%%%%%%%%%%%%%%%%%%%%%%%%%%%%%%%%%%%%%%%%%%%%%%%%%%%%%%%%%%%%%%%%%%%%
\section{Raziskovalni proces}
Raziskovalni proces se izvaja po naslednjih točkah.\\
\textbf{1. Identifikacija problema.\\} 
\textbf{2. Zasnova raziskave.\\} 
\textbf{3. Analiza podatkov.\\}

%%%%%%%%%%%%%%%%%%%%%%%%%%%%%%%%%%%%%%%%%%%%%%%%%%%%%%%%%%%%%%%%%%%%%%%%%%%%%%%%%%%%%%%%%%%%%%%%%%%%%%%%%%%%%%%%%%%%%%%%%%%%%%%%%%%%%%%%%%%%%%
%%%%%%%%%%%%%%%%%%%%%%%%%%%%%%%%%%%%%%%%%%%%%%%%%%%%%%%%%%%%%%%%%%%%%%%%%%%%%%%%%%%%%%%%%%%%%%%%%%%%%%%%%%%%%%%%%%%%%%%%%%%%%%%%%%%%%%%%%%%%%%
\section{Vrednotenje dokazov}
Najpogostejši uporabljeni metodi za ocenjevanje dokazov sta metoda dokazne vrednosti in model verjetnosti hipoteze.\\
Metoda dokazne vrednosti temelji na vrednosti, ki jo ima dokaz za dokazno temo, njen namen pa je ugotoviti, ali med dokazom in zadevno dokazno temo 
obstaja naključna povezava, pri čemer je dokazna tema glavna obtožba v kazenskem primeru. S to metodo dokazujemo določen omejen nabor dokazov, njen cilj pa 
je oceniti verjetnost, da dokazi dokazujejo hipotezo.\\
Z modelom verjetnosti hipoteze pa ocenjujemo verjetnost hipoteze glede na dokaze. Cilj je ugotoviti, kako verjetno je, da je hipoteza, za katero dokazi 
zagotavljajo določeno stopnjo podpore, resnična. Glavna razlika z metodo dokazne vrednosti je, da predpostavlja, da obstaja začetna verjetnost za hipotezo 
pred obravnavo dokazov, t.i. predhodna verjetnost.\\\\

%%%%%%%%%%%%%%%%%%%%%%%%%%%%%%%%%%%%%%%%%%%%%%%%%%%%%%%%%%%%%%%%%%%%%%%%%%%%%%%%%%%%%%%%%%%%%%%%%%%%%%%%%%%%%%%%%%%%%%%%%%%%%%%%%%%%%%%%%%%%%%
%%%%%%%%%%%%%%%%%%%%%%%%%%%%%%%%%%%%%%%%%%%%%%%%%%%%%%%%%%%%%%%%%%%%%%%%%%%%%%%%%%%%%%%%%%%%%%%%%%%%%%%%%%%%%%%%%%%%%%%%%%%%%%%%%%%%%%%%%%%%%%
\section{Koncept verjetnosti}
Pogosto se opravlja primerjava verjetnosti dokazov na podlagi dveh konkurenčnih predlogov, in sicer predloga tožilca in predloga obrambe.\\\\
$H_p \dots$ trditev, ki jo predlaga tožilstvo;\\
$H_d \dots$ trditev, ki jo predlaga obramba;\\\\
V splošnem nas zanima vpliv dokazov na verjetnost krivde($H_p$) in nedolžnosti($H_d$) osumljenca. Gre za dopolnjujoča se dogodka in razmerje verjetij 
teh dveh dogodkov,
\begin{equation}
   \frac{P(H_p)}{P(H_d)}, \vspace{2mm}
\end{equation}
je verjetnost proti nedolžnosti ali verjetnost za krivdo. Ob upoštevanju dodatnih informacij $E$ oziroma dokazov, je razmerje
\begin{equation}
   \frac{P(H_p \lvert E)}{P(H_d \lvert E)} \vspace{2mm},
\end{equation}
verjetnost v prid krivdi ob upoštevanju dokazov $E$.\\\\
Ali je obtoženec kriv glede na znan doka $E$, je glavna stvar, ki nas pri sojenju zanima. Če imamo torej na voljo dokaz $E$, nas zanima pogojna 
verjetnost
\[
    P(kriv \lvert E), \vspace{2mm}
\]
pri čemer nam je lahko v pomoč Bayesovo pravilo. To v teoriji drži, čeprav je v praksi izračun verjetnostne krivde lahko preveč zapleten. Ampak 
z Bayesovim pravilom lahko ocenimo verjetnosti vmesnih trditev oziroma dokazov, ki so ključnega pomena za ugotavljanje obtoženčeve krivde.

%%%%%%%%%%%%%%%%%%%%%%%%%%%%%%%%%%%%%%%%%%%%%%%%%%%%%%%%%%%%%%%%%%%%%%%%%%%%%%%%%%%%%%%%%%%%%%%%%%%%%%%%%%%%%%%%%%%%%%%%%%%%%%%%%%%%%%%%%%%%%%
%%%%%%%%%%%%%%%%%%%%%%%%%%%%%%%%%%%%%%%%%%%%%%%%%%%%%%%%%%%%%%%%%%%%%%%%%%%%%%%%%%%%%%%%%%%%%%%%%%%%%%%%%%%%%%%%%%%%%%%%%%%%%%%%%%%%%%%%%%%%%%
\section{Bayesova statistika}

%%%%%%%%%%%%%%%%%%%%%%%%%%%%%%%%%%%%%%%%%%%%%%%%%%%%%%%%%%%%%%%%%%%%%%%%%%%%%%%%%%%%%%%%%%%%%%%%%%%%%%%%%%%%%%%%%%%%%%%%%%%%%%%%%%%%%%%%%%%%%%
\subsection{Opredelitev}
Bayesova analiza je standardna metoda za posodabljanje verjetnosti po opazovanju več dokazov, zato je zelo primerna za obravnavo in vrednotenje 
dokazov.
Začnemo z nekim predhodnim prepričanjem o hipotezi in ga posodabljamo, ko se dokazi ponovno pojavijo. Pri uporabi Bayesovega sklepanja morajo statistiki utemeljiti predhodne predpostavke, kadar je to mogoče, na primer z
uporabo zunanjih podatkov; v nasprotnem primeru morajo uporabiti razpon vrednosti predpostavk in analizo občutljivosti, da preverijo zanesljivost rezultata
glede na te vrednosti.

%%%%%%%%%%%%%%%%%%%%%%%%%%%%%%%%%%%%%%%%%%%%%%%%%%%%%%%%%%%%%%%%%%%%%%%%%%%%%%%%%%%%%%%%%%%%%%%%%%%%%%%%%%%%%%%%%%%%%%%%%%%%%%%%%%%%%%%%%%%%%%
\subsection{Bayesovo pravilo}
Bayesovo sklepanje temelji na Bayesovem pravilu, ki izraža verjetnost nekega dogodka z verjetnostjo dveh dogodkov in obrnjene pogojne
verjetnosti.
\begin{izrek}
    (Bayesovo pravilo)
    \begin{equation}\label{eq:bpravilo}
        P(H \lvert E) = \frac{P(E \lvert H) \times P(H)}{P(E)}. \vspace{2mm}
     \end{equation}
\end{izrek}

%%%%%%%%%%%%%%%%%%%%%%%%%%%%%%%%%%%%%%%%%%%%%%%%%%%%%%%%%%%%%%%%%%%%%%%%%%%%%%%%%%%%%%%%%%%%%%%%%%%%%%%%%%%%%%%%%%%%%%%%%%%%%%%%%%%%%%%%%%%%%%
\subsection{Bayesovo posodabljanje}
Bayesovo pravilo se razlikuje od Bayesovega posodabljanja. Prvo je matematični izrek, drugo pa logična trditev, kako se sčasoma posodabljajo
apriorne oziroma predhodne verjetnosti dokazov glede na novo zbrane dokaze oziroma prepričanja.
\begin{definicija}
    (Bayesovo posodabljanje)
    Če se dogodek E zgodi ob času $t_1 > t_0$, potem je $P_1(H) = P_0(H \lvert E)$.
\end{definicija}
Ob času $t_0$ dogodku H dodelimo verjetnost $P_0(H)$; to se imenuje predhodna verjetnost oziroma apriorna verjetnost. Ko se zgodi dogodek E
ob času $t_1$, ki vpliva na naša prepričanja o dogodku H, Bayesovo posodabljanje pravi, da je potrebno apriorno verjetnost dogodka H v času $t_1$
enačiti s pogojno verjetnostjo dogodka H glede na dogodek E v času $t_0$. \\

%%%%%%%%%%%%%%%%%%%%%%%%%%%%%%%%%%%%%%%%%%%%%%%%%%%%%%%%%%%%%%%%%%%%%%%%%%%%%%%%%%%%%%%%%%%%%%%%%%%%%%%%%%%%%%%%%%%%%%%%%%%%%%%%%%%%%%%%%%%%%%
\subsection{Bayesova teorija v kazenskem pravu}
Gre za postopek posodabljanja verjetnosti tožilčeve hipoteze na podlagi predhodnih oziroma apriornih verjetnosti. Pretvorimo predhodno verjetnost, 
tj. verjetnost hipoteze pred upoštevanjem določenega dokaza (dokazov), v aposteriorno verjetnost, tj. verjetnost hipoteze po upoštevanju določenega 
dokaza (dokazov).\\

%%%%%%%%%%%%%%%%%%%%%%%%%%%%%%%%%%%%%%%%%%%%%%%%%%%%%%%%%%%%%%%%%%%%%%%%%%%%%%%%%%%%%%%%%%%%%%%%%%%%%%%%%%%%%%%%%%%%%%%%%%%%%%%%%%%%%%%%%%%%%%
\subsection{Predhodna verjetnost in določitev aposteriorne verjetnosti}
\begin{definicija}
    Predhodna verjetnost, ki je uporabljena v vsaki posodobitvi verjetnosti s pomočjo Bayesove teorije, je začetna verjetnost hipoteze 
    oziroma tožilčeve domneve o obtožencu oziroma storilcu kaznivega dejanja.
\end{definicija}
Po končni posodobitvi dobimo verjetnost hipoteze glede na vse dokaze, predložene na sojenju.\\\\
Določitev predhodnih oziroma apriornih verjetnosti je resen problem pri Bayesovemu pristopu v kazenskih postopkih. Različne metode za določitev 
in izračun teh verjetnosti lahko dajejo rezultate, ki se med seboj precej razlikujejo, kar pa je problematično, ker celotna Bayesova teorija 
temelji ravno na teh začetnih izračunih.\\\\
Bistveno vprašanje, ki se postavlja, je, ali naj analitiki sploh poskušajo določiti predhodne verjetnosti in če ja, kako naj jih določijo. Nekateri 
strokovnjaki predlagajo, da bi analitiki morali predpostaviti enake predhodne verjetnosti za vse hipoteze v primeru, kar se imenuje nevtralno stanje. 
To bi se lahko izkazalo za praktičen pristop, saj se lahko statistik s tem izogne vplivu lastnih in odvetniških predsodkov ter mnenj, ki bi lahko 
vplivali na predpostavke o verjetnosti. Na ta način se lahko zagotovi objektivnost analize, saj ne poskušamo prikazati ene hipoteze bolj verjetne 
od druge. Kljub temu pa mislim, da moramo biti do tega pristopa nekoliko kritični, saj je predpostavljanje enake verjetnosti za vse možnosti 
problematično - v realnosti se različne hipoteze razlikujejo po svoji verjetnosti.\\
Ker obstajajo pravila in smernice, kako upoštevati zakonodajo, pravila in postopke sodnega procesa, naj statistik uporabi svoje strokovno znanje za 
izračun apriorne verjetnosti na podlagi razpoložljivih podatkov in brez nepotrebnega vplivanja odvetnikov ali drugih udeležencev postopka. \\\\
Tekom sodnega procesa se pojavljajo nove domneve o obtožencu in novi dokazi s kraja zločina. Smiselno je, da vse to v postopku 
izračuna tudi upoštevamo. Zasledila sem, da se tekom sodnega procesa marsikateri dokaz najprej prizna in je znan sodniku, ki presoja tožilčevo domnevo 
o obtožencu, torej ga upoštevajo v svojih izračunih za posodobitve predhodnih verjetnosti hipotez. Potem pa dokaz iz sodnega procesa umaknejo, ampak 
presnetilo me je, da dokaz največkrat ni umaknjen iz verjetnostnih računov. Mnenja sem, da bi morali statistiki, ko se določen dokaz iz 
sodnega procesa, zaradi tehtnega razloga, umakne, posodobiti vse račune za nazaj in nato nadaljevati posodabljanje verjetnosti. Tako bi dobili primeren 
izračun posteriornih verjetnosti, na katerih bi potem lahko temeljil zaključek sodnega procesa.

%%%%%%%%%%%%%%%%%%%%%%%%%%%%%%%%%%%%%%%%%%%%%%%%%%%%%%%%%%%%%%%%%%%%%%%%%%%%%%%%%%%%%%%%%%%%%%%%%%%%%%%%%%%%%%%%%%%%%%%%%%%%%%%%%%%%%%%%%%%%%%
%%%%%%%%%%%%%%%%%%%%%%%%%%%%%%%%%%%%%%%%%%%%%%%%%%%%%%%%%%%%%%%%%%%%%%%%%%%%%%%%%%%%%%%%%%%%%%%%%%%%%%%%%%%%%%%%%%%%%%%%%%%%%%%%%%%%%%%%%%%%%%
\section{Bayesova analiza}
Najbolj pogosta uporaba Bayesovega pravila je pri ugotavljanju, ali je obtoženec vir sledi DNK-ja s kraja zločina. V ta namen sem opisala 
poenostavljeno in izpopolnjeno Bayesovo analizo, v primeru, ko je dokaz DNK sled.

%%%%%%%%%%%%%%%%%%%%%%%%%%%%%%%%%%%%%%%%%%%%%%%%%%%%%%%%%%%%%%%%%%%%%%%%%%%%%%%%%%%%%%%%%%%%%%%%%%%%%%%%%%%%%%%%%%%%%%%%%%%%%%%%%%%%%%%%%%%%
\subsection{Poenostavljena Bayesova analiza}
Naj bo:\\
$S$ \dots dogodek, da je obtoženec vir sledi DNK s kraja zločina; \\
$M$ \dots dogodek, da se obtoženčev DNK ujema z DNK-jem s kraja zločina; \\
$f$ \dots funkcija pogostosti ujemanja DNK z DNK-jem s kraja zločina. \\
Želimo vedeti, kakšna je verjetnost S glede na M, tj. $P(S \lvert M)$. \\\\
Bayesovo pravilo lahko uporabimo na naslednji način:
\[
   \frac{P(S \lvert M)}{P(\neg S \lvert M)} = \frac{P(M \lvert S)}{P(M \lvert \neg S)} \times \frac{P(S)}{P(\neg S)}. \vspace{2mm}
\]
Največja težava je pri določitvi vrednosti $P(M \lvert S)$, ki je običajno enaka ena - če bi obtoženec dejansko pustil sledove, bi laboratorijske 
analize pokazale ujemanje, kar imenujemo lažni negativni rezultat; to je sicer poenostavitev, saj se lahko zgodi, da analize ne pokažejo ujemanja, 
čeprav je obtoženec pustil sledi. \\
Potrebujemo še verjetnost $P(M \lvert \neg S)$, tj. verjetnost, da se bo našlo ujemanje, če obtoženec ni vir sledi na kraju zločina. To je
običajno enakovredno pogostosti ujemanja DNK-ja z DNK-jem s kraja zločina, tj. $f$; tudi to je poenostavitev, saj se lahko zgodi, da
obtoženec nima enakega DNK profila, vendar so laboratorijske analize pokazale, da ga ima, kar imenujemo lažni pozitivni rezultat.\\\\
Ker poenostavljena Bayesova analiza ne upošteva možnosti lažnih pozitivnih in negativnih rezultatov laboratorijske analize, sem si pogledala še 
izpopolnjeno Bayesovo analizo.

%%%%%%%%%%%%%%%%%%%%%%%%%%%%%%%%%%%%%%%%%%%%%%%%%%%%%%%%%%%%%%%%%%%%%%%%%%%%%%%%%%%%%%%%%%%%%%%%%%%%%%%%%%%%%%%%%%%%%%%%%%%%%%%%%%%%%%%%%%%%
\subsection{Izpopolnjena Bayesova analiza}
Da upoštevam možnosti laboratorijskih napak, namesto $M$ uvedem spremenljivko $M_p$.\\
$M_p$ \dots poročano ujemanje laboratorijske analize; \\
$M_t$ \dots trditev, da obstaja dejansko ujemanje v DNK-ju;\\
$\neg M_t$ \dots trditev, da obstaja neujemanje v DNK-ju.\\\\
Sledi:
\[
   P(M_p \lvert \neg S) = P(M_p \lvert M_t)P(M_t \lvert \neg S) + P(M_p \lvert \neg M_t)P(\neg M_t \lvert \neg S).
\]\\
$P(M_p \lvert \neg M_t)$ opisuje verjetnost lažno pozitivnih rezultatov laboratorijske analize in $P(M_p \lvert M_t)$ verjetnost resničnih 
pozitivnih rezultatov laboratorijske analize.\\
Za pravilno oceno verjetnosti $P(M_p \lvert \neg S)$ potrebujemo statistično oceno pogostosti profila DNK in stopnje napak
laboratorijskih analiz, ki pa so redko na voljo.\\\\
Relativno majhne stopnje napak lahko bistveno zmanjšajo dokazno vrednost DNK dokazov, saj močno zmanjšajo razmerje verjetij. Vpliv stopnje 
laboratorijskih napak kaže, da ne glede na to, kako nizka je pogostost profila, bo ta relativno nepomembna, če pogostosti ne spremlja ocena 
stopnje laboratorijskih napak. Bayesovo pravilo nam omogoča, da ta vidik upoštevamo.\\\\

%%%%%%%%%%%%%%%%%%%%%%%%%%%%%%%%%%%%%%%%%%%%%%%%%%%%%%%%%%%%%%%%%%%%%%%%%%%%%%%%%%%%%%%%%%%%%%%%%%%%%%%%%%%%%%%%%%%%%%%%%%%%%%%%%%%%%%%%%%%%%%
%%%%%%%%%%%%%%%%%%%%%%%%%%%%%%%%%%%%%%%%%%%%%%%%%%%%%%%%%%%%%%%%%%%%%%%%%%%%%%%%%%%%%%%%%%%%%%%%%%%%%%%%%%%%%%%%%%%%%%%%%%%%%%%%%%%%%%%%%%%%%%
\section{Razmerje verjetij}
Občasno se zgodi, da predloga tožilstva in obrambe nista komplementarna in v takih primerih ni mogoče določiti $P(H_p)$ ali $P(H_d)$ (poglavje 1), 
ampak samo vpliv statistike, znane kot razmerje verjetij.





\end{document}