\documentclass{beamer}
\mode<presentation>

\usepackage[utf8]{inputenc}
\usepackage[T1]{fontenc}
\usepackage[slovene]{babel}
\usepackage{lmodern}
\usepackage{array} 
\usepackage{tikz}

\usetheme{Berlin}
\usecolortheme{default}
\useinnertheme[shadows]{rounded}
\useoutertheme{infolines}
\beamertemplatenavigationsymbolsempty

\usepackage{times}
\newcommand{\ds}{\displaystyle}
\newcommand{\ts}{\textstyle}
\newcommand{\presledek}{\vspace{3mm}}
\newcommand{\ph}{\phantom{1}} 
\newcommand{\tr}{{\rm tr \,}}

\newtheorem{definicija}{Definicija}
\newtheorem{izrek}{Izrek}
\newtheorem{trditev}{Trditev}

\begin{document}

%%%%%%%%%%%%%%%%%%%%%%%%%%%%%%%%%%%%%%%%%%%%%%%%%%%%%%%%%%%%%%%%%%%%%%%%%%%%%%%%%%%%%%%%%%%%%%%%%%%%%%%%%%%%%%%%%%%%%%%%%%%%%%%%%%%%%%%%%%%%
%%%%%%%%%%%%%%%%%%%%%%%%%%%%%%%%%%%%%%%%%%%%%%%%%%%%%%%%%%%%%%%%%%%%%%%%%%%%%%%%%%%%%%%%%%%%%%%%%%%%%%%%%%%%%%%%%%%%%%%%%%%%%%%%%%%%%%%%%%%%
\title{Statistika v kazenskem pravu}
\subtitle{Predstavitev diplomske naloge}
\author[Neža Kržan]{Neža Kržan}
 
\institute[FMF]{Mentor: izred. prof. dr. Jaka Smrekar \\ Fakulteta za matematiko in fiziko \\ \vspace{10mm} Ljubljana, 7. september 2023}
\date[7. september 2023] {}

\subject{Talks}

\begin{frame}
   \titlepage
\end{frame}

%%%%%%%%%%%%%%%%%%%%%%%%%%%%%%%%%%%%%%%%%%%%%%%%%%%%%%%%%%%%%%%%%%%%%%%%%%%%%%%%%%%%%%%%%%%%%%%%%%%%%%%%%%%%%%%%%%%%%%%%%%%%%%%%%%%%%%%%%%%%
%%%%%%%%%%%%%%%%%%%%%%%%%%%%%%%%%%%%%%%%%%%%%%%%%%%%%%%%%%%%%%%%%%%%%%%%%%%%%%%%%%%%%%%%%%%%%%%%%%%%%%%%%%%%%%%%%%%%%%%%%%%%%%%%%%%%%%%%%%%%

%%%%%%%%%%%%%%%%%%%%%%%%%%%%%%%%%%%%%%%%%%%%%%%%%%%%%%%%%%%%%%%%%%%%%%%%%%%%%%%%%%%%%%%%%%%%%%%%%%%%%%%%%%%%%%%%%%%%%%%%%%%%%%%%%%%%%%%%%%%%
%%%%%%%%%%%%%%%%%%%%%%%%%%%%%%%%%%%%%%%%%%%%%%%%%%%%%%%%%%%%%%%%%%%%%%%%%%%%%%%%%%%%%%%%%%%%%%%%%%%%%%%%%%%%%%%%%%%%%%%%%%%%%%%%%%%%%%%%%%%%
\section{Statistika v kazenskem pravu}

\begin{frame}
    \frametitle{Statistika v kazenskem pravu}
    \begin{itemize}
        \item Preverjamo \textbf{teorije} in \textbf{hipoteze}.
        \item Preučujemo razmerja med dvema ali več spremenljivkami.\\ \vspace{2mm}
         \begin{definicija}
            \textit{Odvisna spremenljivka} je pojav, ki ga želi statistik preučiti, razložiti ali napovedati.
        \end{definicija}
        \begin{definicija}
            \textit{Neodvisna spremenljivka} je dejavnik ali značilnost, s katero se poskuša pojasniti ali napovedati odvisno spremenljivko.
        \end{definicija}        

        \begin{enumerate}
            \item časovno zaporedje;
            \item obstajati mora empirična povezava med odvisno in neodvisno spremenljivko;
            \item razmerje med neodvisno in odvisno spremenljivko nepristransko;
        \end{enumerate}
    \end{itemize}
\end{frame}

%%%%%%%%%%%%%%%%%%%%%%%%%%%%%%%%%%%%%%%%%%%%%%%%%%%%%%%%%%%%%%%%%%%%%%%%%%%%%%%%%%%%%%%%%%%%%%%%%%%%%%%%%%%%%%%%%%%%%%%%%%%%%%%%%%%%%%%%%%%%
\begin{frame}
   \frametitle{Težave}
   \begin{block}{}
       \centering
       Določitev odvisnih in neodvisnih spremenljivk za modeliranje.
   \end{block}\vspace{3mm}
   \textit{
   V proces določanja spremenljivk pogosto posežejo odvetniki, ki se sklicujejo na pravne zakone in načela. \\
   To lahko postane sporno, saj lahko takšni posegi ovirajo statistike pri izračunu verjetnostnega vpliva spremenljivk.}
\end{frame}

%%%%%%%%%%%%%%%%%%%%%%%%%%%%%%%%%%%%%%%%%%%%%%%%%%%%%%%%%%%%%%%%%%%%%%%%%%%%%%%%%%%%%%%%%%%%%%%%%%%%%%%%%%%%%%%%%%%%%%%%%%%%%%%%%%%%%%%%%%%%%%
%%%%%%%%%%%%%%%%%%%%%%%%%%%%%%%%%%%%%%%%%%%%%%%%%%%%%%%%%%%%%%%%%%%%%%%%%%%%%%%%%%%%%%%%%%%%%%%%%%%%%%%%%%%%%%%%%%%%%%%%%%%%%%%%%%%%%%%%%%%%%%
\section{Raziskovalni proces}
\begin{frame}
    \frametitle{Raziskovalni proces}
    Proces se izvaja po naslednjih točkah.\\ \vspace{2mm}
    \begin{enumerate}
        \item \textbf{Identifikacija problema.\\} 
        \item \textbf{Zasnova raziskave.\\} 
        \item \textbf{Analiza podatkov.\\}
    \end{enumerate}
\end{frame}

%%%%%%%%%%%%%%%%%%%%%%%%%%%%%%%%%%%%%%%%%%%%%%%%%%%%%%%%%%%%%%%%%%%%%%%%%%%%%%%%%%%%%%%%%%%%%%%%%%%%%%%%%%%%%%%%%%%%%%%%%%%%%%%%%%%%%%%%%%%%%%
%%%%%%%%%%%%%%%%%%%%%%%%%%%%%%%%%%%%%%%%%%%%%%%%%%%%%%%%%%%%%%%%%%%%%%%%%%%%%%%%%%%%%%%%%%%%%%%%%%%%%%%%%%%%%%%%%%%%%%%%%%%%%%%%%%%%%%%%%%%%%%
\section{Vrednotenje dokazov}
\begin{frame}
   \frametitle{Vrednotenje dokazov}
   \textbf{METODA DOKAZNE VREDNOSTI} - ali med dokazom in zadevno dokazno temo obstaja naključna povezava.\\ \vspace{3mm}
   \textbf{MODEL VERJETNOSTI HIPOTEZE} - kako verjetno je, da je hipoteza, za katero dokazi zagotavljajo določeno stopnjo podpore, resnična.\\ \vspace{5mm}
   \begin{beamerboxesrounded}[]{Glavna razlika}
      Model verjetnosti hipoteze predpostavlja, da obstaja začetna verjetnost za hipotezo 
      pred obravnavo dokazov.
   \end{beamerboxesrounded} \vspace{3mm}
\end{frame}

%%%%%%%%%%%%%%%%%%%%%%%%%%%%%%%%%%%%%%%%%%%%%%%%%%%%%%%%%%%%%%%%%%%%%%%%%%%%%%%%%%%%%%%%%%%%%%%%%%%%%%%%%%%%%%%%%%%%%%%%%%%%%%%%%%%%%%%%%%%%
%%%%%%%%%%%%%%%%%%%%%%%%%%%%%%%%%%%%%%%%%%%%%%%%%%%%%%%%%%%%%%%%%%%%%%%%%%%%%%%%%%%%%%%%%%%%%%%%%%%%%%%%%%%%%%%%%%%%%%%%%%%%%%%%%%%%%%%%%%%%
\section{Koncept verjetnosti}

\begin{frame}
    \frametitle{Koncept verjetnosti}
    Opravlja se primerjava verjetnosti dokazov na podlagi dveh konkurenčnih predlogov:\\
    $H_p \dots$ trditev, ki jo predlaga tožilstvo;\\
    $H_d \dots$ trditev, ki jo predlaga obramba.\\ \vspace{5mm}
   \begin{beamerboxesrounded}[]{Verjetnost proti nedolžnosti ali verjetnost za krivdo}
      \[
          \frac{P(H_p)}{P(H_d)}
      \]    
  \end{beamerboxesrounded} \vspace{3mm}
  \begin{beamerboxesrounded}[]{Verjetnost v prid krivdi ob upoštevanju informacij $E$}
      \[
          \frac{P(H_p \lvert E)}{P(H_d \lvert E)} 
      \]    
  \end{beamerboxesrounded} \vspace{5mm}
\end{frame}

%%%%%%%%%%%%%%%%%%%%%%%%%%%%%%%%%%%%%%%%%%%%%%%%%%%%%%%%%%%%%%%%%%%%%%%%%%%%%%%%%%%%%%%%%%%%%%%%%%%%%%%%%%%%%%%%%%%%%%%%%%%%%%%%%%%%%%%%%%%%
\begin{frame}
   \frametitle{Koncept verjetnosti}
   Če imamo na voljo dokaz $E$, nas zanima pogojna verjetnost
   \[
       P(kriv \lvert E), \vspace{2mm}
   \]
   pri čemer nam je lahko v pomoč Bayesovo pravilo. \vspace{3mm}
   \begin{itemize}
      \item V praksi izračun verjetnostne krivde lahko zelo zapleten;
      \item z Bayesovim pravilom lahko ocenimo verjetnost vmesnih trditev oziroma dokazov.
   \end{itemize}
\end{frame}

%%%%%%%%%%%%%%%%%%%%%%%%%%%%%%%%%%%%%%%%%%%%%%%%%%%%%%%%%%%%%%%%%%%%%%%%%%%%%%%%%%%%%%%%%%%%%%%%%%%%%%%%%%%%%%%%%%%%%%%%%%%%%%%%%%%%%%%%%%%%
%%%%%%%%%%%%%%%%%%%%%%%%%%%%%%%%%%%%%%%%%%%%%%%%%%%%%%%%%%%%%%%%%%%%%%%%%%%%%%%%%%%%%%%%%%%%%%%%%%%%%%%%%%%%%%%%%%%%%%%%%%%%%%%%%%%%%%%%%%%%
\section{Bayesova statistika}

%%%%%%%%%%%%%%%%%%%%%%%%%%%%%%%%%%%%%%%%%%%%%%%%%%%%%%%%%%%%%%%%%%%%%%%%%%%%%%%%%%%%%%%%%%%%%%%%%%%%%%%%%%%%%%%%%%%%%%%%%%%%%%%%%%%%%%%%%%%%
\begin{frame}
   \frametitle{Opredelitev}
   \begin{beamerboxesrounded}[]{Bayesova analiza}
      standardna metoda za posodabljanje verjetnosti po opazovanju več dokazov, zato je zelo primerna za obravnavo in vrednotenje 
      dokazov.
   \end{beamerboxesrounded} \vspace{3mm}
   \begin{itemize}
      \item Začnemo z nekim predhodnim prepričanjem o hipotezi in ga posodabljamo, ko se dokazi ponovno pojavijo,
      \item dobro utemeljene predhodne predpostavke.
   \end{itemize}

\end{frame}

%%%%%%%%%%%%%%%%%%%%%%%%%%%%%%%%%%%%%%%%%%%%%%%%%%%%%%%%%%%%%%%%%%%%%%%%%%%%%%%%%%%%%%%%%%%%%%%%%%%%%%%%%%%%%%%%%%%%%%%%%%%%%%%%%%%%%%%%%%%%
\begin{frame}
   \frametitle{Bayesovo pravilo}
   \textit{
      Bayesovo sklepanje temelji na Bayesovem pravilu, ki izraža verjetnost nekega dogodka z verjetnostjo dveh dogodkov in obrnjene pogojne verjetnosti.
      \vspace{5mm}
   }
   \begin{izrek}[Bayesovo pravilo]
      \[
         P(H \lvert E) = \frac{P(E \lvert H) \times P(H)}{P(E)}. \vspace{2mm}
      \]
  \end{izrek}

\end{frame}

%%%%%%%%%%%%%%%%%%%%%%%%%%%%%%%%%%%%%%%%%%%%%%%%%%%%%%%%%%%%%%%%%%%%%%%%%%%%%%%%%%%%%%%%%%%%%%%%%%%%%%%%%%%%%%%%%%%%%%%%%%%%%%%%%%%%%%%%%%%%
\begin{frame}
   \frametitle{Bayesovo posodabljanje}
   \textit{
      Bayesovo posodabljanje je logična trditev, kako se sčasoma posodabljajo apriorne oziroma predhodne verjetnosti dokazov glede na novo zbrane dokaze 
      oziroma prepričanja.
      \vspace{5mm}
   }
   \begin{definicija}[Bayesovo posodabljanje]
      Če se dogodek E zgodi ob času $t_1 > t_0$, potem je $P_1(H) = P_0(H \lvert E)$. \vspace{3mm}
  \end{definicija}
  \begin{itemize}
   \item Ob času $t_0$ dogodku H dodelimo verjetnost $P_0(H)$ - predhodna verjetnost oziroma apriorna verjetnost;
   \item zgodi se dogodek E ob času $t_1$, ki vpliva na naša prepričanja o dogodku H,
   \item apriorno verjetnost dogodka H v času $t_1$ enačimo s pogojno verjetnostjo dogodka H glede na dogodek E v času $t_0$.
  \end{itemize}
\end{frame}

%%%%%%%%%%%%%%%%%%%%%%%%%%%%%%%%%%%%%%%%%%%%%%%%%%%%%%%%%%%%%%%%%%%%%%%%%%%%%%%%%%%%%%%%%%%%%%%%%%%%%%%%%%%%%%%%%%%%%%%%%%%%%%%%%%%%%%%%%%%%
\begin{frame}
   \frametitle{Bayesova teorija v kazenskem pravu}
   Postopek posodabljanja verjetnosti tožilčeve hipoteze na podlagi predhodnih oziroma apriornih verjetnosti. \\ \vspace{3mm}
   \begin{block}{}
       \centering
       verjetnost hipoteze pred upoštevanjem določenega dokaza (dokazov) \\ \vspace{2mm}
       $\rightarrow$ \\ \vspace{3mm}
       verjetnost hipoteze po upoštevanju določenega dokaza (dokazov)
   \end{block}
\end{frame}

%%%%%%%%%%%%%%%%%%%%%%%%%%%%%%%%%%%%%%%%%%%%%%%%%%%%%%%%%%%%%%%%%%%%%%%%%%%%%%%%%%%%%%%%%%%%%%%%%%%%%%%%%%%%%%%%%%%%%%%%%%%%%%%%%%%%%%%%%%%%
\begin{frame}
   \frametitle{Predhodna verjetnost in določitev posteriorne verjetnosti}
   \begin{definicija}[Predhodna verjetnost oziroma apriorna verjetnost]
      Predhodna verjetnost, ki je uporabljena v vsaki posodobitvi verjetnosti s pomočjo Bayesove teorije, je začetna verjetnost hipoteze 
      oziroma tožilčeve domneve o obtožencu oziroma storilcu kaznivega dejanja.
  \end{definicija} \vspace{3mm}
  \begin{block}{}
   \centering
   različne metode za določitev in izračun predhodnih verjetnosti\\ \vspace{2mm}
   $\rightarrow$ \\ \vspace{3mm}
   rezultati, ki se med seboj precej razlikujejo
   \end{block}
\end{frame}

\begin{frame}
   \frametitle{Težave z določitvijo predhodne verjetnosti}
   \begin{block}{Ali naj analitiki poskušajo določiti predhodne verjetnosti in če ja, kako naj jih določijo?}
       \begin{itemize}
           \item \textbf{Nevtralno stanje} - analitiki predpostavi enake predhodne verjetnosti za vse hipoteze v primeru;
           \item Analitik naj uporabi svoje strokovno znanje za izračun apriorne verjetnosti na podlagi razpoložljivih podatkov in brez nepotrebnega vplivanja odvetnikov ali drugih udeležencev postopka;
       \end{itemize}
   \end{block}
\end{frame}

\begin{frame}
   \frametitle{Vključevanje novih dokazov}
   Nove dokaze v postopku izračuna upoštevamo. \vspace{3mm}
   \begin{block}{}
      \centering
      nov dokaz $\rightarrow$  dokaz priznan na sodišču $\rightarrow$ upoštevan v izračunih $\rightarrow$ dokaz umaknjen iz procesa
      \vspace{3mm}
   \end{block}
   \begin{itemize}
      \item Ko se določen dokaz iz sodnega procesa umakne, posodobimo vse izračune,
      \item nadaljujemo posodabljanje verjetnosti.
   \end{itemize}
\end{frame}

%%%%%%%%%%%%%%%%%%%%%%%%%%%%%%%%%%%%%%%%%%%%%%%%%%%%%%%%%%%%%%%%%%%%%%%%%%%%%%%%%%%%%%%%%%%%%%%%%%%%%%%%%%%%%%%%%%%%%%%%%%%%%%%%%%%%%%%%%%%%
%%%%%%%%%%%%%%%%%%%%%%%%%%%%%%%%%%%%%%%%%%%%%%%%%%%%%%%%%%%%%%%%%%%%%%%%%%%%%%%%%%%%%%%%%%%%%%%%%%%%%%%%%%%%%%%%%%%%%%%%%%%%%%%%%%%%%%%%%%%%
\section{Razmerje verjetij}

%%%%%%%%%%%%%%%%%%%%%%%%%%%%%%%%%%%%%%%%%%%%%%%%%%%%%%%%%%%%%%%%%%%%%%%%%%%%%%%%%%%%%%%%%%%%%%%%%%%%%%%%%%%%%%%%%%%%%%%%%%%%%%%%%%%%%%%%%%%%
\begin{frame}
   \frametitle{Opredelitev}
   Bayesov izrek v obliki razmerja verjetij
   \[
   \frac{P(H \lvert E)}{P(\bar{H} \lvert E)} = \frac{P(E \lvert H)}{P(E \lvert \bar{H})} \times \frac{P(H)}{P(\bar{H})}. \vspace{2mm}
   \]
   \begin{definicija} 
      Razmerje
      \[
          \frac{P(E \lvert H)}{P(E \lvert \bar{H})} \vspace{2mm}
      \]
       se imenuje razmerje verjetij. 
  \end{definicija}
\end{frame}

\begin{frame}
   \frametitle{Opredelitev}
   \begin{itemize}
      \item Razlika med $P(E \lvert H)$ in $P(H \lvert E)$ je bistvena.
      \item Pri proučevanju vpliva $E$ na $H$ je treba upoštevati tako verjetnost $E$, ko je $H$ resničen in ko je $H$ neresničen.
   \end{itemize}
   \begin{block}{\textbf{Zmota prenesene pogojne verjetnosti}}
      je da dogodek $E$, ki je malo verjeten, če je $\bar{H}$ resničen, pomeni dokaz v prid $H$. Da bi bilo tako, je treba dodatno
      zagotoviti, da E ni tako malo verjeten, če je H resničen. Razmerje verjetij je potem večje od 1 in verjetnost je večja od
      predhodne verjetnosti.
   \end{block}\vspace{2mm}
   razmerje verjetij $> 1 \rightarrow$  dokaz povečuje »verjetnost« krivde\\
   razmerje verjetij $< 1 \rightarrow$  dokaz zmanjšuje »verjetnost« krivde\\
\end{frame}

%%%%%%%%%%%%%%%%%%%%%%%%%%%%%%%%%%%%%%%%%%%%%%%%%%%%%%%%%%%%%%%%%%%%%%%%%%%%%%%%%%%%%%%%%%%%%%%%%%%%%%%%%%%%%%%%%%%%%%%%%%%%%%%%%%%%%%%%%%%%
\begin{frame}
   \frametitle{Razmerje verjetij v kazenskem pravu}
   Oblika Bayesovega izreka o razmerju verjetij v forenzičnem kontekstu.\\ \vspace{2mm}
   $H_p \dots$ obtoženec je resnično kriv - nadomestimo $H$;\\
   $H_d \dots$ obtoženec je resnično nedolžen - nadomestimo $\bar{H}$;\\
   $Ev \dots$ obravnavani dokaz - nadomestimo dogodek $E$;\\ \vspace{2mm}
   \begin{block}{Ob upoštevanju informacij $I$}
       \[
           \frac{P(H_p \lvert Ev, I)}{P(H_d \lvert Ev, I)} = \frac{P(Ev \lvert H_p, I)}{P(Ev \lvert H_d, I)} \times \frac{P(H_p \lvert I)}{P(H_d \lvert I)}. \vspace{2mm}
       \]
   \end{block}
\end{frame}

\begin{frame}
   \frametitle{Razmerje verjetij v kazenskem pravu}
   \begin{definicija}
       Naj bosta  $H_p$ in $H_d$ dve konkurenčni hipotezi ter $I$ informacije o ozadju. Vrednost $V$ dokaza $Ev$ je podana z
       \[
           V = \frac{P(Ev \lvert H_p, I)}{P(Ev \lvert H_d, I)},
       \]
       razmerje verjetij, ki pretvori predhodne verjetnosti
       \[
           \frac{P(H_p \lvert I)}{P(H_d \lvert I)} 
       \]
       v posteriorne verjetnosti
       \[
           \frac{P(H_p \lvert Ev, I)}{P(H_d \lvert Ev, I)}.
       \]
    \end{definicija}     
\end{frame}

%%%%%%%%%%%%%%%%%%%%%%%%%%%%%%%%%%%%%%%%%%%%%%%%%%%%%%%%%%%%%%%%%%%%%%%%%%%%%%%%%%%%%%%%%%%%%%%%%%%%%%%%%%%%%%%%%%%%%%%%%%%%%%%%%%%%%%%%%%%%
%%%%%%%%%%%%%%%%%%%%%%%%%%%%%%%%%%%%%%%%%%%%%%%%%%%%%%%%%%%%%%%%%%%%%%%%%%%%%%%%%%%%%%%%%%%%%%%%%%%%%%%%%%%%%%%%%%%%%%%%%%%%%%%%%%%%%%%%%%%%
\section{Zmote v kazenskem pravu}

\begin{frame}
   \frametitle{Zmote v kazenskem pravu}
   \begin{enumerate}
       \item \textbf{Tožilčeva zmota};
       \item \textbf{Napaka verjetnosti};
       \item \textbf{Zanemarjanje predhodnih verjetnosti};
       \item \textbf{Napaka pri številčnem preračunavanju};
       \item \textbf{Pričakovane vrednosti, ki pomenijo edinstvenost};
       \item \textbf{Zmota obrambnega odvetnika};
       \item \textbf{Napaka baze podatkov obrambnega odvetnika};
       \item \textbf{Zasliševalčeva zmota};
   \end{enumerate}
\end{frame}

%%%%%%%%%%%%%%%%%%%%%%%%%%%%%%%%%%%%%%%%%%%%%%%%%%%%%%%%%%%%%%%%%%%%%%%%%%%%%%%%%%%%%%%%%%%%%%%%%%%%%%%%%%%%%%%%%%%%%%%%%%%%%%%%%%%%%%%%%%%%
\begin{frame}
   \frametitle{Tožilčeva zmota}
   Če je $E$ dokaz in $H$ trditev,  da je obtoženi nedolžen, upoštevamo pogojne verjetnosti: \\
   $P(E \lvert H)$ \dots verjetnost resničnosti dokaz $E$, kljub temu da je obtoženi nedolžen; \\
   $P(H \lvert E)$ \dots verjetnost, da je obtoženi nedolžen kljub dokazu $E$. \\ \vspace{2mm}
   \begin{block}{Tožilčeva zmota}
      zamenjava verjetnost dokaza $E$ glede na hipotezo $H$ z verjetnostjo hipoteze $H$ glede na dokaze $E$ oziroma $P(E \lvert H)$ z $P(H \lvert E)$.
   \end{block} \vspace{2mm}
\end{frame}

\begin{frame}
   \frametitle{Zmota obrambnega odvetnika}
   \begin{block}{Zmota obrambnega odvetnika}
      se zgodi, ko se dokazi obtoženca ujemajo z dokazi kaznivega dejanja štejejo za nepomembne, ker visoka predhodna verjetnost, da obotženec ni storil kaznivega dejanja še vedno povzroči visoko verjetnost, da oseba ustreza dokazom kaznivega dejanja, kljub temu, da ni 
      vpletena v kaznivo dejanje.
   \end{block} \vspace{2mm}
\end{frame}

%%%%%%%%%%%%%%%%%%%%%%%%%%%%%%%%%%%%%%%%%%%%%%%%%%%%%%%%%%%%%%%%%%%%%%%%%%%%%%%%%%%%%%%%%%%%%%%%%%%%%%%%%%%%%%%%%%%%%%%%%%%%%%%%%%%%%%%%%%%%
%%%%%%%%%%%%%%%%%%%%%%%%%%%%%%%%%%%%%%%%%%%%%%%%%%%%%%%%%%%%%%%%%%%%%%%%%%%%%%%%%%%%%%%%%%%%%%%%%%%%%%%%%%%%%%%%%%%%%%%%%%%%%%%%%%%%%%%%%%%%
\section{Načini za izogib zmotam}

%%%%%%%%%%%%%%%%%%%%%%%%%%%%%%%%%%%%%%%%%%%%%%%%%%%%%%%%%%%%%%%%%%%%%%%%%%%%%%%%%%%%%%%%%%%%%%%%%%%%%%%%%%%%%%%%%%%%%%%%%%%%%%%%%%%%%%%%%%%%
\begin{frame}
   \frametitle{Izogib zmotam z uporabo Bayesovih omrežij}
   \begin{block}{}
       Bayesova omrežja pomagajo določiti ustrezne verjetnostne formule, ne da bi prikazali njihovo polno algebrsko obliko, in omogočajo skoraj popolno avtomatizacijo potrebnih verjetnostnih izračunov.
   \end{block} \vspace{3mm}
   \begin{enumerate}
       \item med konkurenčnimi hipotezami izberemo najverjetnejšo;
       \item izbira mora biti podprta z znanstveno utemeljeno argumentacijo;
       \item primerna so za analizo dogodka;
       \item primerno za napovedovanje verjetnosti, da je k dogodku prispeval katerikoli od več možnih znanih vzrokov;
   \end{enumerate}
   \begin{block}{}
       Prednosti Bayesovih mrež se najbolj izrazito pokažejo na zapletenih področjih z več spremenljivkami.
   \end{block}
\end{frame}

%%%%%%%%%%%%%%%%%%%%%%%%%%%%%%%%%%%%%%%%%%%%%%%%%%%%%%%%%%%%%%%%%%%%%%%%%%%%%%%%%%%%%%%%%%%%%%%%%%%%%%%%%%%%%%%%%%%%%%%%%%%%%%%%%%%%%%%%%%%%
%%%%%%%%%%%%%%%%%%%%%%%%%%%%%%%%%%%%%%%%%%%%%%%%%%%%%%%%%%%%%%%%%%%%%%%%%%%%%%%%%%%%%%%%%%%%%%%%%%%%%%%%%%%%%%%%%%%%%%%%%%%%%%%%%%%%%%%%%%%%
\section{Zaključek}

\begin{frame}
   \frametitle{Zaključek}
   \begin{enumerate}
      \item Bayesov pristop na splošno najboljši za vrednotenje dokazov.
   \end{enumerate}
\end{frame}

\begin{frame}
   \frametitle{Zaključek}
   \begin{enumerate}
      \item Bayesov pristop na splošno najboljši za vrednotenje dokazov.
      \item Za določitev in izračun prehodnih verjetnosti poskrbijo statistiki.
   \end{enumerate}
\end{frame}

\begin{frame}
   \frametitle{Zaključek}
   \begin{enumerate}
      \item Bayesov pristop na splošno najboljši za vrednotenje dokazov.
      \item Za določitev in izračun prehodnih verjetnosti poskrbijo statistiki.
      \item Težave pa nastanejo pri vključevanju novih dokazov.
   \end{enumerate}
\end{frame}

\begin{frame}
   \frametitle{Zaključek}
   \begin{enumerate}
      \item Bayesov pristop na splošno najboljši za vrednotenje dokazov.
      \item Za določitev in izračun prehodnih verjetnosti poskrbijo statistiki.
      \item Težave pa nastanejo pri vključevanju novih dokazov.
      \item Najboljši način za izogib zmotam.
   \end{enumerate}
\end{frame}

\begin{frame}
   \frametitle{Zaključek}
   \begin{enumerate}
      \item Bayesov pristop na splošno najboljši za vrednotenje dokazov.
      \item Za določitev in izračun prehodnih verjetnosti poskrbijo statistiki.
      \item Težave pa nastanejo pri vključevanju novih dokazov.
      \item Najboljši način za izogib zmotam.
      \item Bayesova omrežja - ne prikažejo polne algebrske oblike verjetnostne formule in omogočajo skoraj popolno avtomatizacijo verjetnostnih izračunov.
   \end{enumerate}
\end{frame}

\end{document}