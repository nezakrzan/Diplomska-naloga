\documentclass{beamer}
\mode<presentation>

\usepackage[utf8]{inputenc}
\usepackage[T1]{fontenc}
\usepackage[slovene]{babel}
\usepackage{lmodern}
\usepackage{array} 
\usepackage{tikz}

\usetheme{Berlin}
\usecolortheme{default}
\useinnertheme[shadows]{rounded}
\useoutertheme{infolines}
\beamertemplatenavigationsymbolsempty

\usepackage{times}
\newcommand{\ds}{\displaystyle}
\newcommand{\ts}{\textstyle}
\newcommand{\presledek}{\vspace{3mm}}
\newcommand{\ph}{\phantom{1}} 
\newcommand{\tr}{{\rm tr \,}}

\newtheorem{definicija}{Definicija}
\newtheorem{izrek}{Izrek}
\newtheorem{trditev}{Trditev}

\begin{document}

%%%%%%%%%%%%%%%%%%%%%%%%%%%%%%%%%%%%%%%%%%%%%%%%%%%%%%%%%%%%%%%%%%%%%%%%%%%%%%%%%%%%%%%%%%%%%%%%%%%%%%%%%%%%%%%%%%%%%%%%%%%%%%%%%%%%%%%%%%%%
%%%%%%%%%%%%%%%%%%%%%%%%%%%%%%%%%%%%%%%%%%%%%%%%%%%%%%%%%%%%%%%%%%%%%%%%%%%%%%%%%%%%%%%%%%%%%%%%%%%%%%%%%%%%%%%%%%%%%%%%%%%%%%%%%%%%%%%%%%%%
\title{Statistika v kazenskem pravu}
\subtitle{Predstavitev diplomske naloge}
\author[Neža Kržan]{Neža Kržan}
 
\institute[FMF]{Mentor: izred. prof. dr. Jaka Smrekar \\ Fakulteta za matematiko in fiziko \\ \vspace{10mm} Ljubljana, 7. september 2023}
\date[7. september 2023] {}

\subject{Talks}

\begin{frame}
   \titlepage
\end{frame}

%%%%%%%%%%%%%%%%%%%%%%%%%%%%%%%%%%%%%%%%%%%%%%%%%%%%%%%%%%%%%%%%%%%%%%%%%%%%%%%%%%%%%%%%%%%%%%%%%%%%%%%%%%%%%%%%%%%%%%%%%%%%%%%%%%%%%%%%%%%%
%%%%%%%%%%%%%%%%%%%%%%%%%%%%%%%%%%%%%%%%%%%%%%%%%%%%%%%%%%%%%%%%%%%%%%%%%%%%%%%%%%%%%%%%%%%%%%%%%%%%%%%%%%%%%%%%%%%%%%%%%%%%%%%%%%%%%%%%%%%%
\section{Motivacija - zmote v kazenskem pravu}

\begin{frame}
   \frametitle{Tožilčeva zmota}
   \begin{block}{Tožilčeva zmota}
      zamenjava verjetnost dokaza $E$ glede na hipotezo $H$ z verjetnostjo hipoteze $H$ glede na dokaze $E$ oziroma $P(E \lvert H)$ z $P(H \lvert E)$.
   \end{block} \vspace{2mm}
  
\end{frame}

%%%%%%%%%%%%%%%%%%%%%%%%%%%%%%%%%%%%%%%%%%%%%%%%%%%%%%%%%%%%%%%%%%%%%%%%%%%%%%%%%%%%%%%%%%%%%%%%%%%%%%%%%%%%%%%%%%%%%%%%%%%%%%%%%%%%%%%%%%%%
%%%%%%%%%%%%%%%%%%%%%%%%%%%%%%%%%%%%%%%%%%%%%%%%%%%%%%%%%%%%%%%%%%%%%%%%%%%%%%%%%%%%%%%%%%%%%%%%%%%%%%%%%%%%%%%%%%%%%%%%%%%%%%%%%%%%%%%%%%%%
\section{Statistika v kazenskem pravu}

\begin{frame}
    \frametitle{Statistika v kazenskem pravu}
    \begin{itemize}
        \item Preučujemo razmerja med dvema ali več spremenljivkami.\\ \vspace{2mm}
         \begin{definicija}
            \textit{Odvisna spremenljivka} je pojav, ki ga želi statistik preučiti, razložiti ali napovedati.
        \end{definicija}
        \begin{definicija}
            \textit{Neodvisna spremenljivka} je dejavnik ali značilnost, s katero se poskuša pojasniti ali napovedati odvisno spremenljivko.
        \end{definicija}        

        \begin{enumerate}
            \item časovno zaporedje;
            \item obstajati mora empirična povezava med odvisno in neodvisno spremenljivko;
            \item razmerje med neodvisno in odvisno spremenljivko nepristransko;
        \end{enumerate}
    \end{itemize}
\end{frame}

%%%%%%%%%%%%%%%%%%%%%%%%%%%%%%%%%%%%%%%%%%%%%%%%%%%%%%%%%%%%%%%%%%%%%%%%%%%%%%%%%%%%%%%%%%%%%%%%%%%%%%%%%%%%%%%%%%%%%%%%%%%%%%%%%%%%%%%%%%%%
\begin{frame}
   \frametitle{Težave}
   \begin{block}{}
       \centering
       Določitev odvisnih in neodvisnih spremenljivk za modeliranje.
   \end{block}\vspace{3mm}
   \textit{
   V proces določanja spremenljivk pogosto posežejo odvetniki, ki se sklicujejo na pravne zakone in načela. \\
   To lahko postane sporno, saj lahko takšni posegi ovirajo statistike pri izračunu verjetnostnega vpliva spremenljivk.}
\end{frame}

%%%%%%%%%%%%%%%%%%%%%%%%%%%%%%%%%%%%%%%%%%%%%%%%%%%%%%%%%%%%%%%%%%%%%%%%%%%%%%%%%%%%%%%%%%%%%%%%%%%%%%%%%%%%%%%%%%%%%%%%%%%%%%%%%%%%%%%%%%%%
%%%%%%%%%%%%%%%%%%%%%%%%%%%%%%%%%%%%%%%%%%%%%%%%%%%%%%%%%%%%%%%%%%%%%%%%%%%%%%%%%%%%%%%%%%%%%%%%%%%%%%%%%%%%%%%%%%%%%%%%%%%%%%%%%%%%%%%%%%%%
\section{Koncept verjetnosti}

\begin{frame}
    \frametitle{Koncept verjetnosti}
    Opravlja se primerjava verjetnosti dokazov na podlagi dveh konkurenčnih predlogov:\\
    $H_p \dots$ trditev, ki jo predlaga tožilstvo;\\
    $H_d \dots$ trditev, ki jo predlaga obramba.\\ \vspace{5mm}
   \begin{beamerboxesrounded}[]{Verjetnost proti nedolžnosti ali verjetnost za krivdo}
      \[
          \frac{P(H_p)}{P(H_d)}
      \]    
  \end{beamerboxesrounded} \vspace{3mm}
  \begin{beamerboxesrounded}[]{Verjetnost v prid krivdi ob upoštevanju informacij $E$}
      \[
          \frac{P(H_p \lvert E)}{P(H_d \lvert E)} 
      \]    
  \end{beamerboxesrounded} \vspace{5mm}
\end{frame}

%%%%%%%%%%%%%%%%%%%%%%%%%%%%%%%%%%%%%%%%%%%%%%%%%%%%%%%%%%%%%%%%%%%%%%%%%%%%%%%%%%%%%%%%%%%%%%%%%%%%%%%%%%%%%%%%%%%%%%%%%%%%%%%%%%%%%%%%%%%%
\begin{frame}
   \frametitle{Koncept verjetnosti}
   Če imamo na voljo dokaz $E$, nas zanima pogojna verjetnost
   \[
       P(kriv \lvert E), \vspace{2mm}
   \]
   pri čemer nam je lahko v pomoč Bayesovo pravilo. \vspace{3mm}
   \begin{itemize}
      \item V praksi izračun verjetnostne krivde lahko zelo zapleten;
      \item z Bayesovim pravilom lahko ocenimo verjetnost vmesnih trditev oziroma dokazov.
   \end{itemize}
\end{frame}

%%%%%%%%%%%%%%%%%%%%%%%%%%%%%%%%%%%%%%%%%%%%%%%%%%%%%%%%%%%%%%%%%%%%%%%%%%%%%%%%%%%%%%%%%%%%%%%%%%%%%%%%%%%%%%%%%%%%%%%%%%%%%%%%%%%%%%%%%%%%
%%%%%%%%%%%%%%%%%%%%%%%%%%%%%%%%%%%%%%%%%%%%%%%%%%%%%%%%%%%%%%%%%%%%%%%%%%%%%%%%%%%%%%%%%%%%%%%%%%%%%%%%%%%%%%%%%%%%%%%%%%%%%%%%%%%%%%%%%%%%
\section{Bayesova statistika}

%%%%%%%%%%%%%%%%%%%%%%%%%%%%%%%%%%%%%%%%%%%%%%%%%%%%%%%%%%%%%%%%%%%%%%%%%%%%%%%%%%%%%%%%%%%%%%%%%%%%%%%%%%%%%%%%%%%%%%%%%%%%%%%%%%%%%%%%%%%%
\begin{frame}
   \frametitle{Opredelitev}
   \begin{beamerboxesrounded}[]{Bayesova analiza}
      standardna metoda za posodabljanje verjetnosti po opazovanju več dokazov, zato je zelo primerna za obravnavo in vrednotenje 
      dokazov.
   \end{beamerboxesrounded} \vspace{3mm}
   \begin{itemize}
      \item Začnemo z nekim predhodnim prepričanjem o hipotezi in ga posodabljamo, ko se dokazi ponovno pojavijo,
      \item dobro utemeljene predhodne predpostavke.
   \end{itemize}

\end{frame}

%%%%%%%%%%%%%%%%%%%%%%%%%%%%%%%%%%%%%%%%%%%%%%%%%%%%%%%%%%%%%%%%%%%%%%%%%%%%%%%%%%%%%%%%%%%%%%%%%%%%%%%%%%%%%%%%%%%%%%%%%%%%%%%%%%%%%%%%%%%%
\begin{frame}
   \frametitle{Bayesovo pravilo}
   \textit{
      Bayesovo sklepanje temelji na Bayesovem pravilu, ki izraža verjetnost nekega dogodka z verjetnostjo dveh dogodkov in obrnjene pogojne verjetnosti.
      \vspace{5mm}
   }
   \begin{izrek}[Bayesovo pravilo]
      \[
         P(H \lvert E) = \frac{P(E \lvert H) \times P(H)}{P(E)}. \vspace{2mm}
      \]
  \end{izrek}

\end{frame}

%%%%%%%%%%%%%%%%%%%%%%%%%%%%%%%%%%%%%%%%%%%%%%%%%%%%%%%%%%%%%%%%%%%%%%%%%%%%%%%%%%%%%%%%%%%%%%%%%%%%%%%%%%%%%%%%%%%%%%%%%%%%%%%%%%%%%%%%%%%%
\begin{frame}
   \frametitle{Bayesovo posodabljanje}
   \textit{
      Bayesovo posodabljanje je logična trditev, kako se sčasoma posodabljajo apriorne oziroma predhodne verjetnosti dokazov glede na novo zbrane dokaze 
      oziroma prepričanja.
      \vspace{5mm}
   }
   \begin{definicija}[Bayesovo posodabljanje]
      Če se dogodek E zgodi ob času $t_1 > t_0$, potem je $P_1(H) = P_0(H \lvert E)$. \vspace{3mm}
  \end{definicija}
  \begin{itemize}
   \item Ob času $t_0$ dogodku H dodelimo verjetnost $P_0(H)$ - predhodna verjetnost oziroma apriorna verjetnost;
   \item zgodi se dogodek E ob času $t_1$, ki vpliva na naša prepričanja o dogodku H,
   \item apriorno verjetnost dogodka H v času $t_1$ enačimo s pogojno verjetnostjo dogodka H glede na dogodek E v času $t_0$.
  \end{itemize}
\end{frame}

%%%%%%%%%%%%%%%%%%%%%%%%%%%%%%%%%%%%%%%%%%%%%%%%%%%%%%%%%%%%%%%%%%%%%%%%%%%%%%%%%%%%%%%%%%%%%%%%%%%%%%%%%%%%%%%%%%%%%%%%%%%%%%%%%%%%%%%%%%%%
\begin{frame}
   \frametitle{Bayesova teorija v kazenskem pravu}
   Postopek posodabljanja verjetnosti tožilčeve hipoteze na podlagi predhodnih oziroma apriornih verjetnosti. \\ \vspace{3mm}
   \begin{block}{}
       \centering
       verjetnost hipoteze pred upoštevanjem določenega dokaza (dokazov) \\ \vspace{2mm}
       $\rightarrow$ \\ \vspace{3mm}
       verjetnost hipoteze po upoštevanju določenega dokaza (dokazov)
   \end{block}
\end{frame}

%%%%%%%%%%%%%%%%%%%%%%%%%%%%%%%%%%%%%%%%%%%%%%%%%%%%%%%%%%%%%%%%%%%%%%%%%%%%%%%%%%%%%%%%%%%%%%%%%%%%%%%%%%%%%%%%%%%%%%%%%%%%%%%%%%%%%%%%%%%%
\begin{frame}
   \frametitle{Predhodna verjetnost in določitev aposteriorne verjetnosti}
   \begin{definicija}[Predhodna verjetnost oziroma apriorna verjetnost]
      Predhodna verjetnost, ki je uporabljena v vsaki posodobitvi verjetnosti s pomočjo Bayesove teorije, je začetna verjetnost hipoteze 
      oziroma tožilčeve domneve o obtožencu oziroma storilcu kaznivega dejanja.
  \end{definicija} \vspace{3mm}
  \begin{block}{}
   \centering
   različne metode za določitev in izračun predhodnih verjetnosti\\ \vspace{2mm}
   $\rightarrow$ \\ \vspace{3mm}
   rezultati, ki se med seboj precej razlikujejo
   \end{block}
\end{frame}

\begin{frame}
   \frametitle{Težave z določitvijo predhodne verjetnosti}
   \begin{block}{Ali naj analitiki poskušajo določiti predhodne verjetnosti in če ja, kako naj jih določijo?}
       \begin{itemize}
           \item \textbf{Nevtralno stanje} - analitiki predpostavi enake predhodne verjetnosti za vse hipoteze v primeru;
           \item Analitik naj uporabi svoje strokovno znanje za izračun apriorne verjetnosti na podlagi razpoložljivih podatkov in brez nepotrebnega vplivanja odvetnikov ali drugih udeležencev postopka;
       \end{itemize}
   \end{block}
\end{frame}

\begin{frame}
   \frametitle{Vključevanje novih dokazov}
   Nove dokaze v postopku izračuna upoštevamo. \vspace{3mm}
   \begin{block}{}
      \centering
      nov dokaz $\rightarrow$  dokaz priznan na sodišču $\rightarrow$ upoštevan v izračunih $\rightarrow$ dokaz umaknjen iz procesa
      \vspace{3mm}
   \end{block}
   \begin{itemize}
      \item Ko se določen dokaz iz sodnega procesa umakne, posodobimo vse izračune,
      \item nadaljujemo posodabljanje verjetnosti.
   \end{itemize}
\end{frame}

%%%%%%%%%%%%%%%%%%%%%%%%%%%%%%%%%%%%%%%%%%%%%%%%%%%%%%%%%%%%%%%%%%%%%%%%%%%%%%%%%%%%%%%%%%%%%%%%%%%%%%%%%%%%%%%%%%%%%%%%%%%%%%%%%%%%%%%%%%%%
%%%%%%%%%%%%%%%%%%%%%%%%%%%%%%%%%%%%%%%%%%%%%%%%%%%%%%%%%%%%%%%%%%%%%%%%%%%%%%%%%%%%%%%%%%%%%%%%%%%%%%%%%%%%%%%%%%%%%%%%%%%%%%%%%%%%%%%%%%%%
\section{Razmerje verjetij}

%%%%%%%%%%%%%%%%%%%%%%%%%%%%%%%%%%%%%%%%%%%%%%%%%%%%%%%%%%%%%%%%%%%%%%%%%%%%%%%%%%%%%%%%%%%%%%%%%%%%%%%%%%%%%%%%%%%%%%%%%%%%%%%%%%%%%%%%%%%%
\begin{frame}
   \frametitle{Razmerje verjetij v kazenskem pravu}
   Oblika Bayesovega izreka o razmerju verjetij v forenzičnem kontekstu.\\ \vspace{2mm}
   $H_p \dots$ obtoženec je resnično kriv;\\
   $H_d \dots$ obtoženec je resnično nedolžen;\\
   $Ev \dots$ obravnavani dokaz;\\ \vspace{2mm}
   \begin{block}{Ob upoštevanju informacij $I$}
       \[
           \frac{P(H_p \lvert Ev, I)}{P(H_d \lvert Ev, I)} = \frac{P(Ev \lvert H_p, I)}{P(Ev \lvert H_d, I)} \times \frac{P(H_p \lvert I)}{P(H_d \lvert I)}. \vspace{2mm}
       \]
   \end{block}
   \begin{block}{RAZMERJE VERJETIJ}
      razmerje verjetij $> 1 \rightarrow$  dokaz povečuje »verjetnost« krivde\\
      razmerje verjetij $< 1 \rightarrow$  dokaz zmanjšuje »verjetnost« krivde\\
   \end{block}
\end{frame}

%%%%%%%%%%%%%%%%%%%%%%%%%%%%%%%%%%%%%%%%%%%%%%%%%%%%%%%%%%%%%%%%%%%%%%%%%%%%%%%%%%%%%%%%%%%%%%%%%%%%%%%%%%%%%%%%%%%%%%%%%%%%%%%%%%%%%%%%%%%%
%%%%%%%%%%%%%%%%%%%%%%%%%%%%%%%%%%%%%%%%%%%%%%%%%%%%%%%%%%%%%%%%%%%%%%%%%%%%%%%%%%%%%%%%%%%%%%%%%%%%%%%%%%%%%%%%%%%%%%%%%%%%%%%%%%%%%%%%%%%%
\section{Druge metode}

\begin{frame}
   \frametitle{Druge metode}
   \begin{block}{Frekvence}
      \begin{itemize}
         \item Relativne frekvence vedno navajajo ali predpostavljajo, da obstaja referenčni vzorec, na podlagi katerega se lahko oceni pogostost zadevnega dogodka.
         \item Relativna frekvenca lahko podpre vmesno sklepanje o moči dokazov.
      \end{itemize}
   \end{block} \vspace{2mm}
   \begin{block}{Metoda verjetnosti naključnega ujemanja}
      \begin{itemize}
         \item Izraža možnost, da bi imel naključni posameznik, ki ni povezan z obdolžencem, ustrezno lastnost npr. DNK profil.
         \item Predstavljena oziroma razumevana narobe - tožilčeva zmota.
      \end{itemize}
   \end{block}
\end{frame}

%%%%%%%%%%%%%%%%%%%%%%%%%%%%%%%%%%%%%%%%%%%%%%%%%%%%%%%%%%%%%%%%%%%%%%%%%%%%%%%%%%%%%%%%%%%%%%%%%%%%%%%%%%%%%%%%%%%%%%%%%%%%%%%%%%%%%%%%%%%%
%%%%%%%%%%%%%%%%%%%%%%%%%%%%%%%%%%%%%%%%%%%%%%%%%%%%%%%%%%%%%%%%%%%%%%%%%%%%%%%%%%%%%%%%%%%%%%%%%%%%%%%%%%%%%%%%%%%%%%%%%%%%%%%%%%%%%%%%%%%%
\section{Načini za izogib zmotam}

%%%%%%%%%%%%%%%%%%%%%%%%%%%%%%%%%%%%%%%%%%%%%%%%%%%%%%%%%%%%%%%%%%%%%%%%%%%%%%%%%%%%%%%%%%%%%%%%%%%%%%%%%%%%%%%%%%%%%%%%%%%%%%%%%%%%%%%%%%%%
\begin{frame}
   \frametitle{Izogib zmotam z uporabo Bayesovih omrežij}
   \begin{block}{}
       Bayesova omrežja pomagajo določiti ustrezne verjetnostne formule, ne da bi prikazali njihovo polno algebrsko obliko, in omogočajo skoraj popolno avtomatizacijo potrebnih verjetnostnih izračunov.
   \end{block} \vspace{3mm}
   \begin{enumerate}
       \item med konkurenčnimi hipotezami izberemo najverjetnejšo;
       \item izbira mora biti podprta z znanstveno utemeljeno argumentacijo;
       \item primerna so za analizo dogodka;
       \item primerno za napovedovanje verjetnosti, da je k dogodku prispeval katerikoli od več možnih znanih vzrokov;
   \end{enumerate}
   \begin{block}{}
       Prednosti Bayesovih mrež se najbolj izrazito pokažejo na zapletenih področjih z več spremenljivkami.
   \end{block}
\end{frame}

%%%%%%%%%%%%%%%%%%%%%%%%%%%%%%%%%%%%%%%%%%%%%%%%%%%%%%%%%%%%%%%%%%%%%%%%%%%%%%%%%%%%%%%%%%%%%%%%%%%%%%%%%%%%%%%%%%%%%%%%%%%%%%%%%%%%%%%%%%%%
%%%%%%%%%%%%%%%%%%%%%%%%%%%%%%%%%%%%%%%%%%%%%%%%%%%%%%%%%%%%%%%%%%%%%%%%%%%%%%%%%%%%%%%%%%%%%%%%%%%%%%%%%%%%%%%%%%%%%%%%%%%%%%%%%%%%%%%%%%%%
\section{Zaključek}

\begin{frame}
   \frametitle{Zaključek}
   \begin{enumerate}
      \item Bayesov pristop na splošno najboljši za vrednotenje dokazov.
   \end{enumerate}
\end{frame}

\begin{frame}
   \frametitle{Zaključek}
   \begin{enumerate}
      \item Bayesov pristop na splošno najboljši za vrednotenje dokazov.
      \item Za določitev in izračun prehodnih verjetnosti poskrbijo statistiki.
   \end{enumerate}
\end{frame}

\begin{frame}
   \frametitle{Zaključek}
   \begin{enumerate}
      \item Bayesov pristop na splošno najboljši za vrednotenje dokazov.
      \item Za določitev in izračun prehodnih verjetnosti poskrbijo statistiki.
      \item Težave pa nastanejo pri vključevanju novih dokazov.
   \end{enumerate}
\end{frame}

\begin{frame}
   \frametitle{Zaključek}
   \begin{enumerate}
      \item Bayesov pristop na splošno najboljši za vrednotenje dokazov.
      \item Za določitev in izračun prehodnih verjetnosti poskrbijo statistiki.
      \item Težave pa nastanejo pri vključevanju novih dokazov.
      \item Najboljši način za izogib zmotam.
   \end{enumerate}
\end{frame}

\begin{frame}
   \frametitle{Zaključek}
   \begin{enumerate}
      \item Bayesov pristop na splošno najboljši za vrednotenje dokazov.
      \item Za določitev in izračun prehodnih verjetnosti poskrbijo statistiki.
      \item Težave pa nastanejo pri vključevanju novih dokazov.
      \item Najboljši način za izogib zmotam.
      \item Bayesova omrežja - ne prikažejo polne algebrske oblike verjetnostne formule in omogočajo skoraj popolno avtomatizacijo verjetnostnih izračunov.
   \end{enumerate}
\end{frame}

\end{document}