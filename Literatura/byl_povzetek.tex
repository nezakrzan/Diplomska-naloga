\documentclass[a4paper,12pt]{article}
\usepackage[slovene]{babel}
\usepackage[utf8]{inputenc}
\usepackage[T1]{fontenc}
\usepackage{lmodern}
\usepackage{amsmath}
\usepackage{amsfonts}
\pagestyle{empty}

\begin{document}

\title{Bayes and the Law}
\maketitle

\begin{abstract}
    Čeprav se je v zadnjih štiridesetih letih uporaba statističnih podatkov v sodnih postopkih precej povečala, so se uporabljale predvsem 
    klasične statistične metode in ne Bayesove. Vendar naj bi se Bayesovi pristopi izognili številnim težavam klasične statistike. \\
    Članek opisuje potencialno in dejansko rabo Bayesa v pravu in pojasnjuje glavne razloge za njegov pomankljiv vpliv na pravno prakso.  Ti 
    vključujejo napačne predstave prave o Bayesovem izreku, pretirano zanašanje na uporabo razmerja verjetnosti in pomankanje sodobnih 
    računskih metod. 
\end{abstract}
\end{document}