\documentclass[12pt,a4paper]{amsart}
\usepackage[slovene]{babel}
%\usepackage[cp1250]{inputenc}
\usepackage[T1]{fontenc}
\usepackage[utf8]{inputenc}
\usepackage{amsmath,amssymb,amsfonts}
\usepackage{url}
\usepackage[normalem]{ulem}
\usepackage[dvipsnames,usenames]{color}
\usepackage{graphicx}

% Oblika strani
\textwidth 15cm
\textheight 24cm
\oddsidemargin.5cm
\evensidemargin.5cm
\topmargin-5mm
\addtolength{\footskip}{10pt}
\pagestyle{plain}
\overfullrule=15pt % oznaci predlogo vrstico

% Ukazi za matematična okolja
\theoremstyle{definition} % tekst napisan pokončno
\newtheorem{definicija}{Definicija}[section]
\newtheorem{primer}[definicija]{Primer}
\newtheorem{opomba}[definicija]{Opomba}

\renewcommand\endprimer{\hfill$\diamondsuit$}

\theoremstyle{plain} % tekst napisan poševno
\newtheorem{lema}[definicija]{Lema}
\newtheorem{izrek}[definicija]{Izrek}
\newtheorem{trditev}[definicija]{Trditev}
\newtheorem{posledica}[definicija]{Posledica}

\begin{document}

%%%%%%%%%%%%%%%%%%%%%%%%%%%%%%%%%%%%%%%%%%%%%%%%%%%%%%%%%%%%%%%%%%%%%%%%%%%%%%%%%%%%%%%%%%%%%%%%%%%%%%%%%%%%%%%%%%%%%%%%%%%%%%%%%%%%%%%%%%%%%%
\title{}
\author{Neža Kržan}
\maketitle
%%%%%%%%%%%%%%%%%%%%%%%%%%%%%%%%%%%%%%%%%%%%%%%%%%%%%%%%%%%%%%%%%%%%%%%%%%%%%%%%%%%%%%%%%%%%%%%%%%%%%%%%%%%%%%%%%%%%%%%%%%%%%%%%%%%%%%%%%%%%%%
Pri opredelitvi področja uporabe statističnih podatkov s področja kazenskega pravosodja je pomembno vedeti, kdo so pretekli in sedanji uporabniki 
ter za kakšen namen se bodo podatki uporabljali. Vprašanja se lahko nanašajo tudi na to, kdo predloži podatke, kdo jih prejme, kateri podatki se 
predložijo ter v kakšni obliki in v kakšnih časovnih presledkih. Pri obravnavi samega sistema kazenskopravne statistike je treba razmisliti tudi o 
tem, katera ključna politična vprašanja je treba vključiti v oblikovanje vsakega programa za njegovo izboljšanje.

Za dober statistični model v pravosodju, bi morali za izhodišče upoštevati naslednje vidike:
\begin{itemize}
    \item pojavnost kaznivih dejanj (resnost, trendi, struktura, drugo),
    \item profil storilcev kaznivih dejanj in njihove značilnosti,
    \item delovna obremenitev sistema (kazniva dejanja, aretacije, izreki kazni, storilci kaznivih dejanj pod nadzorom),
    \item storilci kaznivih dejanj in primeri, ki se gibljejo po sistemu,
    \item značilnosti žrtev,
    \item porabljena sredstva (človeška in finančna),
    \item korelacije kriminala, kot so gospodarski, demografski in drugi podatki,
    \item družbeni in gospodarski stroški kaznivih dejanj,
    \item odnos državljanov do kriminala in kazenskega pravosodja ter njihova zaskrbljenost v zvezi z njima
\end{itemize}
\end{document}


