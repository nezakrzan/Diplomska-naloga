\documentclass[a4paper,12pt]{article}
\usepackage[slovene]{babel}
\usepackage[utf8]{inputenc}
\usepackage[T1]{fontenc}
\usepackage{lmodern}
\usepackage{amsmath}
\usepackage{amsfonts}

\pagestyle{empty}

\newcommand{\pojem}[1]{\textsc{#1}}

\newcounter{definicija}
\newenvironment{definicija}
{
   \stepcounter{definicija}
   \begin{flushleft}
   \textbf{Definicija \arabic{definicija}: }
}
{
   \hfill $\square$
   \end{flushleft}
}

\begin{document}

%%%%%%%%%%%%%%%%%%%%%%%%%%%%%%%%%%%%%%%%%%%%%%%%%%%%%%%%%%%%%%%%%%%%%%%%%%%%%%%%%%%%%%%%%%%%%%%%%%%%%%%%%%%%%%%%%%%%%%%%%%%%%%%%%%%%%%%%%%%%%%
\title{Bayes and the Law}
\maketitle

%%%%%%%%%%%%%%%%%%%%%%%%%%%%%%%%%%%%%%%%%%%%%%%%%%%%%%%%%%%%%%%%%%%%%%%%%%%%%%%%%%%%%%%%%%%%%%%%%%%%%%%%%%%%%%%%%%%%%%%%%%%%%%%%%%%%%%%%%%%%%%
\begin{abstract}
    Čeprav se je v zadnjih štiridesetih letih uporaba statističnih podatkov v sodnih postopkih precej povečala, so se uporabljale predvsem 
    klasične statistične metode in ne Bayesove. Vendar naj bi se Bayesovi pristopi izognili številnim težavam klasične statistike. \\
    Članek opisuje potencialno in dejansko rabo Bayesa v pravu in pojasnjuje glavne razloge za njegov pomankljiv vpliv na pravno prakso.  Ti 
    vključujejo napačne predstave prave o Bayesovem izreku, pretirano zanašanje na uporabo razmerja verjetnosti in pomankanje sodobnih 
    računskih metod. 
\end{abstract}

%%%%%%%%%%%%%%%%%%%%%%%%%%%%%%%%%%%%%%%%%%%%%%%%%%%%%%%%%%%%%%%%%%%%%%%%%%%%%%%%%%%%%%%%%%%%%%%%%%%%%%%%%%%%%%%%%%%%%%%%%%%%%%%%%%%%%%%%%%%%%%
\section{Uvod}
Uporaba statistike v pravnih postopkih ima dolgo, ampak ne ugledne, zgodovine. Prvi prijavljeni primer podrobne statistične analize, ki je bila 
uporabljena kot dokaz, je bilo leta 1867 v pravnem primeru Howland. Sodišče je sicer našlo izgovore, da teh dokazov niso uporabili. \\
Zgodovinska zadržanost pri uporabi statistične analize kot primeren in veljaven dokaz, ni brezpomenska. Leta 1894 so v pravnem primeru Dreyfus 
sicer uporabili statistično analizo, ampak se je izkazala za napačno. Šele leta 1968 je bil dobro dokumentira pravni primer, v katerem je 
statistična analiza odigrala glavno vlogo. V tem primeru je še en napačen statistični argument dodatno zavrnil statistiko na sodišču. 
V primeru sta se pojavili dve napaki, ki ju lahko zasledimo še v kar nekaj primerih kasneje. \\ \\

Čeprav je v zadnjih 40 letih prišlo do precejšnega porasta uporabe statistike v sodnih postopkih, je bila njena uporaba omejena na malo primerov, 
kjer je bilo za verjetnostno sklepanje uporabljeno testiranje hipotez z uporabo $p$ - vrednosti in intervalov zaupanja. Tudi ta vrsta statistične 
analize ima omejitve. Slabe izkušnje in težave pri interpretaciji s klasično statistiko pomenijo tudi močan odpor do kakršnih koli alternativnih 
pristopov. Zlasti se ta odpor razširi na Bayesov pristop, kljub dejstvu, da je primeren za širok spekter pravnega sklepanja. Obstaja mnogo razlogov 
o družbenih in logičnih ovirah za uporabo Bayesa v sodnih postopkih ali splošnem političnem odločanju, a prevladujoči razlog je najbrž ta, da je 
večina primerov Bayesovega pristopa preveč poenostavila temelje prava, ki so bili modelirani, da so se izrčuni lahko izvajali ročno.

%%%%%%%%%%%%%%%%%%%%%%%%%%%%%%%%%%%%%%%%%%%%%%%%%%%%%%%%%%%%%%%%%%%%%%%%%%%%%%%%%%%%%%%%%%%%%%%%%%%%%%%%%%%%%%%%%%%%%%%%%%%%%%%%%%%%%%%%%%%%%%
\section{Osnove Bayesa za pravno sklepanje}
\begin{definicija}
\pojem{Hipoteza} je izjava (običajno logična), katere resnično vrednost želimo določiti, vendar je na splošno neznana - in je morda nikoli ne 
bomo vedeli z gotovostjo.
\end{definicija}

\begin{definicija}
\pojem{Alternativna hipoteza} je izjava, ki je negacija hipoteze.
\end{definicija}

\begin{definicija}
\pojem{Dokaz} je izjava, ki, če je resnićna, podpira eno ali več hipotez.
\end{definicija}

Predpostavimo, da je:
\begin{itemize}
    \item dokaz $E$ sled DNK, najdena na kraju kaznivega dejanja (zaradi poenostavitve predpostavimo, da je bilo 
    kaznivo dejanje storjeno na otoku z $10000$ ljudmi, in te predstavljajo celoten nabor možnih osumljencev);
    \item bil obtoženec aretiran in odvzet ter analiziran je bil njegov DNK.
\end{itemize}

Smer vzročno - posledične strukture je tu smiselna, saj $H$, ki je resničen(oz. neresničen), povzroči, da je $E$ resničen(oz. neresničen), 
medtem ko $E$ ne more povzročiti $H$. Sklepanje je lahko v obeh smereh. Če opazimo, da je $E$ resničen(oz. neresničen), se naše prepričanje, 
da je $H$ resničen(oz. neresničen), poveča. Prav ta zadnja vrsta sklepanja je osrednjega pomena za celotno pravno sklepanje, saj odvetniki in 
porotniki običajno neformalno uporabljajo naslednji splošno sprejeti postopek za sklepanje o dokazih:
\begin{itemize}
    \item z neko (brezpogojno) predhodno predpostavko o končni hipozei $H$ (npr. "`predpostavka nedolžen, dokler se mu ne dokaže krivda"' je enaka 
    prepričanju, da "`verjetnost, da je obtoženec kriv, ni večja kot pri katerem koli drugem človeku v populaciji"');
    \item posodobitev našega predhodnega prepričanja o hipotezi $H$, ko opazimo dokaz $E$. Pri tem posodabljanju se upošteva verjetnost dokazov.
\end{itemize}

To neformalno razmišljanje se popolnoma ujema z Bayesovim sklepanjem, kjer sta predhodna predpostavka o $H$ in verjetnost dokaza $E$ zajeti z 
verjetnostnimi tabelami(tabela 1 in tabela 2).

\begin{table}[h!]
\begin{center}
    \begin{tabular}{|c|c|} 
     \hline
     Napačno & 0,999 \\ \hline
     Pravilno & 0,001 \\ \hline
    \end{tabular}
    \caption{Tabela verjenosti za hipotezo $H$.}
    \label{table:stevila}
\end{center}
\end{table}

\begin{table}[h!]
\begin{center}
    \begin{tabular}{|c|c|c|} 
     \hline
     $H$: obtoženčev DNK je DNK iz kraja kaznivega dejanja & Napačno & Pravilno \\ \hline
     Napačno & 0,999 & 0.0 \\ \hline
     Pravilno & 0,001 & 1.0 \\ \hline
    \end{tabular}
    \caption{Verjetnostna tabela za $E \lvert H$.}
    \label{table:stevila}
\end{center}
\end{table}

To so tabele za predhodno verjetnost hipoteze $H$, zapisano kot $P(H)$ in pogojno verjetnost dokaza $E$ glede na hipotezo $H$, ki 
jo zapišemo kot $P(E \lvert H)$. Bayesova teorija doloža formulo za posodobitev našega predhodnega prepričanja o hipotezi $H$ glede na 
dokaz $E$, da dobimo naknadno prepričanje o $H$, kar zapišemo kot $P(H \lvert E)$:
\[P(H \lvert E) = \frac{P(E \lvert H)*P(H)}{P(E)} = \frac{P(E \lvert H)*P(H)}{P(E \lvert H)*P(H) + P(E \lvert \text{not}\quad H)*P(\text{not}\quad H)}.\]

Tabela 1(tabela verjenosti za $H$) zajema, da je obtoženec eden od $10000$ ljudi, ki bi lahko bili vir DNK-ja iz kraja zločina. Tabela 2 
(verjetnostna tabela za $E \lvert H$) vsebuje predpostavke, da: 
\begin{itemize}
    \item verjetnost pravilnega ujemanja sledi DNK je ena (torej ni možnosti lažno negativnega ujemanja DNK). Ta verjetnost $P(E \lvert H)$ 
          se imenuje verjetnost obtožbe za dokaz $E$;
    \item verjetnost ujemanja pri osebi, ki ni pustila svojega DNK-ja na kraju dogodka (torej "`verjetnost naključnega ujemanja DNK"'), je $1$ 
          proti $1000$. Ta verjetnost $P(E \lvert \text{not}\quad H)$ se imenuje verjetnost obrambe dokaza $E$.
\end{itemize}

S temi predpostavkami iz Bayeseve teorije sledi, da je v našem primeru posteriorno prepričanje v hipotezo $H$ po gotovitvi, da je dokaz $E$ 
resničen, približno 9\%, tj. naše prepričanje, da je otoženec vir DBK s kraja zločina, se iz verjetnosti 1 proti 10000 premakne na vrednost 9\%. 
Druga možnost je, da se naše prepričanje, da je obtoženec vir DNK s kraja zločina, premakne z vrednosti 99\% na 91\%. \\ \\

Upoštevati je potrebno, da se posterirona verjetnost, da obtoženec ni vir DNK, zelo razlikuje od verjetnosti naključnga ujemanja. Napačna 
predpostavka, da sta verjetnosti $P(\text{not}\quad H \lvert E)$ in $P(E \lvert \text{not}\quad H)$ enaki, je značilna za 
tožilčevo zmoto (imenovano tudi napako prenesenega pogoja). Tožilec lahko na primer izjavi, da je "`verjetnost, da obtoženec ni bil vir DNK 
dokaza, 1 proti 1000'", čeprav je dejansko 91-odstotna. Ta preprosta napaka verjetnostnega sklepanja je vplivala na številne primere, vendar 
se ji je vedno mogoče izogniti z osnovnim razumevanje Bayeseve teorije. Tesno povezana, vendar manj poosta napka verjetnostnega sklepanja je 
napaka obtoženca, pri kateri obramba trdi, da ker je $P(\text{not}\quad H \lvert E)$ po upoštevanu predhodnik in dokazanih dejstev 
še vedno nizka, je treb dokaze zanemariti. \\ \\

Ker je Bayeseva teorija težko razumljiva, nam enakovredna formulacija Bayeseve teorije, imenovana različica Bayesove teorije "`kvot'", 
omogoča razlago vrednosti dokaza $E$, ne da nam bi bilo potrebno upoštevati predhodno verjetnost hipoteze $H$. Natančneje, ta različica 
Bayesove teorije nam pove:\vspace{5mm}
posteriorna verjetnost $H$ je predhodna verjetnost $H$, pomnožena z razmerjem verjetnosti  
\vspace{5mm}

pri čemer je razmerje verjetnosti(LR) preprosto verjetnost obtožbe dokaza $E$, deljena z verjetnostjo obrambe dokaza $E$, tj.
\[\frac{P(H \lvert E)}{P(E \lvert \text{not} \quad H)}.\]

%%%%%%%%%%%%%%%%%%%%%%%%%%%%%%%%%%%%%%%%%%%%%%%%%%%%%%%%%%%%%%%%%%%%%%%%%%%%%%%%%%%%%%%%%%%%%%%%%%%%%%%%%%%%%%%%%%%%%%%%%%%%%%%%%%%%%%%%%%%%%%
\section{Kontekst in pregled Bayesove metode v sodnih postopkih}
Da lahko definiramo Bayesovo sklepanje v sodnih postopkih, klasificirajmo primere, kjer se je uporabilo Bayesovo sklepanje na sodušču, 
ki vključujejo:
\begin{itemize}
    \item Preverjanje hipotez o določenem "dvomljivem" vedenju.
    \item Ugotavljanje, v kolikšni meri dokaz o "lastnosti" pomaga pri identifikaciji.
    \item Uporaba forenzičnih dokazov za sklepanje o vzroku posledic.
    \item Združevanje več statističnih dokazov.
    \item Združevanje več različnih dokazov.
\end{itemize}

\subsection{Prevrejanje hipotez o določenem "dvomljivem" vedenju}
Večina poročanih primerov izrecne uporabe statističnih podatkov v sodnih postopkih spada v to klasifikacijo, ki jo nadalje delimo na:
\begin{itemize}
    \item Vse oblike diskriminacije in pristranskosti.
    \item Kakršne koli oblike goljufije/izigravanja.
    \item Posedovanje nezakonitih materialov/snovi.
\end{itemize}

Običajno se je, kadar je na voljo dovolj ustreznih podatkov, namesto Bayesovega testiranja hipotez uporabilo klasično statistično testiranje 
hipotez, da se ugotovi, ali ali je mogoče hičelno hipotezo "ni dvomljivega vedenja" zavrniti na ustrezni ravni pomembnosti, kljub težavam 
pri razlagi $p$ - vrednosti in intervalov zaupanja, ki se jim sicer z uporabo Bayesovega pristopa lahko izognemo. Možnost napačne 
razlage je ogromna in statistiki se premalo zavedajo, da sodobna orodja omogočajo enostavno izvedbo potrebnih analiz za Bayesovo 
testiranje hipotez.

\subsection{Ugotavljanje, v kolikšni meri dokaz o "lastnosti" pomaga pri identifikaciji}
Ta klasifikacija se nanaša na vse primere, v katerih se uporabljajo statistični dokazi o 'lastnostih' v najširšem pomenu. lastnosti segajo 
od forenzičnih fizičnih lasnosti, kot je DNK, prstnih odtisov, do bolj osnovnih lastnosti, kot so barva kože, višina, barva las\dots. 
Lahko se nanašajo tudi na nečloveške značilnosti, povezane z zločinom ali prizoriščem zločina, kot so oblačila in druga lastnina, 
avtomovil, orožje, steklo\dots. \\ \\

Vsaka statistična uporaba dokazov o lastnostih zahteva določeno oceno (na podlagi vzorčenja ali kako drugače) pojavnosti lastnosti v 
ustrezni populaciji. Velik del odpora proti uporabi takšnih dokazov je posledica pomislkeov glede starosti in veljavbosti teh ocen. Kljub 
temu je hitra rast forenzične statistike v zadnjih 25 letih povzročila ustrezno povečanje uporabe statističnih dokazov o lastnostih. Zato 
ni presenetljivo, da se skoraj vsa objavljena uporaba Bayesa v sodnih postopkih nanaša na to vrsto.

\subsection{Združevanje večih statističnih dokazov}
Kadar je v primer vključenih več delov statističnih dokazov, je potrebna natančna analiza ob upoštevanju morebitnih odvisnosti med 
različnimi dokazi. Bazyesova teorija je idelana za takšno analizo, a na žalost se odvisnoti med dokazi niso dobro zavedali pri 
številnih primerih. 

\subsection{Združevanje večih različnih dokazov}

\end{document}