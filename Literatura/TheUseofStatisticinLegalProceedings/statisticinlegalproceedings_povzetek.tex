\documentclass[a4paper,12pt]{article}
\usepackage[slovene]{babel}
\usepackage[utf8]{inputenc}
\usepackage[T1]{fontenc}
\usepackage{lmodern}
\usepackage{amsmath}
\usepackage{amsfonts}
\usepackage{graphicx}

\setlength{\parindent}{0mm}
\pagestyle{empty}

\newcommand{\pojem}[1]{\textsc{#1}}

\newcounter{definicija}
\newenvironment{definicija}
{
   \stepcounter{definicija}
   \begin{flushleft}
   \textbf{Definicija \arabic{definicija}: }
}
{
   \hfill $\square$
   \end{flushleft}
}

\begin{document}
\title{Uporaba statistike v pravnih postopkih}
\maketitle

\section{Kaj je statistična znanost?}
Za obravnavo različnih vprašanj v zvezi s kontekstom posameznih primerov, predloženih sodiščem, je mogoče uporabiti različna statistična in 
verjetnostna orodja. Izbira teh je odvisna od vprašanj, ki jih je potrebno obravnavati. Vprašanja, na katere lahko odgovori statistična 
znanost, lahko razvrstimo v naslednje kateogrije:

\begin{itemize}
   \item opisna statistika,
   \item sklepanje iz opzovanih podatkov na večjo populacijo,
   \item sklepanje iz opazovanih podatkov do znanstvenega zaključka,
   \item napovedovanje in
   \item vrednotenje.
\end{itemize}

Za vse te situacije je značilna negotovost, verjetnost pa zagotavlja orodja in jezik za obravnavanje in sporočanje teh negotovosti.

\subsection{Uporaba statistike in vrste dokazov}
Statistika je lahko v podporo strokovnemu znanju pri obravnavi dokazov in postopkov, ki vključujejo:
\begin{itemize}
   \item vrednotenje DNK dokazov;
   \item vrednotenje dokazov sledi;
   \item vrednotenje dokazov, ki se ujemajo z vzorci;
   \item vzorčnost bolezni ali poškodbe, kjer se uporablja epidemiološke raziskave za posamezne primere;
\end{itemize}

Postopek, v katerem se lahko uporablja statistiko v sodnih primerih in v katerem se ustrezni pretekli podatki uporabijo za 
sklepanje v trenutnem primeru.

\begin{enumerate}
   \item Identifikacija opazovanja določenega dokaza, ki je pomemben za nekatere dele trenutnega primera.
   \item Zagotovitev seznama ustreznih možnih predlogov glede dokazov, ki lahko vključujejo obtožbe tožilstva in morebitne 
         alternativne trditve obrambe.
   \item Prepoznavanje virov, ki vsebujejo ustrezne pretekle podatke ali ustvarjanje novih baz podatkov.
   \item Metode za uporabo zbranih informacij v podatkovni zbirki za izpeljavo števičnge a ali besednega izraza dokazne 
         vrednosti opažanja, povezanega z dokaznim elemntom, glede na konkurenčne predloge.
   \item Sporočanje dokazne vrednosti dokazov.
\end{enumerate}

\subsection{Obveščanje o dokazni vrednosti pri uporabi statistike}
Kadar se na podlagi podatkov oblikujejo zaključki, ki temeljijo na sttističnih izračunih, je ključnega pomena, da so te podatki 
pregledani. Namreč statistični zaključki so zanesljivi le toliko, kolikor je zanesljiv model in podatki, iz katerega so 
izpeljani. Verjetnost pa je konceptualni pripomoček, ki nam pomaga logično razmišljati in sklepati, kadar se soočamo z negotovostjo 
glede nastanka dogodka v preteklosti, sedanjosti ali preteklosti. Pomaga nam, da o negotovih dogodkih razmišljamo jasno in dosledno.

\section{Zbirke podatkov z ustreznimi preteklimi opazovanji}
Vrednotenje dokazov z uporabo razmerja verjetnosti(LR) pogosto vključuje rabo zbirk podtkov in statističnih predpostavk. Izvesti je 
potrebno validacijske teste, da se ocenijo statistične predpostavke in zagotovi, da so vrednosti LR, predstavljene na sodišču, 
zanesljive.

\subsection{Verjetnostna vrednost, izražena kor razmerje verjetnosti}
Razmerje verjetnosti je:
\[ LR = \frac{\mathbb{P}(A)}{\mathbb{P}(B)},\]
ob predpostavki, da verjetnosti dogodka $A$ in $B$ obstajata. \\ \\

Razmerje verjetnosti se največkrat uporablja pri dokazih, ki vključujejo DNK - kadar je ugotovljeno določeno ujemanje med 
osumljenčevim DNK in DNK, najdenim na kraju zločina. \\ \\

Razmerje verjetnosti ponavadi prikazujejo z besednimi ekvivalenti številkam. V večini 
primerov bo razmerje verjetnosti temeljil na polkvantitativni oceni, sodišču pa se lahko 
predloži besedni ekvivalent. Besedne zveze so številčno opredeljene, da se zagotovi doslednost pri njihovi uporabi. \\ \\

Z dobro kakovostjo in ustreznimi podatki je mogoče pripraviti oceno z uporabo verjetnostnega razmerja, ki se nanaša na dokaz. Vendar 
so razpoložljivi podatki pogosto pomankljivi ali pa jih sploh ni, v takih primerih strokovnjak oblikuje osebno mnenje na podlagi 
poznavanja procesov in osebnih izkušenj. V teh primerih so v pomoč besedni izrazi ali velikostni redi verjetnostnega razmerja, pri 
čemer morajo jasno navesti podlago za vsako tako oblikovano strokovno mnenje. \\ \\

\section{Tožilčeva zmota}
Tožilčeva zmota se pojavi, ko se verjetnost dokazov (skladen DNK, \dots) glede na nedolžnost ( verjetnost naključnega ujemanja) napačno 
razlaga kot verjetnost nedolžnosti glede na dokaz. 

\section{Zmota zagovornika}
Zmota zagovornika oziroma zmota obrambnega odvetnika se pojavi, ko se poroča o tem, koliko ljudi z določeno značilnostjo, se pojavi v 
določeni populaciji. Predpostavlja se, da je storilec del neke poljubno velike populacije in da ni na voljo drugin informacij, torej 
je za vse enako verjetno, da so storilci. Na podlagi teh predpostavk lahko sklepamo, da obstaja majhna verjetnost, da je osumljenec 
storilec kaznivega dejanja. 

\section{Združevanje dokazov}
Standardni statistični pristop je vključitev vseh dokazov v en sam izračun, pri čemer je vsaka predpostavka utežena s svojo relativno 
močjo, izraženo z razmerjem verjetnosti. Tak pristop se uporablja pri združevanju več vidikov ugotovitev forenzikov. 

\end{document}