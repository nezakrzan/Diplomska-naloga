\documentclass[a4paper,12pt]{article}
\usepackage[slovene]{babel}
\usepackage[utf8]{inputenc}
\usepackage[T1]{fontenc}
\usepackage{lmodern}
\usepackage{amsmath}
\usepackage{amsfonts}
\usepackage{enumitem}

\pagestyle{empty}
\setlength{\parindent}{0mm}



\newcommand{\pojem}[1]{\textsc{#1}}
\newcounter{definicija}
\newenvironment{definicija}
{
   \stepcounter{definicija}
   \begin{flushleft}
   \textbf{Definicija \arabic{definicija}: }
}
{
   \hfill $\square$
   \end{flushleft}
}
\newcounter{trditev}
\newenvironment{trditev}
{
   \stepcounter{trditev}
   \begin{flushleft}
   \textbf{Trditev \arabic{trditev}: }
}
{
   \end{flushleft}
}

\begin{document}

%%%%%%%%%%%%%%%%%%%%%%%%%%%%%%%%%%%%%%%%%%%%%%%%%%%%%%%%%%%%%%%%%%%%%%%%%%%%%%%%%%%%%%%%%%%%%%%%%%%%%%%%%%%%%%%%%%%%%%%%%%%%%%%%%%%%%%%%%%%%%%
\title{Bayeseva teorija v kazenskem pravu}
\maketitle

%%%%%%%%%%%%%%%%%%%%%%%%%%%%%%%%%%%%%%%%%%%%%%%%%%%%%%%%%%%%%%%%%%%%%%%%%%%%%%%%%%%%%%%%%%%%%%%%%%%%%%%%%%%%%%%%%%%%%%%%%%%%%%%%%%%%%%%%%%%%%%
Statistične dokaze v kazenskih postopkih se uporablja za vsaj tri namene:
\begin{itemize}
    \item s statističnimi podatki je mogoče odgovoriti na vprašnja o 
    identifikaciji, npr. ali je obtoženec vir sledi kaznivega dejanja ali je imles stik s krejm kaznivega dejanja;
    \item statistične podatke je mogoče uporabiti tudi za oceno, ali so določeni dogodki ali več dogodkov posledica nesreče ali namernega ravnanja;
    \item statistične podatke je mogoče uporabiti za posredno oceno skupnih količin, kadar ni na voljo neposrednega merila teh količin. 
\end{itemize}  
V tem poglavju bo opisano kako nam lahko Bayeseva teorija pomaga pri ocenjevanju dokazne vrednosti statističnih dokazov v teh treh vrstah primerov.

%%%%%%%%%%%%%%%%%%%%%%%%%%%%%%%%%%%%%%%%%%%%%%%%%%%%%%%%%%%%%%%%%%%%%%%%%%%%%%%%%%%%%%%%%%%%%%%%%%%%%%%%%%%%%%%%%%%%%%%%%%%%%%%%%%%%%%%%%%%%%%
\section{Matematična verjetnost}
\begin{definicija}
Predpostavimo, da je $P$ funkcija iz množice dogodkov $\Omega$ v realna števila; potej je $P$ verjetnostna funkcija, če izpolnjuje spodnje aksiome Kolmogorova:\\
\begin{itemize}
    \item $0 \le P(A) \le 1$ za vsak dogodek A iz množice dogodkov $\Omega$;
    \item $P(\top)=1$, pri čemer je $\top$ katerakoli logična tavtologija;
    \item $P(A \vee B) = P(A) + P(B)$, pri čemer sta $A$ in $B$ dogodka iz množice dogodkov in sta med seboj neodvisna.
\end{itemize}
\end{definicija}

\begin{definicija}
Sedaj lahko definiramo pogojno verjetnost dogodkov $A$ in $B$ iz množice dogodkov $\Omega$, kot:
\[P(A \lvert B) = \frac{P(A \wedge B)}{P(B)}.\]
\end{definicija}

Dokažemo lahko tudi nekatere trditve:
\begin{trditev}
$P(A \lvert B) = [P(A) + P(B)] - P(A \wedge B)$, če sta $A$ in $B$ neodvisna.
\end{trditev}

\begin{trditev}
Če je $A \subseteq B$, potem je $P(A) \le P(B)$.
\end{trditev}

\begin{trditev}
$P(A) = 1 - P(\neg A)$.
\end{trditev}

\begin{trditev}
$P(A) = P(A \lvert B) P(B) + P(A \lvert \neg B)P(\neg B)$.
\end{trditev}

%%%%%%%%%%%%%%%%%%%%%%%%%%%%%%%%%%%%%%%%%%%%%%%%%%%%%%%%%%%%%%%%%%%%%%%%%%%%%%%%%%%%%%%%%%%%%%%%%%%%%%%%%%%%%%%%%%%%%%%%%%%%%%%%%%%%%%%%%%%%%%
\section{Bayesova teorija}
Recimo, da imamo hipotezo $H$ in zanjo imamo nekaj dokazov $E$, sedaj pa želimo izvedeti verjetnost hipoteze $H$ glede na dokaze $E$. Bayesova 
teorija nam daje formulo:
\[P(H \lvert E) = \frac{P(E \lvert H)P(H)}{P(E)} = \frac{P(E \lvert H)P(H)}{P(E \lvert H)P(H) + P(E \lvert  \neg H)P(\neg H)} .\]

Bayesova teorija omogoča izračun verjetnosti hipoteze $H$ glede na dogodke $E$ iz treh drugih predpostavk:
\begin{enumerate}[label=(\roman*)]
    \item verjetnost hipitetze $H$ ne glede na dogodke $E$;
    \item verjetnost dokazov $E$, ki je enaka $P(E \lvert H)P(H) + P(E \lvert  \neg H)P(\neg H)$;
    \item verjetnost $P(E \lvert H)$, tj. verjetnost dogodkov $E$ glede na hipotezo $H$.
\end{enumerate}

Obstaja še ena formulacija Bayesovega pravila, ki olajša izračune, še posebaj pri analizi dokazov z DNK:
\[\frac{P(H \lvert E)}{P(\neg H \lvert E)} = \frac{P(E \lvert H)}{P(E \lvert \neg H)} * \frac{P(H)}{P(\neg H)}.\]
Z drugimi besedami:
\[ \text{pogojna verjetnost} = \text{razmerje verjetnosti} * \text{predhodna verjetnost}.\]
Pogojna verjetnost $P(H \lvert E)$ je običajno podata z $\frac{PO}{1+PO}$, kjer je $PO$ pogojne verjetnosti.


\end{document}