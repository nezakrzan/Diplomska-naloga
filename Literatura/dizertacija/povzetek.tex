\documentclass[a4paper,12pt]{article}
\usepackage[slovene]{babel}
\usepackage[utf8]{inputenc}
\usepackage[T1]{fontenc}
\usepackage{lmodern}
\usepackage{amsmath}
\usepackage{amsfonts}

\pagestyle{empty}

\newcommand{\pojem}[1]{\textsc{#1}}

\newcounter{definicija}
\newenvironment{definicija}
{
   \stepcounter{definicija}
   \begin{flushleft}
   \textbf{Definicija \arabic{definicija}: }
}
{
   \hfill $\square$
   \end{flushleft}
}

\begin{document}

%%%%%%%%%%%%%%%%%%%%%%%%%%%%%%%%%%%%%%%%%%%%%%%%%%%%%%%%%%%%%%%%%%%%%%%%%%%%%%%%%%%%%%%%%%%%%%%%%%%%%%%%%%%%%%%%%%%%%%%%%%%%%%%%%%%%%%%%%%%%%%
\title{Statistika in verjetnost v kazenskih postopkih}
\maketitle

%%%%%%%%%%%%%%%%%%%%%%%%%%%%%%%%%%%%%%%%%%%%%%%%%%%%%%%%%%%%%%%%%%%%%%%%%%%%%%%%%%%%%%%%%%%%%%%%%%%%%%%%%%%%%%%%%%%%%%%%%%%%%%%%%%%%%%%%%%%%%%
\section{Kako se statistika uporablja v kazenskem pravu}
Težko je podati jasno opredelitev, kaj mislimo s statističnimi dokazi, vendar lahko ugotovimo skupni vzorec. Izhodišče so nekateri statistični 
podatki, npr. kolikokrat se profil DNK pojavi v podatkovni zbirki. Nato se iz podatkov na podlagi statističnega modela izpelje statistična ali 
verjetnostna ocena, npr. ocena pogostosti profila DNK v populaciji se izpelje na podlagi populacijskega modela ali pa se na podlagi določenih 
predpostavk izpelje verjetnost nekega dogodka. Nazadnje se verjetnostne in statistične ocene uporabijo za sklepanje o krivdi ali nedolžnosti. \\ \\

Osrednotočimo se na tri vrste pravnih argumentov, ki se v veliki meri opirajo na statistične dokaze. Imenujemo jih argument identifikacije, 
argument nenaključnosti in argument skupne količine. Na statistične dokaze se zanašajo za tri različne namene: identifikacijo storilca 
kaznivega dejanja, oceno, ali je do dogodka prišlo po naklučju ali ne, in končno oceno skupne količine(npr. skupne količine nezakonito 
uvožene droge), kadar ni na voljo neposrednega merila skupne količine.

\subsection{Statistična identifikacija}
Argumenti statistične identfikacije:
\textbf{Prepoznavni znak.} Obstajajo dokazi, da ima storilec aznivega dejanja ali kdor koli, ki je obiskal kraj kaznivega dejanja, lastnost $F$. \\
\textbf{Ujemanje.} Obtoženec ima lastnost $F$. \\
\textbf{Statistična pogostost.} V določeni populaciji osumljenecev se lastnost $F$ pojavlja z nizko frekvenco, npr. 1 na milijardo. \\
\textbf{Verjetnostna identifikacija.} Storilec ali kdorkoli, ki je obiskal kraj kaznivega dejanja, je ista oseba kot obtoženec, ali še natančneje, 
zelo verjetno gre za isto osebo.

To je vzorec sklepanja tožilca. Obstaja tudi način za formalizacijo argumenta statistične identifikacije z Bayesovim teoremom. \\\\

\subsection{Naključje ali ni naključje?}
Kazensko pravo pogosto vsebuje argumente, ki jih lahko imenujemo argumenti, ki niso naključni. Argumente lahko shematično predstavimo 
na naslednji način:
\textbf{Dogodek.} Dogodek $E$ se je zgodil ob navzočnosti obtoženca $D$.\\
\textbf{Podobnost.} Drugi dogodki $E^{*}$, ki so podobni dogodku $E$, so se zgodili v prisotnosti obtoženca $D$.\\
\textbf{Statistična pogostost.} Ob hipotezi, da se je vse zgodilo po naključju, je niz podobnosti dogodkov, ki ga sestavljata dogodek $E$ in 
dogodki $E^{*}$ in se vsi zgodijo ob prisotnosti obtoženca $D$, statistično zelo redek.\\
\textbf{Ni naključje.} Do dogodka $E$ ni moglo priti po naključju, temveč je bil najverjetneje posledica obtoženčevega($D$) namernega ravnanja.\\\\

Obstoj podobnih dogodkov, ki so se zgodili ob navzočnosti obtoženca $D$, ima ključno vlogo v argumentaciji, vsaj retorično. Zdaj si oglejmo isto 
aragument brez koraka podobnosti:
\textbf{Dogodek.} Zgodil se je dogodek $E$.
\textbf{Statistična pogostost.} Ob hipotezi, da se je vse zgodilo po naključju, so dogoski, kot je $E$ ob prisotnosti obtoženca $D$, statistično zelo 
redki.\\
\textbf{Ni naključje.} Do dogodka $E$ ni moglo priti po naključju, temveč je bil najverjetneje posledica obtoženčevega($D$) namernega ravnanja.\\
Argument se sedaj zdi očitno napačen. V najboljšem primeru gre za statistični ali verjetnostni modus tollens: če je hipoteza o naključju resnična, potem 
je malo verjetno, da je dogodek $E$ rsničen; toda dogodek $E$ je resničen, zato je malo verjetno, da je hipoteza o naključju resnična. 
%tuki sledi tur primer Lucie de Berg, ker so bili takšni argumenti oblikovani brez posebne previdnosti in so sledile hude sodne napake, str. 125

\subsection{Ocenjevanje količin oz. argument statistične skupne količine}
Ko gre v kazenskem primeru za neko količina, zaradi katere sodiju obtožencu, in je potrebno to količino oceniti za izrek kazni ali odšodnine, uporabljajo 
statistične podatke. Sklepi, ki izhajajo iz statističnega modela, se ne uporabljajo za kazensko sodbo; ne uporabljajo se neposredno na sojenju za dokazovanje krivde 
brez utemeljenga dvoma. Uporabljajo se po obsodbi, bodisi po izreku kazni bodisi na obravnavi o povrnitvi škode.

\subsection{Povzetek}
Pregled sodne prakse je pokazal, da se odnos sodišč do statističnih dokazov razlikuje od primera do primera. Sodišča lahko izrazujo močan odpor in so zelo 
kritična, lahko pa na statistične dokaze gledajo pozitivno, tudi če so pomankljivi(primer de Berk).



\end{document}