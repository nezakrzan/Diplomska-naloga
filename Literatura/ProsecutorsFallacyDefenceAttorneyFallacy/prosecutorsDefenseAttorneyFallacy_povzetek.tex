\documentclass[12pt,a4paper]{amsart}
\usepackage[slovene]{babel}
%\usepackage[cp1250]{inputenc}
\usepackage[T1]{fontenc}
\usepackage[utf8]{inputenc}
\usepackage{amsmath,amssymb,amsfonts}
\usepackage{url}
\usepackage[normalem]{ulem}
\usepackage[dvipsnames,usenames]{color}
\usepackage{graphicx}

% Oblika strani
\textwidth 15cm
\textheight 24cm
\oddsidemargin.5cm
\evensidemargin.5cm
\topmargin-5mm
\addtolength{\footskip}{10pt}
\pagestyle{plain}
\overfullrule=15pt % oznaci predlogo vrstico

% Ukazi za matematična okolja
\theoremstyle{definition} % tekst napisan pokončno
\newtheorem{definicija}{Definicija}[section]
\newtheorem{primer}[definicija]{Primer}
\newtheorem{opomba}[definicija]{Opomba}

\renewcommand\endprimer{\hfill$\diamondsuit$}

\theoremstyle{plain} % tekst napisan poševno
\newtheorem{lema}[definicija]{Lema}
\newtheorem{izrek}[definicija]{Izrek}
\newtheorem{trditev}[definicija]{Trditev}
\newtheorem{posledica}[definicija]{Posledica}

\begin{document}

%%%%%%%%%%%%%%%%%%%%%%%%%%%%%%%%%%%%%%%%%%%%%%%%%%%%%%%%%%%%%%%%%%%%%%%%%%%%%%%%%%%%%%%%%%%%%%%%%%%%%%%%%%%%%%%%%%%%%%%%%%%%%%%%%%%%%%%%%%%%%%
\title{}
\author{Neža Kržan}
\maketitle
%%%%%%%%%%%%%%%%%%%%%%%%%%%%%%%%%%%%%%%%%%%%%%%%%%%%%%%%%%%%%%%%%%%%%%%%%%%%%%%%%%%%%%%%%%%%%%%%%%%%%%%%%%%%%%%%%%%%%%%%%%%%%%%%%%%%%%%%%%%%%%
V kazenskih postopkih pri katerih dokazi kažejo, da se obtoženec in storilec ujemata po nekih lastnostih, porota pogosto prejme statistične 
podatke o pogostosti pojavljanja ujemajoče se lastnosti. V dveh poskusih so študentje preverjali, ali lahko takše dokaze uporabijo pri presoji 
verjetne krivde osumljenca kaznivega dejanja na podlagi pisnih opisov dokazov. V poskusu 1 so bili statistični podatki o stopnji incidence 
predstavljeni kot pogojne verjetnosti ali kot odstotki; izkazalo se je, da so prvi povzročili napake v korist tožilstva, drugi pa povzročili več 
napak v korist obrambe. V poskusu 2 so bili udeleženci izpostavljeni dvema napačnima argumentoma o tem, kako razlagati statistične dokaze. Večina 
udeležencev ni zaznala napake v enem ali obeh argumentih in je izrekla sodbe, ki so bile skladne z napačno presojo. V obeh poskusih je primerjava 
sodb preiskovancev z Bayesovimi normami razkrila splošno težnjo po premajhni uporabi statističnih dokazov. Obravnavane so teoretične in pravne 
posledice teh rezultatov.

%%%%%%%%%%%%%%%%%%%%%%%%%%%%%%%%%%%%%%%%%%%%%%%%%%%%%%%%%%%%%%%%%%%%%%%%%%%%%%%%%%%%%%%%%%%%%%%%%%%%%%%%%%%%%%%%%%%%%%%%%%%%%%%%%%%%%%%%%%%%%%
\section{Uvod}
Kriminalistični laboratoriji imajo pogosto pomembno vlogo pri identifikaciji osumljencev kaznivih dejanj. Nekateri trdijo, da imajo statistični 
dokazi lahko pretiran vpliv na poroto, nekateri pa, da se premalo uporabljajo. Ta napaka je bila označena kot zmota osnovne stopnje. \\
Ker so statistični potaki osnovne stopnje podbni statističnim podatkom o stopnji incidence, je mogoče domnevati, da bodo tudi stopnje incidence 
premalo uporabljene. Med obema vrstama statističnih podatko obstajajo pomembne razlike, zaradi katerih je to posploševanje problematično. Statistika 
osnovne stopnje kaže pogostost ciljnega izida v ustrezni populaciji, medtem ko statistika pogostosti kaže pogostos lastnosti ali značilnosti, ki je zgolj 
diagnostična za ciljni izid.  \\

%%%%%%%%%%%%%%%%%%%%%%%%%%%%%%%%%%%%%%%%%%%%%%%%%%%%%%%%%%%%%%%%%%%%%%%%%%%%%%%%%%%%%%%%%%%%%%%%%%%%%%%%%%%%%%%%%%%%%%%%%%%%%%%%%%%%%%%%%%%%%%
\section{Eksperiment 1}
\subsection{Metoda}
\subsubsection{Udeleženci}
Vsi udeleženci v tem in naslednjem eksperimentu so bili prostovoljci iz univerze, ki so kot spodbudo za sodelovanje dobili dodatne kreditne točke 
pri predmetu. Udeleženci($N = 144$) so bili vodeni v skupinah po približno deset oseb na sejah, ki so trajale pol ure. Vsak udeleženec je bil 
naključno razporejen v enega od dveh eksperimentalnih pogojev.

\subsubsection{Postopek}
Ob začetku eksperimenta so udeleženci dobili pet strani dolg dokument. Na prvi strani z navodili je bilo navedeno,
\begin{enumerate}
   \item da je bil poskus zasnovan za preverjanje sposobnosti ljudi, da na podlagi dokazov, ki vključujejo verjetnost, sprejmejo razumne sklepe;
   \item da morajo udeleženci prebrati opis kazenskega primera in navesti svojo oceno verjetnosti osumljenčeve krivde z apisom odstotka med 0 in 100;
   \item da pri teh ocenah udeleženci ne smejo upoštevati koncepta razumnega dvoma in morajo navesti verjetnost, da je osumljenec res to storil, 
   namesto verjetnosti, da bi porota osumljenca obsodila.
\end{enumerate}
Vodja eksperimenta je z udeleženci pregledal ta navodila in odgovoril na morebitna vprašanja, nato pa so udeleženci prebrali opis kazenskega primera.\\
Primer je vključeval rop trgovine z alkoholnimi pijačami, ki ga je izvedel moški s smučarsko masko. Prodajalec je opisal višino, težo in oblačila roparja, 
ni pa videl njegovega obraza ali las. Policija je v bližini trgovine prijela osumljenca, ki je ustrezal prodajalčevemu opisu, vendar osumljenec ni imel 
smučarske maske in ukradenega denarja, so pa to našli v košu za smeti v bližni kraja. \\
Udeleženci so tedaj na podlagi informacij, ki so jih prejeli, ocenili verjetnost osumljenčeve krivde. Nato pa so prebrali povzetek pričanja sodnega izvedenca, 
ki je potrdil, da se vzorci osumljenčevih las mikroskopsko ne razlikujejo od las, ki so bili najdeni v smučarski maski. Strokovnjak je opisal tudi empirično 
študijo, ki je dala podatke o verjetnosti, da se dva naključno izbrana lasa različnih posameznikov ne bosta razlikovala. \\
Eksperimentalna manipulacija je bila način na podlagi katerega je strokovnjak opisal stopnjo pojavnosti ujemajočih se las. Pri pogoju pogojne verjetnosti je 
izvedenec poročal o stopnji incidence kot o pogojni verjetnosti in navedel, da je študija pokazala, da obstajata le dva odstota možnosti, da bi se obtoženčevi 
lasje ne razlikovali od las storilca, če bi bil nedolžen. Pri pogoju odstotkov in števila je izvedenec poročal, da je študija pokazala, da ima le 2 \% ljudi 
lase, ki se ne razlikujejo od las obdolženca, in navedel, da je v mestu z 1 000 000 prebivalci približno 20 000 takih posameznikov. V vsako stanje je bila 
razporejena polovica udeležencev. \\
Ko so prebrali vse dokaze, so dokončno presodili o verjetnosti osumljenčeve krivde.

\subsection{Rezultati}
\subsubsection{Napačne sodbe}
19 oseb je bilo žrtev tožolčeve zmote, saj so ocenili, da je verjetnost krivde nataonko $0,98$, kar pa je verjetnost, ki bi jo dobili, če bi od ena 
odštevli stopnjo pojavnosti ujemajoče se lastnosti. 18 oseb je bilo žrtev zmote odvetnika, ker je bila njihova končna sodba o krivdi enaka njihovi začetni sodbi 
o krvdi, kar pomeni, da vmesnim dokazom, ki bi lahko vplivali na končenga, niso pripisovali nobene teže. Preostali niso bili žrtve nobenih od teh zmot, ker so bile 
njihove končne ocene krivde višje od začetnih ocen, torej so pripisali nekaj teže vmesnim dokazom, vendar so bile nižje od $0,98$.

\end{document}


